\documentclass{article}
\usepackage[utf8]{inputenc}
\usepackage[includeheadfoot, margin=1em,headheight=2em]{geometry}
\usepackage{titling}
\geometry{a4paper, left=2cm, right=2cm, top=2cm, bottom=2cm}
\usepackage{graphicx}
\usepackage{hyperref}
\usepackage{url}
\usepackage{enumitem}
\providecommand{\versionnumber}{0.0.1}
\usepackage{array}
\usepackage[italian]{babel}
\newcolumntype{P}[1]{>{\centering\arraybackslash}p{#1}}
\renewcommand{\arraystretch}{1.5} % Default value: 1
\setlength{\droptitle}{-6em}
\usepackage{capt-of}
\usepackage{float}


%font
\usepackage[defaultfam,tabular,lining]{montserrat}
\usepackage[T1]{fontenc}
\renewcommand*\oldstylenums[1]{{\fontfamily{Montserrat-TOsF}\selectfont #1}}

%custom bold 
\usepackage[outline]{contour}
\usepackage{xcolor}
\newcommand{\custombold}{\contour{black}}

%table colors
\usepackage{color, colortbl}
\definecolor{Blue}{rgb}{0.51,0.68,0.79}
\definecolor{LightBlue}{rgb}{0.82,0.87,0.90}
\definecolor{LighterBlue}{rgb}{0.93,0.95,0.96}

%Header
\usepackage{fancyhdr, xcolor}
\pagestyle{fancy}
\let\oldheadrule\headrule% Copy \headrule into \oldheadrule
\renewcommand{\headrule}{\color{Blue}\oldheadrule}% Add colour to \headrule
\renewcommand{\headrulewidth}{0.2em}
\fancyhead[L]{Analisi comparativa ChatGPT vs Bedrock}
\fancyhead[C]{Cybersorceres}
\fancyhead[R]{versione \versionnumber}


\title{\Huge{\textbf{Analisi comparativa ChatGPT vs Bedrock}}\vspace{-1em}}

\author{CyberSorcerers Team}
\date{}

 % Imposta labelformat=empty per nascondere il prefisso "Figura X:"
 \usepackage{caption}
\captionsetup[figure]{labelformat=empty}

\begin{document}
\maketitle

\vspace{-3em}
\begin{figure}[h]
  \centering
  \includegraphics[width=6cm, height=6cm]{documenti/logo rotondo.png}
  \label{fig:immagine}
\end{figure}

\vspace{6em}
\large{

\begin{center}
    \begin{tabular}{l c c}
        \rowcolor{Blue} 
        \textbf{Informazioni sul documento} & &\\ [1 ex]
        \rowcolor{LighterBlue}
        Destinatari: & Prof Tullio Vardanega & Prof Riccardo Cardin\\ [1 ex]
        \rowcolor{LightBlue}
         & Zero12 s.r.l & \\ [1 ex]
    \end{tabular}
\end{center}}

\begin{table}[h]
\centering
\begin{tabular}{c}
\rowcolor{Blue}
\textbf{Membri del team} \\
\rowcolor{LighterBlue}
Samuele Vignotto \\
\rowcolor{LightBlue}
Giulia Dentone \\
\rowcolor{LighterBlue}
Andrea Rezzi \\
\rowcolor{LightBlue}
Giovanni Moretti \\
\rowcolor{LighterBlue}
Sabrina Caniato \\
\rowcolor{LightBlue}
Nicola Lazzarin \\
\end{tabular}
\end{table}

\newpage
\custombold{Registro dei Cambiamenti - Changelog\textsubscript{G}}

\begin{center}
\begin{tabular}{P{4em} P{6em} P{7em} P{7em} P{11em}} 
  \rowcolor{Blue}
    \custombold{Versione} & \custombold{Data} & \custombold{Autore} &
    \custombold{ Verificatore} & \custombold{Dettaglio}\\
    \rowcolor{LighterBlue}
    0.0.1 & 26/05/2024 & Giulia Dentone & Nicola Lazzarin & Creazione e impostazione del documento.
    \end{tabular}
    \end{center}



\newpage
\tableofcontents
\listoffigures
\listoftables
\newpage
\section*{Introduzione}

\subsection*{Scopo del documento}

Negli ultimi anni, l'evoluzione dell'intelligenza artificiale ha portato allo sviluppo di modelli avanzati in grado di eseguire una vasta gamma di compiti complessi. Due dei modelli più prominenti in questo campo sono ChatGPT di OpenAI e Bedrock, sviluppato da Amazon Web Services (AWS), i quali sono appunto quelli che, tramite i risultati riscontrati dallo sviluppo del nostro prodotto, andremo a comparare nel seguente documento.
Esamineremo vari aspetti come la capacità di comprensione di un prompt (sia in in linguaggio naturale sia codice), i costi, la qualità delle risposte generate e l'efficienza operativa. Attraverso un'analisi approfondita e dati empirici, cercheremo di evidenziare i punti di forza e le limitazioni di ciascun modello, offrendo un quadro completo delle loro potenzialità e delle aree in cui possono essere migliorati.


\subsection*{Riferimenti}
\subsubsection{Riferimenti normativi}
\begin{itemize}
    \item Capitolato \textbf{C7 - ChatGPT vs BedRock developer Analysis}
    \\ \\
       \href{https://github.com/CyberSorceres/CyberSorceresRepository}{https://github.com/CyberSorceres/CyberSorceresRepository} 
    \item Regolamento del progetto didattico \\ \\ \href{https://www.math.unipd.it/~tullio/IS-1/2023/Dispense/PD2.pdf} 
    {https://www.math.unipd.it/~tullio/IS-1/2023/Dispense/PD2.pdf}
\end{itemize}

\subsubsection{Riferimenti informativi}
\begin{itemize}
    \item Amazon Bedrock \\ \href{https://aws.amazon.com/it/bedrock/}{https://aws.amazon.com/it/bedrock/} \\ 
    \item API ChatGpt \\ \href{https://openai.com/index/introducing-chatgpt-and-whisper-apis/}{https://openai.com/index/introducing-chatgpt-and-whisper-apis/}
\end{itemize}





\end{document}
