\documentclass{article}
\usepackage[utf8]{inputenc}
\usepackage[includeheadfoot, margin=1em,headheight=2em]{geometry}
\usepackage{titling}
\geometry{a4paper, left=2cm, right=2cm, top=2cm, bottom=2cm}
\usepackage{graphicx}
\usepackage{enumitem}
\usepackage{hyperref}
\usepackage{array}
\usepackage[italian]{babel}
\newcolumntype{P}[1]{>{\centering\arraybackslash}p{#1}}
\renewcommand{\arraystretch}{1.5} % Default value: 1
\setlength{\droptitle}{-6em}

%font
\usepackage[defaultfam,tabular,lining]{montserrat}
\usepackage[T1]{fontenc}
\renewcommand*\oldstylenums[1]{{\fontfamily{Montserrat-TOsF}\selectfont #1}}

%custom bold 
\usepackage[outline]{contour}
\usepackage{xcolor}
\newcommand{\custombold}{\contour{black}}

%table colors
\usepackage{color, colortbl}
\definecolor{Blue}{rgb}{0.51,0.68,0.79}
\definecolor{LightBlue}{rgb}{0.82,0.87,0.90}
\definecolor{LighterBlue}{rgb}{0.93,0.95,0.96}

%Header
\usepackage{fancyhdr, xcolor}
\pagestyle{fancy}
\let\oldheadrule\headrule% Copy \headrule into \oldheadrule
\renewcommand{\headrule}{\color{Blue}\oldheadrule}% Add colour to \headrule
\renewcommand{\headrulewidth}{0.2em}
\fancyhead[L]{Presentazione RTB}
\fancyhead[C]{Cybersorceres}


\title{\Huge{\textbf{Presentazione PB}}\vspace{-1em}}
\author{CyberSorcerers Team}
\date{}
\begin{document}
\maketitle
\vspace{-3em}
\begin{figure}[h]
  \centering
  \includegraphics[width=6cm, height=6cm]{documenti/logo rotondo.png}
  \label{fig:immagine}
\end{figure}


\begin{center}
    \begin{tabular}{|l c c|}
    \hline
        \rowcolor{Blue} 
        \textbf{Informazioni sul documento} & &\\ [1 ex]
        \hline
        \rowcolor{LighterBlue}
        Destinatari: & Prf. Tullio Vardanega & Prf. Riccardo Cardin \\ [1 ex]
        \hline
        \rowcolor{LightBlue}
        G al pedice: & Consultare il Glossario & \\ [1 ex]
        \hline
    \end{tabular}
\end{center}


\begin{center}
    \begin{tabular}{|l|c|}
    \hline
        \rowcolor{Blue} 
        \textbf{Membri del gruppo} & \textbf{Numero di matricola} \\ [1 ex]
        \hline
        \rowcolor{LighterBlue}
        Giulia Dentone & 2001687\\ [1 ex]
        \hline
        \rowcolor{LightBlue}
        Samuele Vignotto & 1161712 \\ [1 ex]
        \hline
        \rowcolor{LighterBlue}
        Sabrina Caniato & 2042351\\ [1 ex]
        \hline
        \rowcolor{LightBlue}
        Giovanni Moretti & 1217655 \\ [1 ex]
        \hline
        \rowcolor{LighterBlue}
        Nicola Lazzarin & 2042376\\ [1 ex]
        \hline
        \rowcolor{LightBlue}
        Andrea Rezzi & 2034329\\ [1 ex]
        \hline
    \end{tabular}
\end{center}

\newpage

\begin{flushright}
15-04-2024
\end{flushright}
Egregi proff. Vardanega e Cardin,
\newline
il presente documento impegna il gruppo a candidarsi alla Product Baseline (PB), con la quale si intende portare a termine il progetto di Ingegneria del Software, per il
capitolato:
\newline
\begin{center}
    ChatGPT vs BedRock developer Analysis
\end{center}
proposto dall’azienda Zero12. \\
La repository contenente il prodotto è presente al seguente link:\\

La documentazione relativa al prodotto è presente al seguente link:\\

Al seguente link, ove è possibile una navigazione intuitiva ed immediata della documentazione da parte di personale esterno:
\begin{center}
    \href{https://cybersorceres.github.io/CyberSorceresRepository/}{https://cybersorceres.github.io/CyberSorceresRepository/}
\end{center}
ove è possibile visionare:
\begin{itemize}
    \item Verbali Interni ed Esterni (nel branch\textsubscript{G} “Verbali”);
    \item Norme Way of Working (nel branch\textsubscript{G} “Documentazione interna”);
    \item Piano di Qualifica, Piano di Progetto, Analisi dei Requisiti, Glossario, Specifica Tecnica e Manuale Utente (nel branch\textsubscript{G} “Documentazione esterna”).
\end{itemize}

Con la presente lettera desideriamo confermare il preventivo presentato in sede di candidatura alla revisione RTB.\\ \\
A tal proposito, siamo lieti di confermare che il costo finale per la realizzazione del progetto è di 12090 €, come precedentemente indicato. Si nota che tale budget rimanga in linea con i costi di massima prestabiliti, individuando solo un ritardo sulle tempistiche dovuto principalmente alla realizzazione del Proof of Concept\textsubscript{G} e successivamente tamponato durante l'attivita di codifica. \\ \\
La data di consegna è dunque il 28/05/2024 e comunichiamo inoltre l’intenzione di considerare concluso il progetto una volta terminata la fase di PB: per questo motivo, come stabilito dal regolamento, alleghiamo di seguito un verbale esterno che contiene la dichiarazione esplicita, da parte del proponente, di valutazione del prodotto come Minimum Viable Product (MVP), rispetto allo stato di avanzamento dimostrato al momento della candidatura. 
\newline
\newline
Cordiali saluti,
\newline
Team Cybersorcerers

\end{document}
