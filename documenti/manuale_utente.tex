\documentclass{article}
\usepackage{graphicx}
\usepackage{float}  
\usepackage[utf8]{inputenc}
\usepackage[includeheadfoot, margin=1em,headheight=2em]{geometry}
\usepackage{titling}
\usepackage{hyperref}
\geometry{a4paper, left=2cm, right=2cm, top=2cm, bottom=2cm}
\usepackage{graphicx}
\providecommand{\versionnumber}{0.1.1}
\usepackage{enumitem}
\usepackage{array}
\newcolumntype{P}[1]{>{\centering\arraybackslash}p{#1}}
\renewcommand{\arraystretch}{1.5} % Default value: 1
\setlength{\droptitle}{-6em}
\usepackage{capt-of}
\usepackage{setspace}

%font
\usepackage[defaultfam,tabular,lining]{montserrat}
\usepackage[T1]{fontenc}
\renewcommand*\oldstylenums[1]{{\fontfamily{Montserrat-TOsF}\selectfont #1}}

%custom bold 
\usepackage[outline]{contour}
\usepackage{xcolor}
\newcommand{\custombold}{\contour{black}}

%table colors
\usepackage{color, colortbl}
\definecolor{Blue}{rgb}{0.51,0.68,0.79}
\definecolor{LightBlue}{rgb}{0.82,0.87,0.90}
\definecolor{LighterBlue}{rgb}{0.93,0.95,0.96}

\usepackage{caption}
\captionsetup[figure]{labelformat=empty}

%Header
\usepackage{fancyhdr, xcolor}
\pagestyle{fancy}
\let\oldheadrule\headrule% Copy \headrule into \oldheadrule
\renewcommand{\headrule}{\color{Blue}\oldheadrule}% Add colour to \headrule
\renewcommand{\headrulewidth}{0.2em}
\fancyhead[L]{Manuale Utente}
\fancyhead[C]{Cybersorceres}
\fancyhead[R]{versione \versionnumber}

\title{\Huge{\textbf{Manuale Utente}}\vspace{-1em}}
\author{CyberSorcerers Team}
\date{}
\begin{document}
\maketitle
\vspace{-3em}
\begin{figure}[h]
  \centering
  \includegraphics[width=6cm, height=6cm]{documenti/logo rotondo.png}
  \label{fig:immagine}
\end{figure}

\vspace{6em}
\large{
\begin{center}
    \begin{tabular}{P{24em}}
        \rowcolor{Blue}
        \textbf{Membri del team:}\\
        \rowcolor{LighterBlue}
        \custombold{Sabrina Caniato}\\
        \rowcolor{LightBlue}
        \custombold{Giulia Dentone}\\
        \rowcolor{LighterBlue}
        \custombold{Nicola Lazzarin}\\
        \rowcolor{LightBlue}
        \custombold{Giovanni Moretti}\\
        \rowcolor{LighterBlue}
        \custombold{Andrea Rezzi}\\
        \rowcolor{LightBlue}
        \custombold{Samuele Vignotto}\\
    \end{tabular}
\end{center}

\begin{center}
    \begin{tabular}{l c c}
        \rowcolor{Blue} 
        \textbf{Informazioni sul documento} & &\\ [1 ex]
        \rowcolor{LighterBlue}
        Destinatari: & Prof Tullio Vardanega & Prof Riccardo Cardin\\ [1 ex]
        \rowcolor{LightBlue}
        G al pedice: & Consultare il Glossario & \\ [1 ex]
    \end{tabular}
\end{center}
    
\newpage

\textbf{Registro dei Cambiamenti - Changelog\textsubscript{G}}
\begin{center}
\begin{tabular}{P{4em} P{6em} P{8em} P{8em} P{10em}}
    \rowcolor{LighterBlue}
    0.1.1 & 09/05/2024 & Giulia Dentone & Samuele Vignotto & Aggiunta delle tabelle e compilazione della sezione di Supporto Tecnico.\\
    \rowcolor{LightBlue}
    0.0.1 & 03/05/2024 & Giulia Dentone & Samuele Vignotto &  Definizione struttura del documento e scheletro delle sezioni. Scrittura introduzione ed obiettivi delle diverse sezioni.\\
\end{tabular}
\end{center}
\newpage
\tableofcontents
\newpage

\section{Introduzione}
\subsection{Scopo del documento}
Questo documento ha lo scopo di spiegare come utilizzare l'applicazione e illustrare le sue funzionalità. Fornisce informazioni sugli elementi essenziali necessari per far funzionare correttamente l'applicazione, spiega come installarla localmente e fornisce indicazioni su come utilizzarla in modo efficace.
\subsection{Scopo del prodotto}
L'azienda proponente ha richiesto la creazione di una web app\textsubscript{G} che, tramite l'uso di IA\textsubscript{G} (in questo caso ChatGPT4 e Bedrock) è in grado di creare epic user stories\textsubscript{G} a partire dalle richieste del cliente e confrontarle con il codice sviluppato in modo da informare il cliente dello stato di avanzamento dello sviluppo del prodotto. Inoltre deve essere possibile, sia per il Project Manager\textsubscript{G}, sia per il cliente rilasciare dei feedback (nel primo caso riguardanti l'adeguatezza delle stories, nel secondo caso riguardanti il prodotto finale) al fine di migliorare l'IA\textsubscript{G}. È inoltre richiesta un' analisi comparativa tra le due IA\textsubscript{G} utilizzate e lo sviluppo di un plug-in\textsubscript{G} utile agli sviluppatori e al Project Manager\textsubscript{G}.

\subsection{Glossario}
Alcuni termini presenti nel documento potrebbero essere ambigui, pertanto verranno inseriti nel Glossario v.1.0.0. La loro presenza all'interno di esso sarà indicata tramite una G maiuscola a pedice.

\section{Riferimenti}
\subsection{Riferimenti normativi}
\begin{itemize}
    \item Capitolato \textbf{C7 - ChatGPT vs BedRock developer Analysis}
    \\ \\
       \href{https://github.com/CyberSorceres/CyberSorceresRepository}{https://github.com/CyberSorceres/CyberSorceresRepository} 
    \item Norme del way of working v 1.0.0
    \item Regolamento del progetto didattico \\ \\ \href{https://www.math.unipd.it/~tullio/IS-1/2023/Dispense/PD2.pdf} 
    {https://www.math.unipd.it/~tullio/IS-1/2023/Dispense/PD2.pdf}
\end{itemize}
\subsection{Riferimenti informativi}
\begin{itemize}
    \item Slide del corso di Ingegneria del Software - Analisi dei requisiti \\ \\
    \href{https://www.math.unipd.it/~tullio/IS-1/2023/Dispense/T5.pdf}{https://www.math.unipd.it/~tullio/IS-1/2023/Dispense/T5.pdf}
    \item Slide del corso di Ingegneria del Software - Progettazione e programmazione: Diagrammi delle classi \\ \\
\href{https://www.math.unipd.it/~rcardin/swea/2023/Diagrammi%20delle%20Classi.pdf}{https://www.math.unipd.it/~rcardin/swea/2023/Diagrammi\%20delle\%20Classi.pdf}
\end{itemize}
\subsection{Riferimenti tecnici}
\begin{itemize}
\item Documentazione di React \\ \href{ https://react.dev/}{ https://react.dev/}
\item Documentazione di Typescript \\ \href{https://www.typescriptlang.org/docs/}{https://www.typescriptlang.org/docs/}
\item Documentazione di MongoDB \\ \href{https://www.mongodb.com/docs/}{https://www.mongodb.com/docs/}
\item Documentazione di Amazon AWS \\ \href{https://docs.aws.amazon.com/it_it/}{https://docs.aws.amazon.com/it\_it/}
\end{itemize}

\section{Requisiti}
Questa sezione fornisce un elenco dei requisiti minimi richiesti per eseguire l'applicazione, descrivendo le caratteristiche necessarie per configurare l'ambiente di sviluppo del progetto.
\subsection{Requisiti di sistema}
Affinché l'installazione e l'avvio del prodotto avvengano senza problemi e per garantire un'esperienza d'uso soddisfacente e completa dell'applicazione, è fondamentale installare i seguenti software\textsubscript{G}.
\begin{center}
\begin{tabular}{c|c|c}
\hline
\rowcolor{Blue}
Componente & Versione & Riferimenti per il download \\
\rowcolor{LighterBlue}

\rowcolor{LightBlue}
\end{tabular}
\end{center}

\subsection{Requisiti hardware}
Poiché l'applicazione viene eseguita tramite browser\textsubscript{G}, non sono definiti requisiti specifici da parte del proponente, del capitolato o del progetto stesso. Di conseguenza, i seguenti requisiti sono indicati come linee guida generali per l'esecuzione del prodotto creato.
\begin{center}
\begin{tabular}{c|c}
\hline
\rowcolor{Blue}
Componente & Requisito \\
\rowcolor{LighterBlue}

\rowcolor{LightBlue}
\end{tabular}
\end{center}

\subsection{Requisiti software}
L'applicazione è stata testata sui browser\textsubscript{G} più comuni, utilizzando le versioni iniziali come punto di partenza per lo sviluppo del progetto. Man mano che il progetto avanzava, sono state considerate progressivamente anche le versioni più recenti dei singoli browser\textsubscript{G}.
\begin{center}
\begin{tabular}{c|c}
\hline
\rowcolor{Blue}
Browser & Versione \\
\rowcolor{LighterBlue}

\rowcolor{LightBlue}
\end{tabular}
\end{center}

\section{Installazione}
\subsection{Clonazione della repository}
\subsection{Avvio dell'applicazione}

\section{Istruzioni di utilizzo}
\section{Supporto Tecnico}
Per eventuali malfunzionamenti durante l’utilizzo dell’applicazione, si prega di contattare il supporto tecnico inviando una mail all’indirizzo:
\begin{center}
    \textbf{cybersorcerers23@gmail.com}
\end{center}
Si prega di utilizzare il seguente formato per la mail per evitare l'accidentale mancata visione delle richieste di supporto:
\begin{itemize}
    \item Oggetto: “nome dell’evento da segnalare”;
    \item Corpo;
    \item Data in cui si è riscontrato il malfunzionamento;
    \item Descrizione del malfunzionamento in esame;
    \item Sistema operativo e browser in cui si è verificato il problema;
    \item Allegato (facoltativo): immagini utili per la descrizione del malfunzionamento.
\end{itemize}


\end{document}
