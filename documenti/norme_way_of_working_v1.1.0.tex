\documentclass{article}
\usepackage[utf8]{inputenc}
\usepackage[includeheadfoot, margin=1em,headheight=2em]{geometry}
\usepackage{titling}
\geometry{a4paper, left=2cm, right=2cm, top=2cm, bottom=2cm}
\usepackage{graphicx}
\usepackage{enumitem}
\providecommand{\versionnumber}{1.0.0}
\usepackage{hyperref}
\usepackage{array}
\usepackage{hyperref}
\newcolumntype{P}[1]{>{\centering\arraybackslash}p{#1}}
\renewcommand{\arraystretch}{1.5} % Default value: 1
\setlength{\droptitle}{-6em}



%font
\usepackage[defaultfam,tabular,lining]{montserrat}
\usepackage[T1]{fontenc}
\renewcommand*\oldstylenums[1]{{\fontfamily{Montserrat-TOsF}\selectfont #1}}

%custom bold 
\usepackage[outline]{contour}
\usepackage{xcolor}
\newcommand{\custombold}{\contour{black}}

%table colors
\usepackage{color, colortbl}
\definecolor{Blue}{rgb}{0.51,0.68,0.79}
\definecolor{LightBlue}{rgb}{0.82,0.87,0.90}
\definecolor{LighterBlue}{rgb}{0.93,0.95,0.96}

%Header
\usepackage{fancyhdr, xcolor}
\pagestyle{fancy}
\let\oldheadrule\headrule% Copy \headrule into \oldheadrule
\renewcommand{\headrule}{\color{Blue}\oldheadrule}% Add colour to \headrule
\renewcommand{\headrulewidth}{0.2em}
\fancyhead[L]{Norme del Way of Working}
\fancyhead[C]{Cybersorceres}
\fancyhead[R]{versione \versionnumber}

\title{\Huge{\textbf{Norme di progetto}}\vspace{-1em}}
\author{CyberSorcerers Team}
\date{}
\begin{document}
\maketitle
\vspace{-3em}
\begin{figure}[h]
  \centering
  \includegraphics[width=6cm, height=6cm]{logo rotondo.png}
  \label{fig:immagine}
\end{figure}

\vspace{6em}
\large{
\begin{center}
    \begin{tabular}{P{24em}}
        \rowcolor{Blue}
        \textbf{Membri del team:}\\
        \rowcolor{LighterBlue}
        \custombold{Sabrina Caniato}\\
        \rowcolor{LightBlue}
        \custombold{Giulia Dentone}\\
        \rowcolor{LighterBlue}
        \custombold{Nicola Lazzarin}\\
        \rowcolor{LightBlue}
        \custombold{Giovanni Moretti}\\
        \rowcolor{LighterBlue}
        \custombold{Andrea Rezzi}\\
        \rowcolor{LightBlue}
        \custombold{Samuele Vignotto}\\

    \end{tabular}
\end{center}}
\begin{center}
    \begin{tabular}{l c c}
        \rowcolor{Blue} 
        \textbf{Informazioni sul documento} & &\\ [1 ex]
        \rowcolor{LighterBlue}
        Destinatari: & Prf. Tullio Vardanega & Prf. Riccardo Cardin\\ [1 ex]
    \end{tabular}
\end{center}

\custombold{Registro dei Cambiamenti - Changelog}

\begin{center}
\begin{tabular}{P{5em} P{5em} P{8em} P{8em} P{10em}} 
  \rowcolor{Blue}
    \custombold{Versione} & \custombold{Data} & \custombold{Autore} &
    \custombold{ Verificatore} & \custombold{Dettaglio}\\
    \rowcolor{LighterBlue}
    &  &  &  & \\
    \rowcolor{LightBlue}
    & &  & & \\ 
     & &  &  &\\ 
\end{tabular}
\end{center}
\newpage
\tableofcontents
\newpage

\section{Introduzione}

\subsection{Scopo del documento}
Lo scopo di questo documento, in continuo aggiornamento, è quello di raggruppare in un unico luogo tutte le decisioni prese dal gruppo, per quanto riguarda il proprio \textit{Way of Working}\textsubscript{G}.

\subsection{Scopo del capitolato}
Lo scopo è creare un middleware\textsubscript{G} che riceva in input dei requisiti di business\textsubscript{G} e produca epic e user stories\textsubscript{G} associate ai requisiti di business tramite ChatGPT e AWS BedRock, inoltre è richiesto che venga creato un plugin\textsubscript{G} per VisualStudio Code.
Sarà necessario comparare la capacità di ChatGPT e quella di AWS BedRock nell'interpretare del codice sorgente ed associare le user stories\textsubscript{G} generate. Provare dall'interpretazione dei criteri di accettazione delle user stories\textsubscript{G} ed il codice analizzato se il risultano dei test non gestiti.

\subsection{Glossario}
I termini impiegati in questo testo potrebbero suscitare incertezze circa il loro significato, rendendo quindi necessaria una definizione per evitare ambiguità. Tali termini sono identificati da una lettera "G" maiuscola posta in pedice alla parola, e la loro spiegazione è fornita nel Glossario v1.0.0.

\subsection{Riferimenti}

\subsubsection{Riferimenti normativi}
\begin{itemize}
    \item \textbf{Regolamento del progetto didattico:}
\end{itemize}
\href{https://www.math.unipd.it/~tullio/IS-1/2023/Dispense/PD2.pdf}{PD2.pdf}
\begin{itemize}
    \item \textbf{Capitolato d'appalto C7 - ChatGPT vs BedRock developer Analysis}
\end{itemize}
\href{https://www.math.unipd.it/~tullio/IS-1/2023/Progetto/C7.pdf}{C7.pdf}

\subsubsection{Riferimenti formativi}
\begin{itemize}
    \item \textbf{Documentazione Amazon BedRock}
\end{itemize}
\href{https://docs.aws.amazon.com/bedrock/latest/userguide/what-is-bedrock.html}{Documentazione BedRock}
\begin{itemize}
    \item \textbf{Documentazione OpenAI ChatGPT}
\end{itemize}
\href{https://platform.openai.com/docs/introduction}{Documentazione ChatGPT}
\begin{itemize}
    \item \textbf{Standard ISO/IEC 12207}
\end{itemize}
\href{https://www.math.unipd.it/~tullio/IS-1/2009/Approfondimenti/ISO_12207-1995.pdf}{ISO/IEC 12207-1995}

\section{Processi primari}

\subsection{Fornitura}

\subsubsection{Scopo}
In questa sezione sono indicati tutti i parametri, gli strumenti e i documenti impiegati per portare a termine il processo di fornitura.

\subsubsection{Aspettative}
Le aspettative relative all'implementazione del processo di fornitura comprendono:
\begin{itemize}
    \item Ottenere una struttura documentale chiara;
    \item Definire i tempi di lavoro;
    \item Risolvere eventuali dubbi con il proponente;
    \item Stabilire vincoli con il proponente;
\end{itemize}

\subsubsection{Descrizione}
Il processo di fornitura identifica ogni compito, attività e risorsa necessaria per l'implementazione del progetto. Questo processo sarà avviato solo dopo la comprensione delle richieste del proponente, seguito da uno studio di fattibilità delle richieste e concluso con la definizione di un accordo contrattuale. Le fasi del processo di fornitura includono:
\begin{itemize}
    \item Avvio;
    \item Contrattazione;
    \item Pianificazione;
    \item Esecuzione;
    \item Controllo;
    \item Revisione;
    \item Valutazione;
    \item Consegna;
    \item Completamento;
\end{itemize}

\subsubsection{Proponente}
Il team di sviluppo ha concordato con il proponente di mantenere un contratto regolare per monitorare l'andamento positivo del progetto. Questo viene realizzato attraverso incontri periodici organizzati e un costante scambio asincrono tramite un canale dedicato sulla piattaforma Slack. Il team di sviluppo mira a mantenere un contatto costante con il proponente per discutere i seguenti argomenti:
\begin{itemize}
    \item Vincoli e requisiti obbligatori.
    \item Feedback riguardante le tecnologie utilizzate;
    \item Valutazione delle soluzioni innovative proposte dal team di sviluppo;
    \item Chiarimenti su eventuali dubbi;
    \item Feedback riguardante la documentazione redatta;
    \item Stima dei costi;
\end{itemize}

\subsubsection*{Strumenti}
Di seguito vengono indicati gli strumenti utilizzati dal team per la realizzazione del processo di fornitura:\begin{itemize}
    \item \textit{Microsoft Excel}: utilizzato per creare grafici, eseguire calcoli e per realizzare tabelle;
    \item \textit{Project di Github}: utilizzato per gestire le \textbf{task}.
    \item \textit{draw.io}: utilizzato per la realizzazione dei diagrammi UML.
\end{itemize}



\section{Norme}
\subsection{Gestione dell'ITS e delle task}
Per il progetto verrà utilizzato l'\textit{issue tracking system} di Github. Le task\textsubscript{G} vengono decise e assegnate durante le riunioni interne. Oltre al responsabile della task\textsubscript{G}, al quale viene assegnata tramite l'apposita \textit{feature}, viene segnato nella prima riga della descrizione il verificatore.\\
Quando la task\textsubscript{G} è completata, viene creata una \textit{pull-request}, che andrà approvata dal verificatore.\\
Lo stato degli \textit{issues} è tracciato in un \textit{project} di Github.

\subsection{File di documentazione}
Tutti i file di documentazione vengono posti nella \textit{repository}\textsubscript{G} in formato PDF. I \textit{template}\textsubscript{G} dei file vengono invece posti nell'apposita cartella in formato tex. Inoltre in ogni documento sarà presente un Changelog\textsubscript{G} in modo tale da tracciare la stesura della documentazione.

\subsection{Numero di versione}
Ogni file ha un numero di versione\textsubscript{G} posto in coda al proprio nome. Il numero di versione\textsubscript{G} è di tipo x.y.z. Il cambiamento di ogni cifra è determinato dai seguenti criteri:
\begin{itemize}[noitemsep]
    \item \custombold{Cambiamento di \textbf{x}} $\rightarrow$ indica una modifica sostanziale
    \item \custombold{Cambiamento di \textbf{y}} $\rightarrow$ indica l'aggiunta di una nuova feature\textsubscript{G}
    \item \custombold{Cambiamento di \textbf{z}} $\rightarrow$ indica una modifica minore
\end{itemize}

\subsection{Politica di decisioni}
Le decisioni ufficiali devono essere prese quando tutti i membri del gruppo sono presenti in modo tale da mantenere coerenza. In caso di impossibilità di presenza di tutti i membri del gruppo, deve essere comunque raggiunta una quota di maggioranza.

\subsection{Incontri successivi}
Ad ogni incontro viene fissata la data del successivo.

\section{Processi organizzativi}
\subsection{Formazione}
Il processo di formazione è di fondamentale importanza per garantire che tutti i membri del team acquisiscano le competenze necessarie per svolgere con successo le attività previste dal progetto. La formazione include sessioni di apprendimento relative agli strumenti utilizzati, alle tecnologie adottate e ai processi implementati. L'obiettivo è assicurare una conoscenza approfondita e uniforme all'interno del team. Di seguito, alcune documentazioni utilizzate per la formazione:
\begin{itemize}
    \item GitHub \href{https://docs.github.com/}{https://docs.github.com/}
    \item Jira \href{https://confluence.atlassian.com/jira}{https://confluence.atlassian.com/jira}
    \item Git \href{https://docs.github.com/en/get-started/using-git/about-git}{https://docs.github.com/en/get-started/using-git/about-git}
    \item Amazon AWS \href{https://aws.amazon.com/it/getting-started/hands-on/build-web-app-s3-lambda-api-gateway-dynamodb/}{https://aws.amazon.com/it/getting-started/hands-on/build-web-app-s3-lambda-api-gateway-dynamodb/}
\end{itemize}

\subsection{Impiego delle infrastrutture interne}
L'utilizzo delle infrastrutture interne,è cruciale per mantenere un ambiente di lavoro organizzato ed efficiente. Il team sfrutta a pieno queste risorse per tracciare lo stato delle attività, gestire le modifiche al codice e facilitare la collaborazione tra i membri. Gli strumenti utilizzati sono i seguenti:
\begin{itemize}
    \item Github: issue tracking system largamente utilizzato, affidabile e intuitivo
    \item Telegram: comunicare velocemente tra membri tramite chat
    \item Google Meet: per videochiamarci e fare riunioni interne ed esterne da remoto, evitando l'ostacolo della distanza tra i membri
    \item Slack: comunicare rapidamente col proponente
    \item Google Calendar: segnare le milestone e gli obiettivi principali, oltre alle riunioni
    \item Gmail: creare account sulle varie piattaforme, comunicare con i professori del corso a nome del gruppo e in una prima fase anche comunicare con il proponente
    
\end{itemize}

\subsection{Gestione organizzativa}
La gestione organizzativa del progetto coinvolge la pianificazione, la distribuzione delle responsabilità e la gestione degli incontri. Il team adotta un approccio collaborativo, dove ciascun membro ha un ruolo definito e contribuisce al raggiungimento degli obiettivi. Gli incontri periodici sono fondamentali per monitorare l'avanzamento, risolvere eventuali problemi e pianificare le prossime attività.
\subsubsection{Ruoli}
\subsubsection{Metodo organizzativo}
Il team, con l'obiettivo di ottimizzare i tempi e massimizzare l'efficienza delle proprie operazioni, ha scelto di adottare una metodologia Agile, implementando le pratiche di miglioramento continuo attraverso il framework Scrum\textsubscript{G}. Questo approccio consente di organizzare il lavoro in brevi intervalli temporali, noti come sprint, della durata di circa due settimane. Nello specifico, il processo si articola in diverse fasi:
\begin{itemize}
    \item Pianificazione dello Sprint: Il team si riunisce il primo giorno dello sprint per pianificare le attività da svolgere durante il periodo corrente. Attraverso sessioni di brainstorming, vengono identificate le attività da completare (backlog), stabilendo preventivamente gli impegni e le risorse necessarie.
    \item Revisione dello Sprint: Al termine dello sprint, si tiene una riunione di revisione alla quale partecipano tutti i membri del team. L'obiettivo è definire gli obiettivi raggiunti e produrre almeno un incremento, ovvero un prodotto software utilizzabile. Durante questa fase, vengono analizzate in modo approfondito le risorse impiegate rispetto agli obiettivi, distinguendo tra quelli raggiunti e non raggiunti. Ciò consente di identificare aree di miglioramento e di stabilire nuovi obiettivi per gli sprint successivi.
    \item Retrospective dello Sprint: Concluso lo sprint, il team si riunisce per una valutazione generale del suo andamento. Si analizza ciò che è stato realizzato con successo e ciò che può essere migliorato. Questa fase è cruciale per definire le strategie di ripianificazione delle attività, decidendo come procedere con quelle attuali o future.
\end{itemize}

\subsubsection{Gestione degli incontri}
\subsubsection{Gestione della comunicazione interna ed esterna}
\subsubsection{Gestione delle task}

\section{Processi di supporto}

\subsection{Gestione della qualità}
La gestione della qualità è un approccio sistemico che coinvolge la definizione di standard di qualità, la misurazione delle performance\textsubscript{G} e l'implementazione di azioni correttive. Le metriche di qualità vengono regolarmente monitorate, e i processi vengono migliorati costantemente per garantire l'aderenza agli standard stabiliti e il raggiungimento degli obiettivi di qualità.

\subsubsection{Verifica}
Il processo di verifica è una pratica svolta in maniera continua che coinvolge l'analisi approfondita di ogni componente del progetto. La revisione dei documenti, l'analisi statica e dinamica del codice, insieme a test\textsubscript{G} specifici, contribuiscono a garantire la qualità e l'affidabilità del prodotto. Le correzioni sono apportate tempestivamente in risposta ai risultati della verifica\\
DA CONTINUAREEEEEEEEEEEEE

\subsubsection{Validazione}
Il processo di validazione è il momento culminante in cui il prodotto sviluppato viene testato e confrontato con i requisiti del proponente. Questa fase include una serie di test, sia funzionali che non funzionali, per garantire che il prodotto soddisfi le aspettative. I risultati vengono documentati nel piano di qualifica, evidenziando il grado di conformità e identificando eventuali aree di miglioramento.\\
DA CONTINUAREEEEEEEEEEEE

\subsection{Gestione della configurazione}
La gestione della configurazione è essenziale per tracciare e controllare le modifiche apportate ai componenti del progetto. Utilizzando strumenti di controllo di versione\textsubscript{G}, il team assicura la coerenza e la tracciabilità di tutte le versioni\textsubscript{G} del codice sorgente, della documentazione e di altri artefatti. Questo approccio contribuisce a evitare conflitti e garantisce la riproducibilità del progetto in qualsiasi punto temporale. Il nostro gruppo ha deciso di utilizzare per la gestione della configurazione il servizio GitHub, basato sul sistema di controllo di versione distribuito Git. \\ Il link alla Repository pubblica del progetto è il seguente: \\ \href{https://github.com/CyberSorceres/CyberSorceresRepository}{https://github.com/CyberSorceres/CyberSorceresRepository}
\subsection*{Gestione della repository}
\begin{itemize}
    \item All'interno della Repository\textsubscript{G} è presente una cartella "documenti" contenente i documenti versionati del progetto
    \item All'interno della cartella "documenti" è presente una cartella "verbali" con al suo interno i verbali, interni caricati nel formato tex ed esterni in formato pdf
    \item per il Proof of Concept\textsubscript{G} (PoC), è stato creato una repository separato, contenente tutti i file sorgente relativi al PoC\textsubscript{G} che sarà consegnato in occasione della prima revisione RTB.
\end{itemize}

\subsection{Norme di utilizzo per la repository}
Di seguito descriviamo le norme poste per il mantenimento dell'ordine della Repository\textsubscript{G} per tutta la durata del progetto:
\begin{itemize}
    \item segnalare problemi (issue\textsubscript{G}) e proporre modifiche (pull request\textsubscript{G}). Il team si impegna a risolvere gli issue\textsubscript{G}in modo tempestivo e valutare attentamente le pull request\textsubscript{G} per attuare i miglioramenti necessari
    \item flusso di lavoro con Branches\textsubscript{G}: La repository segue un flusso di lavoro basato su branches\textsubscript{G}, con un branch\textsubscript{G} principale stabile e branch di sviluppo separati. Ciò consente di isolare nuove funzionalità o correzioni di bug prima di integrarle nella versione principale.
    \item nominare i tutti i file seguendo le convenzioni decise dal gruppo e caricarli al suo interno nel formato deciso
\end{itemize}



\subsection{Documentazione}
La documentazione è un pilastro fondamentale del processo. Tutti i documenti sono gestiti in modo rigoroso, versionati e archiviati. La struttura documentale, definita nel piano di progetto, è regolarmente aggiornata per riflettere le modifiche apportate durante lo sviluppo del progetto. La documentazione serve come riferimento chiave per tutti i membri del team e per gli stakeholder\textsubscript{G} interessati. I documenti prodotti dal gruppo possono essere divisi in due categorie: formali e informali.

\subsubsection{Documenti informali}
I documenti informali hanno lo scopo di essere di facile utilizzo e manutenzione ma non sono documenti ufficiali. Si parla quindi di documenti che non ancora stati approvati dal responsabile del progetto, bozze, appunti e documenti che non necessitano di versionamento. Come team abbiamo deciso di usufruire di Google Drive, piattaforma che ci permette di comunicare sincronicamente. Il team lo ha utilizzato ad esempio per la creazione di tabelle dove ogni membro avrebbe potuto aggiornare il monte ore svolto per quel ruolo, o comunicare agli altri membri i propri impegni settimanali e facilitare la struttura organizzativa interna. Gli strumenti utilizzati per questo tipo di documenti sono:
\begin{itemize}
    \item Google Docs
    \item Google Drive
    \item Draw.io
    \item Overleaf
\end{itemize}

\subsubsection{Documenti formali}
Di seguito viene indicata la documentazione formale elaborata.

\subsubsection*{\textbf{Piano di qualifica}}
Lo scopo di un piano di qualifica è definire un quadro dettagliato per assicurare che il software sviluppato soddisfi gli standard di qualità prestabiliti e le specifiche dei requisiti. Questo piano, parte integrante della gestione della qualità del progetto, delineerà le strategie, le procedure e le risorse necessarie per garantire che il prodotto software sia affidabile, efficace e conforme alle aspettative del cliente. Il \textbf{Piano di qualifica} viene redatto dal \textbf{verificatore} e comprende le azioni indispensabili per assicurare l'eccellenza del prodotto e dei processi. Esso è composto dalle seguenti sezioni:\begin{itemize}
    \item Qualità di processo;
    \item Qualità di prodotto;
    \item Specifica dei test;
    \item Resoconto attività di verifica.
\end{itemize}
\subsubsection*{\textbf{Piano di progetto}}
Lo scopo di un piano di progetto è definire in modo chiaro e dettagliato come il processo di sviluppo del software sarà pianificato, gestito e controllato durante tutto il ciclo di vita del progetto. Il piano di progetto fornisce una guida strategica e operativa per il team di sviluppo e gli altri stakeholder coinvolti, stabilendo obiettivi, tempi, risorse e responsabilità. Di seguito vengono indicate le sezioni del documento \textbf{Piano di progetto}:\begin{itemize}
    \item Analisi dei rischi;
    \item Modello di sviluppo;
    \item Pianificazione;
    \item Preventivo;
    \item Consultivo di periodo;
    \item Attualizzazione dei rischi.
\end{itemize}

\subsubsection*{Norme di way of working}
Lo scopo di un documento  è fornire linee guida dettagliate e standardizzate sulle pratiche e i processi che devono essere seguiti durante lo sviluppo del software. Queste norme sono progettate per stabilire un quadro coerente e uniforme per l'intero team di sviluppo, garantendo coerenza, qualità e efficienza nelle attività quotidiane. Di seguito vengono indicate le sezioni del documento \textbf{Way of working}:\begin{itemize}
    \item Processi primari;
    \item Norme;
    \item Processi organizzativi;
    \item Processi di supporto;
\end{itemize}

\subsubsection*{Analisi dei requisiti}
Lo scopo è definire in modo completo e dettagliato le esigenze e le specifiche del sistema software che deve essere sviluppato. Questo documento svolge un ruolo fondamentale nella fase iniziale del ciclo di vita\textsubscript{G} del progetto, servendo come base per la progettazione, lo sviluppo, il test e la valutazione del prodotto software.
Di seguito vengono indicate le sezioni del documento \textbf{Analisi dei requisiti}:\begin{itemize}
    \item User Cases (dall'UC1 all' UC18);
    \item Requisiti
\end{itemize}
\subsubsection*{Glossario}
Lo scopo di questo documento è quello di contenere tutte le definizioni necessarie alla totale comprensione della documentazione redatta. I termini che presentano la lettera G come pedice saranno presenti, in ordine alfabetico, all'interno del glossario. I termini saranno letteralmente riportati o saranno riportate delle declinazioni molto simili del termine. 

\subsubsection{Verbali interni}
Gli scopi di un verbale interno sono principalmente i seguenti:
\begin{itemize}
    \item Registro delle Decisioni: Registra le decisioni prese durante incontri interni del team di sviluppo. Queste decisioni possono riguardare questioni tecniche, piani di progetto, assegnazioni di compiti, ecc.
    \item Comunicazione Interna: Fornisce un mezzo formale per comunicare informazioni rilevanti tra i membri del team. Può includere aggiornamenti sullo stato delle attività, problemi risolti, ostacoli incontrati e soluzioni proposte.à
    \item Tracciamento delle Attività: Documenta le attività in corso, le discussioni e gli accordi raggiunti durante il processo di sviluppo. Aiuta nel monitoraggio del progresso e nell'identificazione di eventuali ritardi o sfide.
    \item Archivio Storico: Costituisce un archivio storico delle decisioni e delle attività, facilitando il riferimento futuro e la comprensione del contesto per i membri del team.
    \item Risposta a Rischi e Problemi: Registra le discussioni relative a rischi o problemi emersi durante lo sviluppo e le strategie di mitigazione o risoluzione proposte.
\end{itemize}

I verbali interni vengono redatti da un membro del gruppo presente alla riunione, e vengono compilati tramite i template \textsubscript{G} creati dal gruppo, e vengono resi disponibili il prima possibile all'interno della Repository\textsubscript{G} in modo tale da aggiornare nella maniera più celere i membri assenti alla riunione. I verbali interni non necessitano di firma, e vengono versionati\textsubscript{G}, rispettando le norme sopra citate.

\subsubsection{Verbali esterni}
I verbali esterni nascono per i seguenti scopi:
\begin{itemize}
    \item Comunicazione con gli Stakeholder: Fornisce una documentazione formale delle discussioni e delle decisioni prese durante incontri con gli stakeholder esterni, come clienti, partner o altri team coinvolti nel progetto.
    \item Conferma di Accordi: Serve come conferma scritta degli accordi raggiunti tra le parti, riducendo il rischio di malintesi o interpretazioni divergenti.
    \item Documentazione di Riunioni con Clienti: Registra le discussioni avute con il cliente, comprese le richieste specifiche, le risposte alle domande e le spiegazioni dettagliate fornite durante le interazioni.
    \item Monitoraggio dello Stato del Progetto: Comunica agli stakeholder esterni lo stato del progetto, i progressi compiuti e le sfide incontrate. Contribuisce a mantenere un elevato livello di trasparenza e fiducia.
    \item Base per l'Approvazione: Può essere utilizzato come documento di riferimento per ottenere l'approvazione formale da parte degli stakeholder su decisioni, modifiche o altri aspetti critici del progetto. I verbali esterni infatti necessitano di firma da parte del proponente.
\end{itemize}

I verbali esterni vengono redatti da un membro del gruppo presente alla riunione, e vengono compilati tramite i template \textsubscript{G} creati dal gruppo, e vengono resi disponibili il prima possibile all'interno della Repository\textsubscript{G} dopo essere stati mandati ai proponenti. Necessario è infatti che, vista la tipologia di informazioni archiviate all'interno di questi documenti, necessitano di firma da parte del proponente e del Responsabile del gruppo. Infine vengono versionati\textsubscript{G}, rispettando le norme sopra citate.

\subsubsection{Struttura dei documenti e vincoli formali}
\subsubsection*{Nome dei documenti}
Ogni documento viene nominato tramite una convezione decisa dal gruppo. I nomi dei documenti è in carattere interamente minuscolo, ogni parola suddivisa da un underscore e successivamente il versionamento, fatta eccezione dei verbali in cui al posto del versionamento è presente la data dell'incontro in forma "giorno"\_"mese"\_"anno".
\subsubsection*{Prima pagina}
\begin{itemize}
    \item Presenterà sempre il nome del documento, il nome del team e il logo scelto
    \item Tabella con i nomi dei membri del gruppo e una tabella con informazioni riguardanti il documento (come i suoi destinatari).
    \item Sarà poi presente un Changelog\textsubscript{G} con il verionamento del file relativo alla modifica messa in atto, oltre al suo autore, verificatore, una breve spiegazione della modifica e la data in cui questa è stata messa in atto.
\end{itemize}

\subsection{Header}
Sarà presente in ogni documento un header contenente il nome del documento, il nome del team e il versionamento.
\subsubsection*{Versionamento}
Ogni documento che non sia un verbale viene versionato secondo la norma sopra citata (vedi sezione 3.3). Vengono versionati tutti i documenti fatta eccezione, per loro natura, dei verbali interni ed esterni.
\subsubsection*{Indici}
Successivamente agli elementi citati nella sezione 5.3.5 sarà presente un indice contenente tutte le sezioni del documento. Al termine del documento sarà presente inoltre un indice contenente il riferimento a tutti gli elementi grafici (tabelle ed immagini) presenti. 

\subsubsection{Elementi grafici}
\subsubsection{Tabelle}
Ogni tabella significativa contenuta nei documenti sarà accompagnata da una legenda, e l'insieme di queste legende sarà un indice presente alla fine di ogni documento,
\subsubsection{Immagini}
Le immagini presenti sono state create usando due diversi strumenti grafici:
\begin{itemize}
    \item Canva: utilizzato per la creazione del logo del team. 
    \item Draw.io: per le immagini rappresentative degli Use Case.
\end{itemize}
Ogni immagine, al di fuori del logo, sarà accompagnata da una descrizione, e l'insieme di queste descrizioni sarà presente in un indice alla fine di ogni documento. 

\subsection*{Glossario}
\subsection*{Scopo del documento}
Il Glossario è un documento dove sono presenti tutte le definizioni delle parole che il gruppo ha ritenuto necessitassero di una definizione per facilitare la comprensione dei documenti. 
\subsection*{Norme di compilazione}
\begin{itemize}
    \item I termini sono ordinati in ordine alfabetico e sono facilmente individuabili all'interno del Glossario tramite una legenda presente all'inizio del documento.
    \item I termini sono ordinati in ordine alfabetico e sono facilmente individuabili all'interno del Glossario tramite una legenda presente all'inizio del documento.
    \item Le parole contenute nel Glossario sono presenti nei documenti e sono, per ogni loro occorrenza, indicate con una "G" al pedice.
    \item Il team ha deciso di escludere dal glossario le parole contenute nei titoli delle sezioni e nelle didascalie delle immagini, poichè già certamente presenti all'interno del testo dei documenti. 
    \item Il Glossario viene aggiornato sulla piattaforma Overleaf da ogni membro ogni qualvolta viene compilata una sezione testuale, in modo tale che l'aggiunta dei termini sia costante e completa. 
    \item Prima di aggiungere una parola nel Glossario ogni membro più velocemente verificare se la parola è già presente al suo interno tramite la legenda.
\end{itemize}  

\subsection{Verbali}
\subsection*{Esterni}
Ogni verbale esterno avrà la seguente struttura:
\begin{itemize}
    \item Titolo del documento
    \item Logo del team
    \item Data della riunione, orario della riunione, metodologia di svolgimento (in presenza o da remoto) e redattore del verbale
    \item Una tabella contenente i presenti e gli assenti alla riunione
    \item La firma del responsabile interno e quella del responsabile esterno del progetto
    \item Un header per ogni pagina con il nome del team, il tipo di documento e la data della riunione
    \item Obiettivo della riunione: sezione nella quale viene identificato il motivo per il quale si è tenuta la riunione e ciò che si vuole discutere. 
    \item Dubbi: sezione contenente i dubbi emersi durante la presentazione del lavoro
    \item Decisioni: lista di decisioni discusse e messe in atto dal team a seguito dell'esito della riunione
    \item Azioni: in cui vengono stilate eventuali azioni svolte durante la riunione
    \item Attività future: vengono elencate le attività che il team ha intenzione di svolgere nel successivo Sprint. 
    \item Eventuali note
\end{itemize}

\subsection*{Interni}
Ogni verbale esterno avrà la seguente struttura:
\begin{itemize}
    \item Titolo del documento
    \item Logo del team
    \item Data della riunione, orario della riunione, metodologia di svolgimento (in presenza o da remoto) e redattore del verbale
    \item Una tabella contenente i presenti e gli assenti alla riunione
    \item Un header per ogni pagina con il nome del team, il tipo di documento e la data della riunione
    \item Obiettivo della riunione: sezione nella quale viene identificato il motivo per il quale si è tenuta la riunione e ciò che si vuole discutere. 
    \item Dubbi: sezione contenente i dubbi emersi durante la presentazione del lavoro
    \item Decisioni: lista di decisioni discusse e messe in atto dal team a seguito dell'esito della riunione
    \item Azioni: in cui vengono stilate eventuali azioni da svolgere nel successivo sprint, associando il compito ad un membro o più del gruppo ed indicando il termine massimo per il suo completamento.
    \item Attività future: vengono elencate le attività che il team ha intenzione di svolgere nel successivo Sprint. 
    \item Eventuali note
\end{itemize}





\end{document}
