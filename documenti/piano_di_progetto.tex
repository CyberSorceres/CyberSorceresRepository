\documentclass{article}
\providecommand{\versionnumber}{1.2.0}
\usepackage[utf8]{inputenc}
\usepackage[includeheadfoot, margin=1em,headheight=2em]{geometry}
\usepackage{titling}
\usepackage{hyperref}
\usepackage[italian]{babel}
\geometry{a4paper, left=2cm, right=2cm, top=2cm, bottom=2cm}
\usepackage{graphicx}
\usepackage{float}
\usepackage{enumitem}
\usepackage{array}
\usepackage{eurosym}
\newcolumntype{P}[1]{>{\centering\arraybackslash}p{#1}}
\renewcommand{\arraystretch}{1.5} % Default value: 1
\setlength{\droptitle}{-6em}
\usepackage{capt-of}

%font
\usepackage[defaultfam,tabular,lining]{montserrat}
\usepackage[T1]{fontenc}
\renewcommand*\oldstylenums[1]{{\fontfamily{Montserrat-TOsF}\selectfont #1}}

%custom bold 
\usepackage[outline]{contour}
\usepackage{xcolor}
\newcommand{\custombold}{\contour{black}}

%table colors
\usepackage{color, colortbl}
\definecolor{Blue}{rgb}{0.51,0.68,0.79}
\definecolor{LightBlue}{rgb}{0.82,0.87,0.90}
\definecolor{LighterBlue}{rgb}{0.93,0.95,0.96}

%Header
\usepackage{fancyhdr, xcolor}
\pagestyle{fancy}
\let\oldheadrule\headrule% Copy \headrule into \oldheadrule
\renewcommand{\headrule}{\color{Blue}\oldheadrule}% Add colour to \headrule
\renewcommand{\headrulewidth}{0.2em}
\fancyhead[L]{Piano di progetto}
\fancyhead[C]{Cybersorceres}
\fancyhead[R]{versione \versionnumber}

\title{\Huge{\textbf{Piano di progetto}}\vspace{-1em}}
\author{CyberSorcerers Team}
\date{}
\begin{document}
\maketitle
\vspace{-3em}
\begin{figure}[h]
  \centering
  \includegraphics[width=6cm, height=6cm]{documenti/logo rotondo.png}
  \label{fig:immagine}
\end{figure}

\vspace{3em}
\large{
\begin{center}
    \begin{tabular}{P{24em}}
        \rowcolor{Blue}
        \textbf{Membri del team:}\\
        \rowcolor{LighterBlue}
        \custombold{Sabrina Caniato}\\
        \rowcolor{LightBlue}
        \custombold{Giulia Dentone}\\
        \rowcolor{LighterBlue}
        \custombold{Nicola Lazzarin}\\
        \rowcolor{LightBlue}
        \custombold{Giovanni Moretti}\\
        \rowcolor{LighterBlue}
        \custombold{Andrea Rezzi}\\
        \rowcolor{LightBlue}
        \custombold{Samuele Vignotto}\\
    \end{tabular}
\end{center}}

\begin{center}
    \begin{tabular}{l c c}
        \rowcolor{Blue} 
        \textbf{Informazioni sul documento} & &\\ [1 ex]
        \rowcolor{LighterBlue}
        Destinatari: & Prof Tullio Vardanega & Prof Riccardo Cardin\\ [1 ex]
        \rowcolor{LightBlue}
        G al pedice: & Consultare il Glossario & \\ [1 ex]
    \end{tabular}
\end{center}
    

\newpage
\textbf{Registro dei Cambiamenti - Changelog}
\begin{center}
\begin{tabular}{P{5em} P{6em} P{8em} P{8em} P{10em}} 
  \rowcolor{Blue}
    \custombold{Versione} & \custombold{Data} & \custombold{Autore} &
    \custombold{ Verificatore} & \custombold{Dettaglio}\\
    \rowcolor{LighterBlue}
    1.2.0 & 06/05/2024 & Samuele Vignotto & Giulia Dentone & Concluso consuntivo periodo Sviluppo Proof of Concept\textsubscript{G} e aggiunte retrospettive\\
    \rowcolor{LightBlue}
    1.1.1 & 06/05/2024 & Giulia Dentone & Samuele Vignotto & Aggiunta di un rischio organizzativo interno.\\
    \rowcolor{LighterBlue}
    1.1.0 & 06/05/2024 & Giulia Dentone & Samuele Vignotto & Correzione della sezione 3.\\
    \rowcolor{LightBlue}
     1.0.0 & 14/04/2024 & Nicola Lazzarin & Samuele Vignotto & Aggiornamento a versione pronta al rilascio.\\
    \rowcolor{LighterBlue}
     0.5.3 & 10/04/2024 & Sabrina Caniato & Samuele Vignotto & Update Analisi dei rischi tecnologici.\\
    \rowcolor{LightBlue}
     0.5.2 & 5/04/2024 & Nicola Lazzarin & Giulia Dentone & Correzione grafici.\\
    \rowcolor{LighterBlue}
     0.5.1 & 1/03/2024 & Nicola Lazzarin & Samuele Vigonotto & Update dei diagrammi di Gantt.\\
    \rowcolor{LightBlue}
     0.5.0 & 20/01/2024 & Giulia Dentone & Nicola Lazzarin & Aggiunta della mitigazione dei rischi.\\
    \rowcolor{LighterBlue}
     0.4.2 & 17/01/2024 & Samuele Vignotto & Andrea Rezzi & Update del consuntivo.\\
    \rowcolor{LightBlue}
     0.4.1 & 13/01/2024 & Sabrina Caniato & Giulia Dentone & Update del consuntivo.\\
    \rowcolor{LighterBlue}
     0.4.0 & 12/01/2024 & Samuele Vignotto & Giovanni Moretti & Compilazione del consuntivo.\\
    \rowcolor{LightBlue}
     0.3.3 & 07/01/2024 & Nicola Lazzarin & Sabrina Caniato & Inserimento delle caption delle immagini.\\
     \end{tabular}
\end{center}

\begin{center}
\begin{tabular}{P{5em} P{6em} P{8em} P{8em} P{10em}} 
    \rowcolor{LighterBlue}
     0.3.2 & 03/01/2024 & Andrea Rezzi & Nicola Lazzarin & Update dei riepiloghi.\\ 
    \rowcolor{LightBlue}
    0.3.1 & 03/01/2024 & Giovanni Moretti & Sabrina Caniato & Update dei riepiloghi.\\
    \rowcolor{LighterBlue}
     0.3.0 & 27/12/2023 & Samuele Vignotto & Giovanni Moretti & Update della sezione preventivo con i riepiloghi.\\ 
    \rowcolor{LighterBlue}
     0.2.1 & 19/12/2023 & Samuele Vignotto & Giulia Dentone & Aggiunta sezioni e contenuto sezione "Periodi" della sezione "Pianificazione".\\
    \rowcolor{LightBlue}
     0.2.0 & 18/12/2023 & Giulia Dentone & Nicola Lazzarin & Compilazione della sezione Pianificazione con descrizione fasi e periodi.\\
    \rowcolor{LighterBlue}
     0.1.1 & 16/12/2023 & Giovanni Moretti & Sabrina Caniato & Update della sezione dei rischi.\\

    \rowcolor{LightBlue}
    0.1.0 & 15/12/2023 & Sabrina Caniato & Giovanni Moretti & Aggiunta della sezione dei rischi.\\
    \rowcolor{LighterBlue}
     0.0.1 & 14/12/2023 & Andrea Rezzi & Nicola Lazzarin & efinizione struttura del documento e scheletro delle sezioni. Scrittura introduzione ed obiettivi delle diverse sezioni.\\ 
\end{tabular}
\end{center}
\newpage
\tableofcontents
\newpage
\section{Introduzione}
\subsection{Scopo del documento}
Lo scopo del documento è quello di normare il processo di sviluppo del progetto, in tempi e modalità. In particolare viene effettuata un'analisi dei rischi, e delle relative azioni e modalità che verranno adottate per mitigarli. il documento viene redatto con un approccio incrementale, al fine di poter implementare modifiche concordate dal gruppo o dal proponente.

\subsection{Obiettivo del prodotto}
Prima di poter procedere all'analisi dei rischi, è necessario identificare chiaramente l'obiettivo del prodotto: la creazione di una web app che, tramite l’uso di IA\textsubscript{G}(ChatGPT4 e Bedrock) crei, a partire dalle richieste del cliente, epic user stories\textsubscript{G} e confrontarle con il codice sviluppato. Il fine è quello di informare il cliente dello stato di avanzamento dello sviluppo del prodotto e rendere possibile al Project Manager\textsubscript{G} e al cliente rilasciare dei
feedback (riguardanti, in base all'utente utilizzatore, l'adeguatezza delle stories\textsubscript{G} o il prodotto finale) al fine di migliorare l’IA\textsubscript{G}. In ultimo è richiesto un confronto tra le IA\textsubscript{G} utilizzate e lo sviluppo di un plugin utile agli sviluppatori e al Project Manager\textsubscript{G}.

\subsection{Glossario}
I termini impiegati in questo testo potrebbero suscitare incertezze circa il loro significato, rendendo quindi necessaria una definizione per evitare ambiguità. Tali termini sono identificati da una lettera "G" maiuscola posta in pedice alla parola, e la loro spiegazione è fornita nel Glossario v1.0.0.

\subsection{Riferimenti}
\subsubsection{Riferimenti normativi}
\begin{itemize}
    \item \textbf{Regolamento del progetto didattico:}
\end{itemize}
\href{https://www.math.unipd.it/~tullio/IS-1/2023/Dispense/PD2.pdf}{PD2.pdf}
\begin{itemize}
    \item \textbf{Capitolato d'appalto C7 - ChatGPT\textsubscript{G} vs BedRock\textsubscript{G} developer Analysis}
\end{itemize}
\href{https://www.math.unipd.it/~tullio/IS-1/2023/Progetto/C7.pdf}{C7.pdf}

\subsubsection{Riferimenti informativi}
\begin{itemize}
\item Lezioni del corso di Ingegneria del Software "I processi di ciclo di vita del SW": \\ \\
\url{https://www.math.unipd.it/~tullio/IS-1/2023/Dispense/T2.pdf} 
\item Lezioni del corso di Ingegneria del Software "Amministrazione di progetto": \\ \\
\url{https://www.math.unipd.it/~tullio/IS-1/2023/Dispense/PD5.pdf}
\end{itemize}

\section{Analisi dei rischi}
Questa sezione si occupa di analizzare le difficoltà riscontrabili dal proponente ed evitare problemi che possono intercorrere tra lo stato di avanzamento e il completamento del progetto. Analizzeremo dunque ciascun rischio, descrivendolo e giudicando il suo grado di rischio, pericolosità, precauzione e le misure di mitigazione adottate. Il fine è quello di permettere una loro facile identificazione e un continuo monitoraggio. Abbiamo deciso di suddividerli secondo tre differenti categorie:le difficoltà personali, le difficoltà organizzative interne ed esterne e le difficoltà tecnologiche/software.

\subsection{Rischi organizzativi}
\subsubsection{Comunicazione interna}
\begin{center}
\begin{tabular}{P{10em} P{20em}} 
    \rowcolor{LighterBlue}
     Descrizione & Mancata reperibilità sincrona dei membri del gruppo, data da eventuali impegni personali\\ 
    \rowcolor{LightBlue}
    Occorrenza & Media\\
    \rowcolor{LighterBlue}
    Pericolosità & Alta \\
    \rowcolor{LightBlue}
    Precauzioni & Ogni membro del gruppo comunica vocalmente i propri impegni straordinari della settimana ad ogni riunione interna \\
    \rowcolor{LighterBlue}
    Misure di contenimento & Ogni membro del gruppo compilerà un documento Drive interno  con i suoi impegni fissi o inderogabili \\
\end{tabular}
\captionof{table}{Rischi della comunicazione interna}
\label{tab:cominterna}
\end{center}

\subsubsection{Comunicazione esterna}
\begin{center}
\begin{tabular}{P{10em} P{20em}} 
    \rowcolor{LighterBlue}
     Descrizione & Difficoltà nella comunicazione repentina con l'azienda\\ 
    \rowcolor{LightBlue}
    Occorrenza & Bassa\\
    \rowcolor{LighterBlue}
    Pericolosità & Alta \\
    \rowcolor{LightBlue}
    Precauzioni & Richiedere al cliente il modo più semplice e celere di ottenere una sua risposta \\
    \rowcolor{LighterBlue}
    Misure di contenimento & Creazione di un canale Slack attivo\\
\end{tabular}
\captionof{table}{Rischi della comunicazione esterna}
\label{tab:comesterna}
\end{center}

\subsubsection{Mancata esperienza professionale}
\begin{center}
\begin{tabular}{P{10em} P{20em}} 
    \rowcolor{LighterBlue}
     Descrizione & Gran parte dei i membri del gruppo non ha esperienze significative in ambito di sviluppo o professionali\\ 
    \rowcolor{LightBlue}
    Occorrenza & Alta\\
    \rowcolor{LighterBlue}
    Pericolosità & Media \\
    \rowcolor{LightBlue}
    Precauzioni & Ogni membro deve essere trasparente nel comunicare le sue competenze \\
    \rowcolor{LighterBlue}
    Misure di contenimento & Domandare e chiarire in corso d'opera al docente eventuali perplessità e dubbi ed effettuare sedute di formazione con il cliente (facente parte del settore)Il cliente potrebbe richiedere delle modifiche in corso d'opera dei requisiti 
 \\
\end{tabular}
\captionof{table}{Mancata esperienza professionale}
\label{tab:espprof}
\end{center}

\subsubsection{Modifiche in corso d'opera}
\begin{center}
\begin{tabular}{P{10em} P{20em}} 
    \rowcolor{LighterBlue}
     Descrizione & Il cliente potrebbe richiedere delle modifiche in corso d'opera dei requisiti\\ 
    \rowcolor{LightBlue}
    Occorrenza & Bassa\\
    \rowcolor{LighterBlue}
    Pericolosità & Alta \\
    \rowcolor{LightBlue}
    Precauzioni & Il gruppo di impegna ad essere trasparente e a comunicare molto con il cliente \\
    \rowcolor{LighterBlue}
    Misure di contenimento & Tramite il canale Slack prestabilito il gruppo comunicherà al cliente in tempo reale ad ogni conclusione di un obiettivo prestabilito \\
\end{tabular}
\captionof{table}{Rischi di modifiche in corso d'opera}
\label{tab:modifiche}
\end{center}

\subsection{Rischi tecnologici}
\subsubsection{Strumenti software}
\begin{center}
\begin{tabular}{P{10em} P{20em}} 
    \rowcolor{LighterBlue}
     Descrizione & Il gruppo non ha esperienza con strumenti software di tracciamento e gestione di un progetto\\ 
    \rowcolor{LightBlue}
    Occorrenza & Bassa\\
    \rowcolor{LighterBlue}
    Pericolosità & Media \\
    \rowcolor{LightBlue}
    Precauzioni & Ogni membro comunica eventuali difficoltà e riceve aiuto da parte di membri più esperti \\
    \rowcolor{LighterBlue}
    Misure di contenimento & Scegliere i software più conosciuti, affidabili, intuitivi e meglio documentati \\
\end{tabular}
\captionof{table}{Rischi strumenti sofrware}
\label{tab:sofrware}
\end{center}

\subsubsection{Esperienza tecnologica dei membri}
\begin{center}
\begin{tabular}{P{10em} P{20em}} 
    \rowcolor{LighterBlue}
     Descrizione & La maggior parte dei i membri del gruppo partecipano per la prima volta allo svolgimento di un progetto complesso \\ 
    \rowcolor{LightBlue}
    Occorrenza & Media\\
    \rowcolor{LighterBlue}
    Pericolosità & Media \\
    \rowcolor{LightBlue}
    Precauzioni & I membri comunicheranno vicendevolmente le proprie lacune\\
    \rowcolor{LighterBlue}
    Misure di contenimento & I membri si impegnano a colmare le proprie lacune attraverso lo studio e la pratica \\
\end{tabular}
\captionof{table}{Rischi per l'esperienza tecnologica}
\label{tab:esptec}
\end{center}

\subsubsection{Assenza di pattern strutturali nelle tecnologie scelte}
\begin{center}
\begin{tabular}{P{10em} P{20em}} 
    \rowcolor{LighterBlue}
     Descrizione & Nelle tecnologie scelte non sono presenti pattern strutturali obligatori. \\ 
    \rowcolor{LightBlue}
    Occorrenza & Media\\
    \rowcolor{LighterBlue}
    Pericolosità & Alta \\
    \rowcolor{LightBlue}
    Precauzioni & Già dal POC abbiamo iniziato a provare alcuni tipi di pattern architetturali.\\
    \rowcolor{LighterBlue}
    Misure di contenimento & Stare più attenti quando faremo la parte di Designe e chiedere più incontri al professor Cardin se abbiamo dubbi. \\
\end{tabular}
\captionof{table}{Rischi per l'assenza dei pattern nelle tecnologie}
\label{tab:esptec}
\end{center}

\subsubsection{Database NoSQL}
\begin{center}
\begin{tabular}{P{10em} P{20em}} 
    \rowcolor{LighterBlue}
     Descrizione & Il database scelto è completamente estraneo ai membri del gruppo e ci limita a fare le join. \\ 
    \rowcolor{LightBlue}
    Occorrenza & Media\\
    \rowcolor{LighterBlue}
    Pericolosità & Media \\
    \rowcolor{LightBlue}
    Precauzioni & Tabelle ben definite ed utilizzo di librerie che ne aiutano la gestione.\\
    \rowcolor{LighterBlue}
    Misure di contenimento & I membri del gruppo si sono impegnati a studiare bene il database prima di usarlo. \\
\end{tabular}
\captionof{table}{Rischi per l'utilizzo di database NoSQL}
\label{tab:esptec}
\end{center}

\section{Pianificazione}
Il team ha scelto di adottare il metodo Agile, che prevede l'implementazione di un modello di rilascio continuo e un costante miglioramento delle funzionalità del prodotto. Questo approccio consente di individuare facilmente i requisiti e di assegnare loro una priorità, garantendo così lo sviluppo graduale e la stabilità del prodotto lungo tutto il processo di realizzazione. Sotto questo modello, ogni fase del lavoro è suddivisa in iterazioni gestibili, permettendo una dettagliata analisi delle attività svolte in ciascuna di esse. Gli obiettivi di ogni iterazione vengono valutati periodicamente e adattati, se necessario, in risposta a eventuali problemi temporali o organizzativi che possano emergere (seguendo il ciclo PDCA\textsubscript{G}). Ci sono diversi vantaggi nell'adozione di questo approccio:
\begin{itemize}
    \item Tracciabilità delle difficoltà: Le sfide incontrate in ogni fase del processo sono tracciate e affrontate tempestivamente, con la possibilità di rivedere gli obiettivi qualora si riscontrino ostacoli;
    \item Revisioni strutturate: Ogni iterazione è suddivisa in periodi specifici, ognuno dei quali include una dettagliata analisi delle attività svolte. Queste revisioni permettono di valutare il lavoro svolto e di apportare modifiche o migliorie necessarie per il proseguimento del progetto;
    \item Individuazione degli errori: Grazie agli incrementi progressivi nel lavoro, gli errori possono essere facilmente identificati e corretti durante il processo di sviluppo, riducendo così il rischio di ritardi o problemi più gravi in fasi successive;
    \item Semplificazione dei test: La pianificazione temporale delle attività facilita l'implementazione dei test, consentendo una fase di verifica più efficiente e precisa. Ciò favorisce anche la rapidità nell'apportare eventuali correzioni o miglioramenti al prodotto.
\end{itemize}

Inoltre, l'approccio Agile consente di rispondere in modo efficace ai cambiamenti, permettendo una continua evoluzione della documentazione in linea con l'avanzamento del progetto. Le informazioni vengono aggiornate e integrate man mano che il lavoro procede, garantendo così una documentazione sempre aggiornata e pertinente.

La pianificazione è conseguente a tre principali fasi, ciascuna terminante con una revisione. Le fasi sono le seguenti:
\begin{itemize}
    \item RTB (Requirements and Technology Baseline\textsubscript{G})
    \item PB (Product Baseline\textsubscript{G})
    \item CA (Costumer Acceptance\textsubscript{G})
\end{itemize}
Per rispettare al meglio la pianificazione concordata dunque, tutte le  attività rimangono verificabili tramite l’utilizzo di baseline\textsubscript{G} e comprovato dall’utilizzo di milestone all'interno della Repository\textsubscript{G}.

\subsection{Periodi}
Per rispettare le scadenze si è deciso di organizzare lo sviluppo del progetto nei seguenti periodi:
\begin{itemize}
    \item \custombold{Analisi};
    \begin{itemize}
        \item Durante il periodo di Analisi, il team dedica il proprio impegno all'analisi approfondita del capitolato, con particolare attenzione alla redazione della prima versione dei documenti fondamentali, quali l'\textit{Analisi dei Requisiti}, il \textit{Piano di Qualifica} e il \textit{Piano di Progetto}. Inoltre viene migliorato anche il documento \textit{Norme Way of Working}.
    \end{itemize}
    \item \custombold{Progettazione Technology Baseline\textsubscript{G}};
    \begin{itemize}
        \item Nel corso del periodo Progettazione Technology Baseline\textsubscript{G}, il team si dedica alla pianificazione delle milestone\textsubscript{G} e all'identificazione dei requisiti da implementare nel Proof of Concept\textsubscript{G}. Inoltre, si procede al perfezionamento della documentazione precedentemente elaborata.
    \end{itemize}
    \item \custombold{Sviluppo Proof of Concept\textsubscript{G}};
    \begin{itemize}
        \item Nell'ambito del periodo Sviluppo Proof of Concept\textsubscript{G}, il team si dedica all'implementazione dei requisiti definiti nella precedente fase. Inoltre, si procede al perfezionamento della documentazione precedentemente elaborata.
    \end{itemize}
    \item \custombold{Sviluppo Requisiti Obbligatori};
    \begin{itemize}
        \item Nel corso del periodo di Sviluppo dei Requisiti Obbligatori, il team si impegna a realizzare l'implementazione di tutti i requisiti obbligatori concordati. Inoltre, si procede al perfezionamento della documentazione precedentemente elaborata.
    \end{itemize}
    \item \custombold{Sviluppo Requisiti Opzionali};
    Nel corso del periodo di Sviluppo dei Requisiti Opzionali, il team si impegna a realizzare l'implementazione di tutti i requisiti opzionali concordati. Inoltre, si procede al perfezionamento della documentazione precedentemente elaborata.
    \item \custombold{Validazione e Collaudo};
    Nel periodo di Validazione e Collaudo, il team assume l'impegno di completare il progetto mediante l'esecuzione delle attività di validazione e collaudo del prodotto.
\end{itemize}
La suddivisione dei periodi lungo la linea temporale viene riassunta graficamente tramite la seguente sequenza temporale per favorire una migliore comprensione:\\
\begin{figure}[h]
\includegraphics[width=18cm, height=5cm]{documenti/grafici/Sequenza temporale progetto.png}
\caption{Sequenza temporale del progetto}
\label{fig:STP}
\end{figure}

\subsubsection{Analisi}
\custombold{Periodo:} dal 11/11/2023 al 23/12/2023\\
Questo periodo inizia con l'assegnazione del capitolato d'appalto e termina all'inizio del periodo di Progettazione Technology Baseline\textsubscript{G}. Inizialmente vengono identificati gli strumenti per il lavoro collaborativo e quelli più adatti per la redazione della documentazione. Successivamente, si procede con un'analisi preliminare per individuare i requisiti necessari allo sviluppo del prodotto.\\
Durante questo periodo, data l'inesperienza del gruppo nelle tematiche trattate, si decide di dedicare tempo alla formazione. Inoltre, vengono redatti ulteriori documenti relativi alle strategie e alla qualità che il gruppo CyberSorceres si propone di rispettare.
\begin{itemize}
    \item \custombold{Formazione:} Tramite ore concordate con il proponente e tramite autoformazione tramite corsi online.
    \item \custombold{Norme del Way of Working:} Si procede con l'individuazione degli strumenti che saranno utilizzati per la stesura della documentazione e per la collaborazione. Le norme sono emanate dall'Amministratore e il rispetto di queste norme dovrà essere certificato dai verificatori. Viene quindi redatto il documento \textit{Norme Way of Working}.
    \item \custombold{Piano di progetto:} Il Responsabile, basandosi sulle date concordate per le revisioni di avanzamento e sulle scadenze stabilite dal gruppo, redige il \textit{Piano di Progetto}.
    \item \custombold{Analisi dei requisiti:} Utilizzando il capitolato d'appalto e attraverso incontri con il proponente, gli Analisti identificano i requisiti del sistema e redigono una prima versione dell'\textit{Analisi dei requisiti}.
    \item \custombold{Piano di qualifica:} L'Amministratore redige i piani e le procedure di gestione per la qualità, mentre i verificatori illustrano l'esito e la completezza delle verifiche effettuate.
    \item \custombold{Glossario:} Viene redatto il \textit{Glossario}. Questo documento viene aggiornato in maniera continuativa.
\end{itemize}
\begin{figure}[H]
    \centering
    \includegraphics[width=\textwidth,height=\textheight,keepaspectratio]{documenti/grafici/diagramma-gantt-periodo-analisi.png}
    \caption{Diagramma di Gantt relativo al periodo di Analisi}
    \label{fig:GA}
\end{figure}

\subsubsection{Progettazione Technology Baseline}
\custombold{Periodo:} dal 23/12/2023 al 12/01/2024\\
Questo periodo inizia dopo l'Analisi e termina all'inizio della Codifica del Proof of Concept\textsubscript{G}. Al termine di questo periodo, è previsto un incontro con il proponente durante il quale verrà presentata la soluzione generale individuata e verranno annotati eventuali correzioni. Verranno inoltre apportati incrementi ai documenti prodotti nei periodi precedenti.\\
L'analisi del sistema effettuata in questo periodo serve come base tecnologica e progettuale per la codifica finale del Proof of Concept, che sarà realizzata nel periodo successivo.
\begin{itemize}
    \item \custombold{Technology Baseline:} In questa attività vengono studiate, analizzate e selezionate le tecnologie, i framework e le librerie da utilizzare nello sviluppo del prodotto.
    \item \custombold{Aggiornamento documentazione:} In questo periodo vengono apportate delle modifiche incrementali ai documenti precedentemente redatti.
\end{itemize}
\begin{figure}[H]
    \centering
    \includegraphics[width=\textwidth,height=\textheight,keepaspectratio]{documenti/grafici/diagramma-gantt-periodo-progettazione-technology-baseline.png}
    \caption{Diagramma di Gantt relativo al periodo di Progettazione Technology Baseline}
    \label{fig:GPTB}
\end{figure}

\subsubsection{Sviluppo Proof of Concept}
\custombold{Periodo:} dal 12/01/2024 al 15/03/2024\\
Questo periodo inizia dopo il periodo di Progettazione per la Technology Baseline e termina con la scadenza di consegna dei documenti per la revisione di Requirements and Technology Baseline (RTB\textsubscript{G}). L'attività principale di questo periodo è la realizzazione di un Proof of Concept\textsubscript{G}.
\begin{itemize}
    \item \custombold{Proof of Concept:} Il Proof of Concept è una dimostrazione eseguibile che servirà come base di partenza su cui effettuare incrementi nei periodi successivi.
    \item \custombold{Aggiornamento documentazione:} In questo periodo vengono apportate delle modifiche incrementali ai documenti precedentemente redatti.
\end{itemize}
\begin{figure}[H]
    \centering
    \includegraphics[width=\textwidth,height=\textheight,keepaspectratio]{documenti/grafici/diagramma-gantt-periodo-sviluppo-proof-of-concept.png}
    \caption{Diagramma di Gantt relativo al periodo di Sviluppo Proof of Concept}
    \label{fig:GSPOC}
\end{figure}

\subsubsection{Sviluppo requisiti obbligatori}
\custombold{Periodo:} dal 15/03/2024 al 8/05/2024\\
Questo periodo inizia dopo la revisione di Requirements and Technology Baseline (RTB\textsubscript{G}) e termina con l'inizio dello Sviluppo requisiti opzionali. Quindi, il Proof of Concept\textsubscript{G} precedentemente sviluppato sarà utilizzato come base per incrementare il prodotto. Una delle attività di questo periodo è la definizione della Product Baseline\textsubscript{G}.
\begin{itemize}
    \item \custombold{Product Baseline:} Questa attività illustra la baseline architetturale del prodotto attraverso i diagrammi delle classi, dimostrando la coerenza con quanto mostrato durante l'attività di Technology Baseline.
    \item \custombold{Test:} I Programmatori scrivono i test di unità e integrazione relativi ai componenti sviluppati in questo periodo.
    \item \custombold{Codifica:} Viene eseguito lo sviluppo del codice del prodotto da parte dei Programmatori relativamente ai componenti descritti dai requisiti obbligatori.
    \item \custombold{Manuale:} Comincia la redazione del documento \textit{Manuale Utente}. Questo documento fornisce indicazioni sull'utilizzo del sistema da parte degli utenti.
    \item \custombold{Aggiornamento documentazione:} In questo periodo vengono apportate delle modifiche incrementali ai documenti precedentemente redatti.
\end{itemize}
\begin{figure}[H]
    \centering
    \includegraphics[width=\textwidth,height=\textheight,keepaspectratio]{documenti/grafici/diagramma-gantt-periodo-sviluppo-requisiti-obbligatori.png}
    \caption{Diagramma di Gantt relativo al periodo di Sviluppo Requisiti Obbligatori}
    \label{fig:GSRO}
\end{figure}

\subsubsection{Sviluppo requisiti opzionali}
\custombold{Periodo:} dal 8/05/2024 al 15/05/2024\\
Questo periodo inizia dopo lo Sviluppo requisiti obbligatori e termina con l'inizio dello periodo di Verifica e validazione. In preparazione alla revisione di Product Baseline (PB\textsubscript{G}), si procede con lo sviluppo dei requisiti opzionali.
\begin{itemize}
    \item \custombold{Test:} I Programmatori scrivono i test di unità e integrazione relativi ai componenti sviluppati in questo periodo.
    \item \custombold{Codifica:} Viene eseguito lo sviluppo del codice del prodotto da parte dei Programmatori relativamente ai componenti descritti dai requisiti opzionali.
    \item \custombold{Aggiornamento documentazione:} In questo periodo vengono apportate delle modifiche incrementali ai documenti precedentemente redatti.
\end{itemize}
\begin{figure}[H]
    \centering
    \includegraphics[width=\textwidth,height=\textheight,keepaspectratio]{documenti/grafici/diagramma-gantt-periodo-sviluppo-requisiti-opzionali.png}
    \caption{Diagramma di Gantt relativo al periodo di Sviluppo Requisiti Opzionali}
    \label{fig:GSROp}
\end{figure}

\subsubsection{Validazione e collaudo}
\custombold{Periodo:} dal 15/05/2024 al 28/05/2024\\
Questo periodo inizia dopo lo Sviluppo requisiti opzionali e termina con la scadenza di consegna dei documenti per la revisione di Customer Acceptance (CA\textsubscript{G}). Il sistema verrà collaudato e ci si assicurerà che il prodotto realizzato sia pienamente conforme alle aspettative.
\begin{itemize}
    \item \custombold{Test:} Oltre ai test di unità e integrazione, vengono eseguiti test di sistema.
    \item \custombold{Collaudo:} Il prodotto viene eseguito e testato in tutte le sue funzionalità, verificando che siano stati soddisfatti tutti i requisiti.
    \item \custombold{Validazione:} Viene verificato che il prodotto sia conforme alle specifiche e soddisfi le richieste del cliente.
    \item \custombold{Aggiornamento documentazione:} In questo periodo vengono apportate delle modifiche incrementali ai documenti precedentemente redatti.
\end{itemize}
\begin{figure}[H]
    \centering
    \includegraphics[width=\textwidth,height=\textheight,keepaspectratio]{documenti/grafici/diagramma-gantt-periodo-valida-e-collaudo.png}
    \caption{Diagramma di Gantt relativo al periodo di Validazione e Collaudo}
    \label{fig:GVC}
\end{figure}
\newpage
\section{Preventivo}
Ogni membro del gruppo può assumere più di un ruolo, sia contemporaneamente che in fasi diverse del progetto, a condizione che non vi sia un conflitto di interessi tra i ruoli assunti. La divisione del lavoro che sarà illustrata di seguito assicurerà una distribuzione equa del carico di lavoro individuale e dei vari ruoli.\\
\\
Nelle tabelle saranno impiegate abbreviazioni per indicare i nomi dei ruoli, secondo quanto specificato nella seguente tabella:
\begin{center}
    \begin{tabular}{c|c}
    \rowcolor{Blue}
    \custombold{Ruolo} & \custombold{Abbreviazione}\\
    \rowcolor{LighterBlue}
    \custombold{Responsabile} & Re\\
    \rowcolor{LightBlue}
    \custombold{Amministratore} & Am\\
    \rowcolor{LighterBlue}
    \custombold{Analista} & An\\
    \rowcolor{LightBlue}
    \custombold{Progettista} & Pt\\
    \rowcolor{LighterBlue}
    \custombold{Programmatore} & Pr\\
    \rowcolor{LightBlue}
    \custombold{Verificatore} & Ve\\
    \end{tabular}
    \captionof{table}{Abbreviazioni dei ruoli}
\label{tab:ruoli}
\end{center}
\subsection{Riepilogo economico e delle ore - Periodo Analisi}
Durante il periodo di Analisi, ciascun membro assumerà i ruoli secondo la seguente distribuzione:\\
\\
\begin{center}
\begin{tabular}{c|c|c|c|c|c|c|c}
\rowcolor{Blue}
\custombold{Nominativo} & \custombold{Re} & \custombold{Am} & \custombold{An} & \custombold{Pt} & \custombold{Pr} & \custombold{Ve} & \custombold{Ore Totali}\\
\hline
\rowcolor{LighterBlue}
Sabrina Caniato & 9 & 0 & 14 & 0 & 0 & 5 & 28\\
\rowcolor{LightBlue}
Giulia Dentone & 0 & 12 & 13 & 0 & 0 & 5 & 30\\
\rowcolor{LighterBlue}
Nicola Lazzarin & 0 & 12 & 14 & 0 & 0 & 5 & 31\\
\rowcolor{LightBlue}
Giovanni Moretti & 12 & 0 & 15 & 0 & 0 & 5 & 32\\
\rowcolor{LighterBlue}
Andrea Rezzi & 8 & 0 & 14 & 0 & 0 & 5 & 27\\
\rowcolor{LightBlue}
Samuele Vignotto & 0 & 12 & 11 & 0 & 0 & 5 & 28\\
\rowcolor{LighterBlue}
\custombold{Ore totali} & 29 & 36 & 81 & 0 & 0 & 30 & 176\\
\end{tabular}
\captionof{table}{Preventivo ore per l'analisi}
\label{tab:preventivoAnalisi}
\end{center}


\begin{figure}[h]
    \centering
\includegraphics[width=17cm, height=10cm]{documenti/grafici/Divisione_ore_lavorative_Analisi.png}    
    \caption{Grafico divisione ore lavorative periodo di Analisi}
    \label{fig:preventivoAnalisi}
\end{figure}

\newpage

In questo periodo i costi da affrontare sono:
\begin{center}
    \begin{tabular}{c|c|c}
    \rowcolor{Blue}
    \custombold{Ruolo} & \custombold{Ore} & \custombold{Costo \euro}\\
    \rowcolor{LighterBlue}
    Responsabile & 29 & 870\\
    \rowcolor{LightBlue}
    Amministratore & 36 & 720\\
    \rowcolor{LighterBlue}
    Analista & 81 & 2025\\
    \rowcolor{LightBlue}
    Progettista & 0 & 0\\
    \rowcolor{LighterBlue}
    Programmatore & 0 & 0\\
    \rowcolor{LightBlue}
    Verificatore & 30 & 450\\
    \rowcolor{LighterBlue}
    \custombold{Totale} & \custombold{176} & \custombold{4065}\\
    \end{tabular}
    \captionof{table}{Costi nel periodo di Analisi}
    \label{tab:costiAnalisi}
\end{center}

\begin{figure}[h]
    \centering
    \includegraphics[width=17cm, height=10cm]{documenti/grafici/Torta_percentuale_costi_Analisi.jpg}
    \caption{grafico della divisione percentuale dei costi sostenuti nel periodo di Analisi}
    \label{fig:enter-label}
\end{figure}


\newpage

\subsection{Riepilogo economico e delle ore parziale - Periodo Progettazione Technology Baseline\textsubscript{G}}
Durante il periodo di Progettazione Technology Baseline\textsubscript{G}, ciascun membro assumerà i ruoli secondo la seguente distribuzione:\\
\\
\begin{center}
\begin{tabular}{c|c|c|c|c|c|c|c}
\rowcolor{Blue}
\custombold{Nominativo} & \custombold{Re} & \custombold{Am} & \custombold{An} & \custombold{Pt} & \custombold{Pr} & \custombold{Ve} & \custombold{Ore Totali}\\
\hline
\rowcolor{LighterBlue}
Sabrina Caniato & 0 & 0 & 10 & 5 & 0 & 2 & 17\\
\rowcolor{LightBlue}
Giulia Dentone & 0 & 0 & 6 & 2 & 0 & 2 & 10\\
\rowcolor{LighterBlue}
Nicola Lazzarin & 0 & 0 & 7 & 3 & 0 & 2 & 12\\
\rowcolor{LightBlue}
Giovanni Moretti & 0 & 3 & 4 & 1 & 0 & 2 & 10\\
\rowcolor{LighterBlue}
Andrea Rezzi & 0 & 8 & 4 & 3 & 0 & 2 & 17\\
\rowcolor{LightBlue}
Samuele Vignotto & 12 & 0 & 1 & 1 & 0 & 2 & 16\\
\rowcolor{LighterBlue}
\custombold{Ore totali} & 12 & 11 & 32 & 15 & 0 & 12 & 82\\
\end{tabular}
\captionof{table}{Preventivo ore Progettazione Technology Baseline\textsubscript{G}}
\label{tab:PTB}
\end{center}

\begin{figure}[h]
    \centering
    \includegraphics[width=17cm, height=10cm]{documenti/grafici/Divisione_ore_lavorative_Progettazione_Technology_Baseline.png}
    \caption{Grafico divisione ore lavorative periodo di Progettazione Techonology Baseline\textsubscript{G}}
    \label{fig:PTB}
\end{figure}

\newpage
In questo periodo i costi da affrontare sono:
\begin{center}
    \begin{tabular}{c|c|c}
    \rowcolor{Blue}
    \custombold{Ruolo} & \custombold{Ore} & \custombold{Costo \euro}\\
    \rowcolor{LighterBlue}
    Responsabile & 12 & 360\\
    \rowcolor{LightBlue}
    Amministratore & 11 & 220\\
    \rowcolor{LighterBlue}
    Analista & 32 & 800\\
    \rowcolor{LightBlue}
    Progettista & 15 & 375\\
    \rowcolor{LighterBlue}
    Programmatore & 0 & 0\\
    \rowcolor{LightBlue}
    Verificatore & 12 & 180\\
    \rowcolor{LighterBlue}
    \custombold{Totale} & \custombold{82} & \custombold{1935}\\
    \end{tabular}
    \captionof{table}{Preventivo costi Progettazione Technology Baseline\textsubscript{G}}
\label{tab:costiPTB}
\end{center}

\begin{figure}[h]
    \centering
    \includegraphics[width=17cm, height=10cm]{documenti/grafici/Torta_percentuale_costi_Progettazione_Technology_Baseline.png}
 \caption{grafico della divisione percentuale dei costi sostenuti nel periodo di Progettazione Technology Baseline\textsubscript{G}}
    \label{fig:costiPTB}
\end{figure}
   
\newpage

\subsection{Riepilogo economico e delle ore parziale - Periodo Sviluppo Proof of Concept\textsubscript{G}}
Durante il periodo di Sviluppo Proof of Concept, ciascun membro assumerà i ruoli secondo la seguente distribuzione:\\
\\
\begin{center}
\begin{tabular}{c|c|c|c|c|c|c|c}
\rowcolor{Blue}
\custombold{Nominativo} & \custombold{Re} & \custombold{Am} & \custombold{An} & \custombold{Pt} & \custombold{Pr} & \custombold{Ve} & \custombold{Ore Totali}\\
\hline
\rowcolor{LighterBlue}
Sabrina Caniato & 0 & 0 & 1 & 3 & 6 & 2 & 12\\
\rowcolor{LightBlue}
Giulia Dentone & 12 & 0 & 1 & 1 & 5 & 2 & 21\\
\rowcolor{LighterBlue}
Nicola Lazzarin & 0 & 0 & 1 & 3 & 9 & 2 & 15\\
\rowcolor{LightBlue}
Giovanni Moretti & 0 & 5 & 1 & 2 & 7 & 2 & 17\\
\rowcolor{LighterBlue}
Andrea Rezzi & 0 & 0 & 3 & 3 & 8 & 2 & 16\\
\rowcolor{LightBlue}
Samuele Vignotto & 0 & 0 & 8 & 3 & 5 & 2 & 18\\
\rowcolor{LighterBlue}
\custombold{Ore totali} & 12 & 5 & 15 & 15 & 40 & 12 & 99\\
\end{tabular}
    \captionof{table}{Preventivo ore Proof of Concept\textsubscript{G}}
\label{tab:POC}
\end{center}

\begin{figure}[h]
    \centering
    \includegraphics[width=17cm, height=10cm]{documenti/grafici/Divisione_ore_lavorative_Sviluppo_Proof_of_Concept.png}    \caption{Grafico divisione ore lavorative periodo di Sviluppo Proof of Concept\textsubscript{G}}
    \label{fig:POC}
\end{figure}


\newpage
In questo periodo i costi da affrontare sono:
\begin{center}
    \begin{tabular}{c|c|c}
    \rowcolor{Blue}
    \custombold{Ruolo} & \custombold{Ore} & \custombold{Costo \euro}\\
    \rowcolor{LighterBlue}
    Responsabile & 12 & 360\\
    \rowcolor{LightBlue}
    Amministratore & 5 & 100\\
    \rowcolor{LighterBlue}
    Analista & 15 & 375\\
    \rowcolor{LightBlue}
    Progettista & 15 & 375\\
    \rowcolor{LighterBlue}
    Programmatore & 40 & 600\\
    \rowcolor{LightBlue}
    Verificatore & 12 & 180\\
    \rowcolor{LighterBlue}
    \custombold{Totale} & \custombold{99} & \custombold{1990}\\
    \end{tabular}
        \captionof{table}{Preventivo costi Proof of Concept\textsubscript{G}}
\label{tab:costiPOC}
\end{center}
\begin{figure}[h]
    \centering
\includegraphics[width=17cm, height=10cm]{documenti/grafici/Torta_percentuale_costi_Sviluppo_Proof_of_Concept.png}    \caption{Grafico della divisione percentuale dei costi sostenuti nel periodo di Sviluppo Proof of Concept\textsubscript{G}}
    \label{fig:enter-label}
\end{figure}

\newpage

\subsection{Riepilogo economico e delle ore parziale - Periodo Sviluppo Requisiti Obbligatori}
Durante il periodo di Sviluppo Requisiti Obbligatori, ciascun membro assumerà i ruoli secondo la seguente distribuzione:\\
\\
\begin{center}
\begin{tabular}{c|c|c|c|c|c|c|c}
\rowcolor{Blue}
\custombold{Nominativo} & \custombold{Re} & \custombold{Am} & \custombold{An} & \custombold{Pt} & \custombold{Pr} & \custombold{Ve} & \custombold{Ore Totali}\\
\hline
\rowcolor{LighterBlue}
Sabrina Caniato & 0 & 7 & 0 & 2 & 6 & 2 & 17\\
\rowcolor{LightBlue}
Giulia Dentone & 0 & 0 & 0 & 4 & 6 & 2 & 12\\
\rowcolor{LighterBlue}
Nicola Lazzarin & 7 & 0 & 0 & 4 & 0 & 2 & 13\\
\rowcolor{LightBlue}
Giovanni Moretti & 0 & 0 & 2 & 3 & 6 & 2 & 13\\
\rowcolor{LighterBlue}
Andrea Rezzi & 0 & 0 & 0 & 3 & 7 & 2 & 12\\
\rowcolor{LightBlue}
Samuele Vignotto & 0 & 0 & 0 & 4 & 6 & 2 & 12\\
\rowcolor{LighterBlue}
\custombold{Ore totali} & 7 & 7 & 2 & 20 & 31 & 12 & 79\\
\end{tabular}
        \captionof{table}{Preventivo ore Periodo Sviluppo Requisiti Obligatori}
\label{tab:PSRO}
\end{center}

\begin{figure}[h]
    \centering
\includegraphics[width=17cm, height=10cm]{documenti/grafici/Divisione_ore_lavorative_Sviluppo_Requisiti_Obbligatori.png}    \caption{Grafico divisione ore lavorative periodo di Sviluppo Requisiti Obbligatori}
    \label{fig:PSRO}
\end{figure}

\newpage
In questo periodo i costi da affrontare sono:
\begin{center}
    \begin{tabular}{c|c|c}
    \rowcolor{Blue}
    \custombold{Ruolo} & \custombold{Ore} & \custombold{Costo \euro}\\
    \rowcolor{LighterBlue}
    Responsabile & 7 & 210\\
    \rowcolor{LightBlue}
    Amministratore & 7 & 140\\
    \rowcolor{LighterBlue}
    Analista & 2 & 50\\
    \rowcolor{LightBlue}
    Progettista & 20 & 500\\
    \rowcolor{LighterBlue}
    Programmatore & 31 & 465\\
    \rowcolor{LightBlue}
    Verificatore & 12 & 180\\
    \rowcolor{LighterBlue}
    \custombold{Totale} & \custombold{79} & \custombold{1545}\\
    \end{tabular}
    \captionof{table}{Preventivo costi Periodo Sviluppo Requisiti Obligatori}
\label{tab:costiPSRO}
\end{center}

\begin{figure}[h]
    \centering
    \includegraphics[width=17cm, height=10cm]{documenti/grafici/Torta_percentuale_costi_Sviluppo_Requisiti_Obbligatori.jpg}    
    \caption{Grafico della divisione percentuale dei costi sostenuti nel periodo di Sviluppo Requisiti Obbligatori}
    \label{fig:costiPSRO}
\end{figure}

\newpage

\subsection{Riepilogo economico e delle ore parziale - Periodo Sviluppo Requisiti Opzionali}
Durante il periodo di Sviluppo Requisiti Opzionali, ciascun membro assumerà i ruoli secondo la seguente distribuzione:\\
\\
\begin{center}
\begin{tabular}{c|c|c|c|c|c|c|c}
\rowcolor{Blue}
\custombold{Nominativo} & \custombold{Re} & \custombold{Am} & \custombold{An} & \custombold{Pt} & \custombold{Pr} & \custombold{Ve} & \custombold{Ore Totali}\\
\hline
\rowcolor{LighterBlue}
Sabrina Caniato & 0 & 3 & 0 & 0 & 4 & 2 & 9\\
\rowcolor{LightBlue}
Giulia Dentone & 0 & 0 & 2 & 0 & 5 & 2 & 9\\
\rowcolor{LighterBlue}
Nicola Lazzarin & 5 & 0 & 0 & 0 & 5 & 2 & 12\\
\rowcolor{LightBlue}
Giovanni Moretti & 0 & 0 & 0 & 4 & 5 & 2 & 11\\
\rowcolor{LighterBlue}
Andrea Rezzi & 0 & 2 & 0 & 3 & 4 & 2 & 11\\
\rowcolor{LightBlue}
Samuele Vignotto & 0 & 0 & 0 & 4 & 4 & 2 & 10\\
\rowcolor{LighterBlue}
\custombold{Ore totali} & 5 & 5 & 2 & 11 & 27 & 12 & 62\\
\end{tabular}
   \captionof{table}{Preventivo ore Periodo Sviluppo Requisiti Opzionali}
\label{tab:PSROp}
\end{center}
\begin{figure}[h]
    \centering
    \includegraphics[width=17cm, height=10cm]{documenti/grafici/Divisione_ore_lavorative_Sviluppo_Requisiti_Opzionali.png}\caption{ Grafico divisione ore lavorative periodo di Sviluppo Requisiti Opzionali}
    \label{fig:PSROp}
\end{figure}

\newpage
In questo periodo i costi da affrontare sono:
\begin{center}
    \begin{tabular}{c|c|c}
    \rowcolor{Blue}
    \custombold{Ruolo} & \custombold{Ore} & \custombold{Costo \euro}\\
    \rowcolor{LighterBlue}
    Responsabile & 5 & 150\\
    \rowcolor{LightBlue}
    Amministratore & 5 & 100\\
    \rowcolor{LighterBlue}
    Analista & 2 & 50\\
    \rowcolor{LightBlue}
    Progettista & 11 & 275\\
    \rowcolor{LighterBlue}
    Programmatore & 27 & 405\\
    \rowcolor{LightBlue}
    Verificatore & 12 & 180\\
    \rowcolor{LighterBlue}
    \custombold{Totale} & \custombold{62} & \custombold{1160}\\
    \end{tabular}
       \captionof{table}{Preventivo costi Periodo Sviluppo Requisiti Opzionali}
\label{tab:costiPSROp}
\end{center}
\begin{figure}[h]
    \centering
    \includegraphics[width=17cm, height=10cm]{documenti/grafici/Torta_percentuale_costi_Sviluppo_Requisiti_Opzionali.jpg} \caption{Grafico della divisione percentuale dei costi sostenuti nel periodo di Sviluppo Requisiti Opzionali}
    \label{fig:costiPSROp}
\end{figure}

\newpage

\subsection{Riepilogo economico e delle ore parziale - Periodo Validazione E Collaudo}
Durante il periodo di Validazione e Collaudo, ciascun membro assumerà i ruoli secondo la seguente distribuzione:\\
\\
\begin{center}
\begin{tabular}{c|c|c|c|c|c|c|c}
\rowcolor{Blue}
\custombold{Nominativo} & \custombold{Re} & \custombold{Am} & \custombold{An} & \custombold{Pt} & \custombold{Pr} & \custombold{Ve} & \custombold{Ore Totali}\\
\hline
\rowcolor{LighterBlue}
Sabrina Caniato & 3 & 2 & 0 & 0 & 5 & 2 & 12\\
\rowcolor{LightBlue}
Giulia Dentone & 0 & 0 & 0 & 6 & 5 & 2 & 13\\
\rowcolor{LighterBlue}
Nicola Lazzarin & 0 & 0 & 0 & 3 & 7 & 2 & 12\\
\rowcolor{LightBlue}
Giovanni Moretti & 0 & 4 & 0 & 3 & 3 & 2 & 12\\
\rowcolor{LighterBlue}
Andrea Rezzi & 4 & 2 & 0 & 2 & 2 & 2 & 12\\
\rowcolor{LightBlue}
Samuele Vignotto & 0 & 0 & 0 & 3 & 6 & 2 & 11\\
\rowcolor{LighterBlue}
\custombold{Ore totali} & 7 & 8 & 0 & 17 & 28 & 12 & 72\\
\end{tabular}
       \captionof{table}{Preventivo ore Periodo Validazione e Collaudo}
\label{tab:PVC}
\end{center}

\begin{figure}[h]
    \centering
    \includegraphics[width=17cm, height=10cm]{documenti/grafici/Divisione_ore_lavorative_Validazione_e_Collaudo.png}    \caption{Grafico divisione ore lavorative periodo di Validazione e Collaudo}
    \label{fig:PVC}
\end{figure}

\newpage
In questo periodo i costi da affrontare sono:
\begin{center}
    \begin{tabular}{c|c|c}
    \rowcolor{Blue}
    \custombold{Ruolo} & \custombold{Ore} & \custombold{Costo \euro}\\
    \rowcolor{LighterBlue}
    Responsabile & 7 & 210\\
    \rowcolor{LightBlue}
    Amministratore & 8 & 160\\
    \rowcolor{LighterBlue}
    Analista & 0 & 0\\
    \rowcolor{LightBlue}
    Progettista & 17 & 425\\
    \rowcolor{LighterBlue}
    Programmatore & 28 & 420\\
    \rowcolor{LightBlue}
    Verificatore & 12 & 180\\
    \rowcolor{LighterBlue}
    \custombold{Totale} & \custombold{72} & \custombold{1395}\\
    \end{tabular}
    \captionof{table}{Preventivo costi Periodo Validazione e Collaudo}
\label{tab:costiPVC}
\end{center}

\begin{figure}[h]
    \centering
    \includegraphics[width=17cm, height=10cm]{documenti/grafici/Torta_percentuale_costi_Validazione_e_Collaudo.jpg}    \caption{Grafico della divisione percentuale dei costi sostenuti nel periodo di Validazione e Collaudo}
    \label{fig:costiPVC}
\end{figure}

\newpage

\subsection{Riepilogo economico e delle ore totale}
I costi e le ore totali vengono riassunti nella seguente tabella:
\begin{center}
    \begin{tabular}{c|c|c|c|c}
    \rowcolor{Blue}
    \custombold{Ruolo} & \custombold{Costo orario} & \custombold{Ore per ruolo} & \custombold{Ore per membro} & \custombold{Costo totale}\\
    \rowcolor{LighterBlue}
    Responsabile & 30 & 72 & 12 & 2160\\
    \rowcolor{LightBlue}
    Amministratore & 20 & 72 & 12 & 1440\\
    \rowcolor{LighterBlue}
    Analista & 25 & 132 & 22 & 3300\\
    \rowcolor{LightBlue}
    Progettista & 25 & 78 & 13 & 1950\\
    \rowcolor{LighterBlue}
    Programmatore & 15 & 126 & 21 & 1890\\
    \rowcolor{LightBlue}
    Verificatore & 15 & 90 & 15 & 1350\\
    \rowcolor{LighterBlue}
    \custombold{Totale} & - & 570 & 95 & 12090\\
    \end{tabular}
    \captionof{table}{Preventivo ore totali}
\label{tab:ore}
\end{center}
\newpage

\section{Consuntivo}
Questa sezione presenta le spese effettivamente sostenute dal team Cyber Sorceres. Vengono dettagliate le ore e i costi associati a ciascun ruolo per l'esecuzione delle attività pianificate. Inoltre, è fornito un bilancio finanziario che rappresenta la differenza tra il consuntivo di periodo e il preventivo. Tale bilancio può assumere le seguenti condizioni:
\begin{itemize}
    \item \custombold{Positivo}: se la spesa effettiva è inferiore a quanto preventivato;
    \item \custombold{Pareggio}: se la spesa effettiva è uguale a quanto preventivato;
    \item \custombold{Negativo}: se la spesa effettiva è maggiore a quanto preventivato.
\end{itemize}
Il bilancio è indicato tra parentesi accanto ai valori rilevati dal consuntivo di periodo. Se il valore tra parentesi è assente, ciò indica che l'aspettativa del preventivo è stata rispettata.

\subsection{Periodo di Analisi}
\subsubsection{Variazione della pianificazione}
\begin{center}
\begin{tabular}{c|c|c|c|c|c|c|c}
\rowcolor{Blue}
\custombold{Nominativo} & \custombold{Re} & \custombold{Am} & \custombold{An} & \custombold{Pt} & \custombold{Pr} & \custombold{Ve} & \custombold{Ore Totali}\\
\hline
\rowcolor{LighterBlue}
Sabrina Caniato & 9 (+2) & 0 & 14 & 0 & 0 & 5 & 28\\
\rowcolor{LightBlue}
Giulia Dentone & 0 & 12 & 13(-3) & 0 & 0 & 5 & 30\\
\rowcolor{LighterBlue}
Nicola Lazzarin & 0 & 12(-2) & 14 & 0 & 0 & 5 & 31\\
\rowcolor{LightBlue}
Giovanni Moretti & 12 & 0 & 15(+2) & 0 & 0 & 5 & 32\\
\rowcolor{LighterBlue}
Andrea Rezzi & 8 & 0 & 14(-2) & 0 & 0 & 5 & 27\\
\rowcolor{LightBlue}
Samuele Vignotto & 0 & 12(+1) & 11 & 0 & 0 & 5 & 28\\
\rowcolor{LighterBlue}
\custombold{Ore totali} & 29(+2) & 36(-1) & 81(-3) & 0 & 0 & 30 & 176(-2)\\
\end{tabular}
\captionof{table}{Variazione pianificazione nell'analisi}
\label{tab:varPian}
\end{center}
\subsubsection{Variazione dei costi}
\begin{center}
    \begin{tabular}{c|c|c}
    \rowcolor{Blue}
    \custombold{Ruolo} & \custombold{Ore} & \custombold{Costo \euro}\\
    \rowcolor{LighterBlue}
    Responsabile & 29(+2) & 870(+50)\\
    \rowcolor{LightBlue}
    Amministratore & 36(-1) & 720(-20)\\
    \rowcolor{LighterBlue}
    Analista & 81(-3) & 2025(-75)\\
    \rowcolor{LightBlue}
    Progettista & 0 & 0\\
    \rowcolor{LighterBlue}
    Programmatore & 0 & 0\\
    \rowcolor{LightBlue}
    Verificatore & 30 & 450\\
    \rowcolor{LighterBlue}
    \custombold{Totale} & \custombold{176(-2)} & \custombold{4065(-35)}\\
    \end{tabular}
    \captionof{table}{Variazione costi nell'analisi}
\label{tab:varCosti}
\end{center}
\subsubsection{Ragione degli scostamenti}
\begin{itemize}
    \item \custombold{Responsabile}: inesperienza nel ricoprire tale ruolo;
    \item \custombold{Amministratore}: esperienze pregresse di un membro del gruppo;
    \item \custombold{Analista}: il proponente ha fornito al gruppo molto supporto nella stesura dell'\textit{Analisi dei Requisiti}.
\end{itemize}
\subsubsection{Considerazioni rispetto al preventivo}
Il bilancio evidenzia un risultato positivo rispetto al preventivo per questo periodo. Tuttavia, non si ritiene necessaria alcuna ripianificazione per il prossimo periodo, poiché l'importo risparmiato non è considerato significativo. Inoltre, poiché sono stati raggiunti tutti gli obiettivi precedentemente pianificati, non si è verificato alcun rallentamento nell'avanzamento delle attività.
\subsubsection{Retrospettive}
\begin{itemize}
    \item \custombold{Comunicazione e supporto del proponente}:  La disponibilità e il supporto offerto dal proponente durante la stesura dell'Analisi dei Requisiti hanno giocato un ruolo significativo nel facilitare il processo. Questo suggerisce l'importanza di mantenere una comunicazione aperta e collaborativa con gli stakeholder\textsubscript{G} esterni per ottenere informazioni cruciali e supporto durante le fasi di progetto.
    \item \custombold{Esperienze precedenti e ruoli assegnati}:  Le esperienze pregresse di alcuni membri del team hanno influito positivamente sui ruoli di Amministratore e Responsabile. Tuttavia, l'inesperienza nel ricoprire certi ruoli, ha portato a un lieve scostamento nell'utilizzo delle risorse. È fondamentale considerare attentamente l'esperienza e le competenze dei membri del team quando si assegnano i ruoli, cercando un equilibrio tra competenze acquisite e opportunità di crescita.
    \item \custombold{Risparmio nei costi e risultati positivi}: Nonostante alcuni scostamenti nelle ore pianificate per alcuni ruoli, il bilancio ha riportato un risultato positivo rispetto al preventivo per il periodo di analisi. Questo suggerisce una gestione efficiente delle risorse e un'attenta pianificazione delle attività. Tuttavia, è importante mantenere la vigilanza sui costi e considerare attentamente ogni variazione rispetto al preventivo, anche se non significativa.
    \item \custombold{Nessuna ripianificazione necessaria}: Nonostante il risultato positivo, non è stata ritenuta necessaria alcuna ripianificazione per il prossimo periodo, poiché il risparmio non è stato considerato significativo e tutti gli obiettivi precedentemente pianificati sono stati raggiunti.
\end{itemize}
In sintesi, le retrospettive del periodo di analisi indicano la necessità di mantenere una comunicazione efficace con gli stakeholder, valutare attentamente le competenze dei membri del team quando si assegnano i ruoli, monitorare da vicino i costi e le risorse e rimanere flessibili nel adattarsi a eventuali cambiamenti nelle attività future.


\subsection{Periodo di Progettazione Technology Baseline\textsubscript{G}}
\subsubsection{Variazione della pianificazione}
\begin{center}
\begin{tabular}{c|c|c|c|c|c|c|c}
\rowcolor{Blue}
\custombold{Nominativo} & \custombold{Re} & \custombold{Am} & \custombold{An} & \custombold{Pt} & \custombold{Pr} & \custombold{Ve} & \custombold{Ore Totali}\\
\hline
\rowcolor{LighterBlue}
Sabrina Caniato & 0 & 0 & 10 & 5 & 0 & 2 & 17\\
\rowcolor{LightBlue}
Giulia Dentone & 0 & 0(+5) & 6 & 2 & 0 & 2 & 10(+5)\\
\rowcolor{LighterBlue}
Nicola Lazzarin & 0 & 0 & 7 & 3 & 0 & 2 & 12\\
\rowcolor{LightBlue}
Giovanni Moretti & 0 & 3(+5) & 4 & 1(+1) & 0 & 2 & 10(+6)\\
\rowcolor{LighterBlue}
Andrea Rezzi & 0 & 8 & 4 & 3 & 0 & 2 & 17\\
\rowcolor{LightBlue}
Samuele Vignotto & 12 & 0 & 1 & 1(+1) & 0 & 2 & 16(+1)\\
\rowcolor{LighterBlue}
\custombold{Ore totali} & 12 & 11(+10) & 32 & 15(+2) & 0 & 12 & 82(+12)\\
\end{tabular}
\captionof{table}{Variazione pianificazione del periodo di progettazione Technology Beseline}
\label{tab:varPTB}
\end{center}
\subsubsection{Variazione dei costi}
\begin{center}
    \begin{tabular}{c|c|c}
    \rowcolor{Blue}
    \custombold{Ruolo} & \custombold{Ore} & \custombold{Costo \euro}\\
    \rowcolor{LighterBlue}
    Responsabile & 12 & 360\\
    \rowcolor{LightBlue}
    Amministratore & 11(+10) & 220(+200)\\
    \rowcolor{LighterBlue}
    Analista & 32 & 800\\
    \rowcolor{LightBlue}
    Progettista & 15(+2) & 375(+50)\\
    \rowcolor{LighterBlue}
    Programmatore & 0 & 0\\
    \rowcolor{LightBlue}
    Verificatore & 12 & 180\\
    \rowcolor{LighterBlue}
    \custombold{Totale} & \custombold{82(+12)} & \custombold{1935(+250)}\\
    \end{tabular}
    \captionof{table}{Variazione costi del periodo di progettazione Technology Beseline}
\label{tab:varPTB}
\end{center}
\subsubsection{Ragione degli scostamenti}
\begin{itemize}
    \item \custombold{Amministratore}: difficoltà organizzative interne al gruppo.
    \item \custombold{Progettista}: nessuna esperienza con gli strumenti forniti dal proponente.
\end{itemize}
\subsubsection{Considerazioni rispetto al preventivo}
Il bilancio risulta negativo, ma il gruppo conta di recuperare nelle prossime fasi del progetto poiché ha maggiore esperienza.
\subsubsection{Retrospettive}
\begin{itemize}
    \item \custombold{Difficoltà organizzative interne}: La variazione significativa nelle ore pianificate per il ruolo di Amministratore indica difficoltà organizzative interne al gruppo. È importante adottare misure correttive per migliorare la gestione delle attività amministrative, ad esempio attraverso la definizione di procedure più efficienti o l'assegnazione di responsabilità più chiare.
    \item \custombold{Mancanza di esperienza sugli strumenti}: La mancanza di esperienza con gli strumenti forniti dal proponente ha influenzato le ore pianificate per il ruolo di Progettista. Questo sottolinea l'importanza di acquisire familiarità con gli strumenti e le tecnologie necessarie per lo sviluppo del progetto fin dalle fasi iniziali. Potrebbe essere utile investire tempo in formazione e apprendimento autonomo per ridurre il divario di competenze e migliorare l'efficienza nelle fasi successive del progetto.
    \item \custombold{Risultato negativo nel bilancio}: Nonostante le aspettative, il bilancio per il periodo di progettazione Technology Baseline è risultato negativo. È importante analizzare le cause di questo scostamento e adottare misure correttive per mitigare gli impatti finanziari negativi nelle fasi successive del progetto. Questo potrebbe includere la revisione delle pianificazioni, l'ottimizzazione delle risorse e la riduzione dei costi non essenziali.
    \item \custombold{Ottimismo nel recupero}: Nonostante il bilancio negativo, il gruppo esprime ottimismo nel recuperare nelle prossime fasi del progetto grazie all'accumulo di esperienza. È importante mantenere una visione realistica delle proprie capacità e impegnarsi attivamente nel miglioramento continuo.
\end{itemize}
In conclusione, le retrospettive del periodo di progettazione Technology Baseline evidenziano la necessità di affrontare le difficoltà organizzative interne, acquisire competenze sugli strumenti necessari e adottare misure correttive per gestire i risultati finanziari negativi. Questi apprendimenti possono contribuire a guidare il gruppo verso una maggiore efficienza e successo nelle fasi successive del progetto.

\subsection{Periodo di Sviluppo Proof of Concept\textsubscript{G}}
\subsubsection{Prospetto orario}
\begin{center}
\begin{tabular}{c|c|c|c|c|c|c|c}
\rowcolor{Blue}
\custombold{Nominativo} & \custombold{Re} & \custombold{Am} & \custombold{An} & \custombold{Pt} & \custombold{Pr} & \custombold{Ve} & \custombold{Ore Totali}\\
\hline
\rowcolor{LighterBlue}
Sabrina Caniato & 0 & 0 & 1 & 3 & 6 & 2 & 12\\
\rowcolor{LightBlue}
Giulia Dentone & 12 & 0 & 1 & 1 & 5(-3) & 2 & 21\\
\rowcolor{LighterBlue}
Nicola Lazzarin & 0 & 0 & 1 & 3 & 9 & 2 & 15\\
\rowcolor{LightBlue}
Giovanni Moretti & 0 & 5 & 1 & 2(-1) & 7 & 2 & 17\\
\rowcolor{LighterBlue}
Andrea Rezzi & 0 & 0 & 3 & 3 & 8 & 2 & 16\\
\rowcolor{LightBlue}
Samuele Vignotto & 0 & 0 & 8 & 3(-1) & 5(-1) & 2 & 18\\
\rowcolor{LighterBlue}
\custombold{Ore totali} & 12 & 5 & 15 & 15(-2) & 40(-4) & 12 & 99(-6)\\
\end{tabular}
\captionof{table}{Variazione pianificazione del periodo di sviluppo del proof of concept\textsubscript{G}}
\label{tab:varPOC}
\end{center}
\subsubsection{Prospetto economico}
\begin{center}
    \begin{tabular}{c|c|c}
    \rowcolor{Blue}
    \custombold{Ruolo} & \custombold{Ore} & \custombold{Costo \euro}\\
    \rowcolor{LighterBlue}
    Responsabile & 12 & 360\\
    \rowcolor{LightBlue}
    Amministratore & 5 & 100\\
    \rowcolor{LighterBlue}
    Analista & 15 & 375\\
    \rowcolor{LightBlue}
    Progettista & 15(-2) & 375(-50)\\
    \rowcolor{LighterBlue}
    Programmatore & 40(-4) & 600(-60)\\
    \rowcolor{LightBlue}
    Verificatore & 12 & 180\\
    \rowcolor{LighterBlue}
    \custombold{Totale} & \custombold{99(-6)} & \custombold{1990(-110)}\\
    \end{tabular}
    \captionof{table}{Variazione costi del periodo di sviluppo del proof of concept\textsubscript{G}}
\label{tab:varcostiPOC}
\end{center}
\subsubsection{Ragione degli scostamenti}
\begin{itemize}
    \item \custombold{Progettista}: maggiore efficienza dovuta all'esperienza dei membri del gruppo ricavata dalla precedente fase.
    \item \custombold{Programmatore}: efficienza migliore del previsto grazie alle esperienze pregresse del gruppo.
\end{itemize}
\subsubsection{Considerazioni rispetto al preventivo}
Il bilancio risulta positivo, il gruppo sta recuperando grazie alla maggiore esperienza.
\subsubsection{Retrospettive}
\begin{itemize}
    \item \custombold{Efficienza migliorata}: La variazione delle ore pianificate per i ruoli di Progettista e Programmatore indica un'efficienza migliore rispetto alle aspettative. Questo è attribuibile principalmente all'esperienza acquisita dai membri del gruppo durante le fasi precedenti del progetto.
    \item \custombold{Risparmio nei costi}: La maggiore efficienza ha portato anche a un risparmio nei costi, come evidenziato dalla riduzione delle ore pianificate e dei costi associati per i ruoli di Progettista e Programmatore. Questo suggerisce una gestione efficiente delle risorse e una migliore pianificazione delle attività, consentendo al gruppo di ottenere un bilancio positivo per il periodo di sviluppo Proof of Concept\textsubscript{G}.
    \item \custombold{Recupero positivo nel bilancio}: Nonostante alcuni scostamenti nelle ore pianificate, il bilancio per il periodo di sviluppo Proof of Concept\textsubscript{G} risulta positivo. Questo è un segnale incoraggiante e suggerisce che il gruppo sta recuperando dalle difficoltà incontrate nei periodi precedenti.
\end{itemize}
In conclusione, le retrospettive del periodo di sviluppo Proof of Concept\textsubscript{G} evidenziano l'importanza dell'esperienza pregressa nel migliorare le prestazioni del team, la gestione efficiente delle risorse e il recupero positivo nel bilancio.

\subsection{Periodo di Sviluppo Requisiti Obbligatori - a finire}
\subsection{Periodo di Sviluppo Requisiti Opzionali - a finire}
\subsection{Periodo di Validazione e Collaudo - a finire}

\newpage
\section{Mitigazione dei rischi}

\subsection{Rischi organizzativi interni}

\subsubsection{Assegnazione dei ruoli}
\begin{center}
\begin{tabular}{P{10em} P{20em}} 
    \rowcolor{LightBlue}
     Descrizione &  A causa della mancata esperienza del gruppo non siamo stati in grado di stimare correttamente il numero di ore necessario alla progettazione e allo sviluppo\\ 
    \rowcolor{LighterBlue}
    Mitigazione dei rischi &  Abbiamo approfondito il ruolo di ogni singola figura in modo da assegnare i ruoli coerentemente alle necessità\\
\end{tabular}
\captionof{table}{Mitigazione assegnzione dei ruoli}
\label{tab:mitruoli}
\end{center}

\subsection{Momenti di stallo}
\begin{center}
\begin{tabular}{P{10em} P{20em}} 
    \rowcolor{LightBlue}
     Descrizione &  A causa della mancata esperienza del gruppo possono esserci momenti di stallo nell'avanzamento non opportunamente e tempestivamente affrontati.\\ 
    \rowcolor{LighterBlue}
    Mitigazione dei rischi &  Abbiamo deciso di mantenere sprint lunghi due settimane in modo da favorire il confronto tra i membri sulla risoluzione degli stalli.\\
\end{tabular}
\captionof{table}{Mitigazione dei momenti di stallo}
\label{tab:mitruoli}
\end{center}

\subsubsection{Impegni accademici}
\begin{center}
\begin{tabular}{P{10em} P{20em}} 
    \rowcolor{LightBlue}
     Descrizione & Abbiamo riscontato difficoltà nell'essere tutti contemporaneamente presenti a causa delle sessioni di esame o tirocinio. Inoltre abbiamo incontrato diverse necessità accademiche per quanto riguarda le tempistiche di consegna. \\ 
    \rowcolor{LighterBlue}
    Mitigazione dei rischi & Abbiamo distribuito equamente le attività in modo tale che il monte ore e l'impegno fosse coerente con quanto specificato nel Preventivo e nel Consuntivo.  \\
\end{tabular}
\captionof{table}{Mitigazione impegni accademici}
\label{tab:mitimpegni}
\end{center}

\subsubsection{Scarsa velocità da parte del cliente}
\begin{center}
\begin{tabular}{P{10em} P{20em}} 
    \rowcolor{LightBlue}
     Descrizione & Abbiamo riscontato difficoltà nell'ottenere rapidamente gli stumenti necessari al gruppo soprattutto nella fase di sviluppo. \\ 
    \rowcolor{LighterBlue}
    Mitigazione dei rischi & Abbiamo sollecitato il proponente durante un meeting esterno e abbiamo rafforzato i canali di comunicazione. \\
\end{tabular}
\captionof{table}{Mitigazione velocità del cliente}
\label{tab:mitcliente}
\end{center}

\subsection{Rischi organizzativi esterni}
\subsubsection{Scadenze}
\begin{center}
\begin{tabular}{P{10em} P{20em}} 
    \rowcolor{LightBlue}
     Descrizione & Rispettare le scadenze prefissate dal gruppo è risultato più difficile del previsto poichè, a causa dell'inesperienza, non ci si aspettava un simile carico di lavoro. \\ 
    \rowcolor{LighterBlue}
    Mitigazione dei rischi &  Abbiamo pianificato dettagliatamente l'avanzamento del progetto, prefissandoci Milestione più "realistiche", aumentandone il numero ma diminuendone la portata. Inoltre tutte le attività svolte sono state monitorate dal responsabile che interveniva celermente qualora necessario.\\
\end{tabular}
\captionof{table}{Mitigazione scadenze}
\label{tab:mitscadenze}
\end{center}

\subsubsection{Aumento dei costi}
\begin{center}
\begin{tabular}{P{10em} P{20em}} 
    \rowcolor{LightBlue}
     Descrizione &  Si potrebbero verificare degli aumenti dei costi di implementazione, a causa di eventuali richieste aggiuntive da parte del proponente o dati dall'inesperienza del gruppo nell'uso delle tecnologie richieste.\\ 
    \rowcolor{LighterBlue}
    Mitigazione dei rischi &  Abbiamo monitorato costantemente i costi del progetto e abbiamo richiesto degi incontri di formazione al proponente in modo tale da indirizzarci alle librerie a loro più vantaggiose in termini di costi e vantaggi.\\
\end{tabular}
\captionof{table}{Mitigazione costi}
\label{tab:mitcosti}
\end{center}


\subsection{Rischi tecnologici}


\subsubsection{Tecnologie}\begin{center}
\begin{tabular}{P{10em} P{20em}} 
    \rowcolor{LightBlue}
     Descrizione & Gli strumenti necessari per una buona realizzazione del progetto sono in larga parte mai stati utilizzati precedentemente da parte dei membri del gruppo, tanto meno vi è esperienza nell'integrazione delle le varie tecnologie.\\ 
    \rowcolor{LighterBlue}
    Mitigazione dei rischi & Il gruppo si è impegnato in uno studio autonomo degli strumenti, approfondito tramite l'utilizzo di questi ultimi anche nei Proof of Concept\textsubscript{G}. Per l'integrazione si è deciso di focalizzarsi sull'integrazione delle tecnologie a partire dalle prime fasi del progetto.\\
\end{tabular}
\captionof{table}{Mitigazione tecnologie}
\label{tab:mittecnologie}
\end{center}




\end{document}