\documentclass{article}
\usepackage{graphicx}
\usepackage{float}  
\usepackage[utf8]{inputenc}
\usepackage[includeheadfoot, margin=1em,headheight=2em]{geometry}
\usepackage{titling}
\usepackage{hyperref}
\geometry{a4paper, left=2cm, right=2cm, top=2cm, bottom=2cm}
\usepackage{graphicx}
\providecommand{\versionnumber}{2.0.0}
\usepackage{enumitem}
\usepackage{array}
\usepackage[italian]{babel}
\newcolumntype{P}[1]{>{\centering\arraybackslash}p{#1}}
\renewcommand{\arraystretch}{1.5} % Default value: 1
\setlength{\droptitle}{-6em}
\usepackage{capt-of}
\usepackage{setspace}

%font
\usepackage[defaultfam,tabular,lining]{montserrat}
\usepackage[T1]{fontenc}
\renewcommand*\oldstylenums[1]{{\fontfamily{Montserrat-TOsF}\selectfont #1}}

%custom bold 
\usepackage[outline]{contour}
\usepackage{xcolor}
\newcommand{\custombold}{\contour{black}}

%table colors
\usepackage{color, colortbl}
\definecolor{Blue}{rgb}{0.51,0.68,0.79}
\definecolor{LightBlue}{rgb}{0.82,0.87,0.90}
\definecolor{LighterBlue}{rgb}{0.93,0.95,0.96}

\usepackage{caption}
\captionsetup[figure]{labelformat=empty}

%Header
\usepackage{fancyhdr, xcolor}
\pagestyle{fancy}
\let\oldheadrule\headrule% Copy \headrule into \oldheadrule
\renewcommand{\headrule}{\color{Blue}\oldheadrule}% Add colour to \headrule
\renewcommand{\headrulewidth}{0.2em}
\fancyhead[L]{Piano di qualifica}
\fancyhead[C]{Cybersorceres}
\fancyhead[R]{versione \versionnumber}

\title{\Huge{\textbf{Piano di Qualifica}}\vspace{-1em}}
\author{CyberSorcerers Team}
\date{}
\begin{document}
\maketitle
\vspace{-3em}
\begin{figure}[h]
  \centering
  \includegraphics[width=6cm, height=6cm]{documenti/logo rotondo.png}
  \label{fig:immagine}
\end{figure}


\begin{center}
    \begin{tabular}{|l c c|}
    \hline
        \rowcolor{Blue} 
        \textbf{Informazioni sul documento} & &\\ [1 ex]
        \hline
        \rowcolor{LighterBlue}
        Destinatari: & Prof Tullio Vardanega & Prof Riccardo Cardin\\ [1 ex]
        \hline
        \rowcolor{LightBlue}
        G al pedice: & Consultare il Glossario & \\ [1 ex]
        \hline
    \end{tabular}
\end{center}

\vspace{6em}
\begin{center}
    \begin{tabular}{|P{24em}|}
    \hline
        \rowcolor{Blue}
        \textbf{Membri del team:}\\
        \hline
        \rowcolor{LighterBlue}
        \custombold{Sabrina Caniato}\\
        \hline
        \rowcolor{LightBlue}
        \custombold{Giulia Dentone}\\
        \hline
        \rowcolor{LighterBlue}
        \custombold{Nicola Lazzarin}\\
        \hline
        \rowcolor{LightBlue}
        \custombold{Giovanni Moretti}\\
        \hline
        \rowcolor{LighterBlue}
        \custombold{Andrea Rezzi}\\
        \hline
        \rowcolor{LightBlue}
        \custombold{Samuele Vignotto}\\
        \hline
    \end{tabular}
\end{center}

    
\newpage

\textbf{Registro dei Cambiamenti - Changelog\textsubscript{G}}
\begin{center}
\begin{tabular}{P{4em} P{6em} P{8em} P{8em} P{10em}} 
  \rowcolor{Blue}
    \custombold{Versione} & \custombold{Data} & \custombold{Autore} &
    \custombold{ Verificatore} & \custombold{Dettaglio}\\
    \rowcolor{LighterBlue}
    2.0.0 & 27/05/2024 & Samuele Vignotto & Giovanni Moretti & Controllo finale del documento.\\
    \rowcolor{LightBlue}
    1.4.1 & 24/05/2024 & Samuele Vignotto & Giulia Dentone & Aggiornamento grafici.\\
    \rowcolor{LighterBlue}
    1.4.0 & 23/05/2024 & Nicola Lazzarin & Andrea Rezzi & Aggiunta test di accettazione.\\
    \rowcolor{LightBlue}
    1.3.0 & 22/05/2024 & Nicola Lazzarin & Andrea Rezzi & Aggiunti i test di integrazione\\
    \rowcolor{LighterBlue}
    1.2.0 & 20/05/2024 & Nicola Lazzarin & Andrea Rezzi & Aggiunti i test di unità.\\
    \rowcolor{LightBlue}
    1.1.1 & 06/05/2024 & Giulia Dentone & Samuele Vignotto & Corretta ed estesa la sezione di Valutazione per il miglioramento. \\ 
    \rowcolor{LighterBlue}
    1.1.0 & 04/05/2024 & Samuele Vignotto & Giulia Dentone & Aggiunte conclusione grafici.\\
    \rowcolor{LightBlue}
    1.0.0 & 09/04/2024 & Samuele Vignotto & Giovanni Moretti & Verifica documento per approvazione al rilascio.\\
    \rowcolor{LighterBlue}
    0.4.1 & 19/03/2024 & Samuele Vignotto & Nicola Lazzarin & Aggiunta tabella test\textsubscript{G} di sistema.\\
    \rowcolor{LightBlue}
    0.4.0 & 08/03/2024 & Giulia Dentone & Samuele Vignotto & Inizio stesura della sezione testing\textsubscript{G}.\\
    \rowcolor{LighterBlue}
    0.3.3 & 22/01/2024 & Sabrina Caniato & Giulia Dentone & Update della qualità di processo.\\
    \rowcolor{LightBlue}
    0.3.2 & 09/01/2024 & Giovanni Moretti & Nicola Lazzarin & Update delle valutazioni per il miglioramento.\\
    \rowcolor{LighterBlue}
    0.3.1 & 02/01/2024 & Andrea Rezzi & Sabrina Caniato & Update delle valutazioni. per il miglioramento \\
    \rowcolor{LightBlue}
    0.3.0 & 21/12/2023 & Andrea Rezzi & Sabrina Caniato & Aggiunta delle valutazioni per il miglioramento. \\
    \rowcolor{LighterBlue}
    0.2.1 & 18/12/2023 & Giulia Dentone & Samuele Vignotto & Update della qualità di prodotto. \\
    \rowcolor{LightBlue}
    0.2.0 & 18/12/2023 & Samuele Vignotto & Giovanni Moretti & Descrizione della qualità di prodotto.\\
\end{tabular}
\end{center}

\begin{center}
\begin{tabular}{P{4em} P{6em} P{8em} P{8em} P{10em}} 
    \rowcolor{LighterBlue}
    0.1.0 & 17/12/2023 & Sabrina Caniato & Andrea Rezzi & Descrizione della qualità di processo. \\
    \rowcolor{LightBlue}
    0.0.1 & 14/12/2023 & Giovanni Moretti & Nicola Lazzarin &  Definizione struttura del documento e scheletro delle sezioni. Scrittura introduzione ed obiettivi delle diverse sezioni.\\
\end{tabular}
\end{center}
\newpage
\tableofcontents
\listoffigures
\listoftables

\newpage
\section{Introduzione e scopo}
Il Piano di Qualifica è un documento che ci prefissiamo di aggiornare periodicamente dato che definisce l'approccio e le strategie per garantire la qualità di un progetto software. Questo piano è parte integrante del processo di gestione della qualità e fornisce una linea guida dettagliata su come il controllo e l'assicurazione della qualità verranno implementati durante l'intero ciclo di vita\textsubscript{G} del progetto.\\
In questo documento cercheremo di definire delle metriche di misurazione dell'efficacia e dell'efficienza del progetto, in base anche agli accorgimenti forniti dal proponente.\\

Il piano di qualifica conterrà:

\begin{itemize}
    \item Definizione chiara degli obiettivi e delle metriche di qualità che il progetto propone di raggiungere.
        
    \item Specifica dei criteri che determineranno se il prodotto soddisfa gli standard di qualità stabiliti.
    
    \item Descrizione dettagliata dei processi di test\textsubscript{G} che saranno implementati e la definizione delle strategie utilizzate per l'esecuzione di essi.

    \item Procedure per gestire eventuali deviazioni rispetto agli standard di qualità pianificati.

\end{itemize}

\subsection{Glossario}
I termini impiegati in questo testo potrebbero suscitare incertezze circa il loro significato, rendendo quindi necessaria una definizione per evitare ambiguità. Tali termini sono identificati da una lettera "G" maiuscola posta in pedice alla parola, e la loro spiegazione è fornita nel Glossario v1.0.0.

\subsection{Riferimenti}
\textbf{Riferimenti normativi}
\begin{itemize}
    \item \href{https://www.math.unipd.it/~tullio/IS-1/2023/Progetto/C7.pdf}{C7.pdf}
\end{itemize}
\textbf{Riferimenti informativi}
\begin{itemize}
    \item Lezione del corso di Ingegneria del software "Qualità del software": \\ \\
    \url{https://www.math.unipd.it/~tullio/IS-1/2023/Dispense/T7.pdf}
    \item Lezione del corso di Ingegneria del software "Qualità di processo": \\ \\
    \url{https://www.math.unipd.it/~tullio/IS-1/2023/Dispense/T8.pdf}
    \item Lezione del corso di Ingeneria del software "Verifica e validazione: introduzione": \\ \\
    \url{https://www.math.unipd.it/~tullio/IS-1/2023/Dispense/T9.pdf}
    \item ISOG/IECG 9126:2001 SWE Product Quality \\ \\
    \url{https://www.iso.org/standard/22749.html}
    \item ISO/IEC 14598:1999 SW Product Evaluation \\ \\
    \url{https://www.iso.org/standard/24902.html}
    \item  ISO/IEC 25000:2005 SQuaRE: Systems and software Quality Requirements and Evaluation 
        \begin{itemize}
        \item 25010:2011 Quality model;
        \item 25020:2019 Quality measurement framework;
        \item 25030:2007 Quality requirements;
        \item 25040:2011 Quality evaluation.
    \end{itemize}
    \url{https://cdn.standards.iteh.ai/samples/35683/917c6ad92a6e4c6c9326547e53f2dd7a/ISO-IEC-25000-2005.pdf}
    \item ISO 9000:2015 \\ \\
    \url{https://coopcrea.it/wp-new/wp-content/uploads/2017/10/ISO-UNI-9001_2015-Fondamenti-e-vocabolario.pdf}
    \item ISO 9004:2018 \\ \\
    \url{https://parsetraining.com/wp-content/uploads/2018/07/ISO-9004-2018.pdf}
    \item ISO/IEC 33020:2019 \\ \\
    \url{https://cdn.standards.iteh.ai/samples/78526/e84f5951f904440092d79e0e881c1122/ISO-IEC-33020-2019.pdf}
\end{itemize}

\section{Qualità di processo}
\subsection{Scopo ed obiettivi}
La qualità è determinata univocamente dai processi che compongono un prodotto, misurata attraverso che permettano di valutare tali processi e accertarsi che siano conformi agli obiettivi di qualità previsti. Da mettere in atto è Ciclo PDCA (Plan - Do - Check- Act)\textsubscript{G}, che garantisce un miglioramento continuo nell’utilizzo dei processi e delle risorse tramite una prima fase di pianificazione, seguita da una verifica con le metriche previste e infine un'integrazione o correzione del prodotto in base ai risultati precedentemente ottenuti.\\
\\
\begin{table}[h]
\centering
\begin{tabular}{|P{10em} |P{23em}| P{10em}|}
\hline
\rowcolor{Blue}
 \custombold{Obiettivo} & \custombold{Descrizione} & \custombold{Metriche} \\
 \hline
 \rowcolor{LightBlue}
 & \custombold{Processi primari} &\\
 \hline
 \rowcolor{LighterBlue}
  \custombold{Fornitura} & Procedura che implica la selezione di metodologie e risorse adeguate per soddisfare le esigenze del cliente. & \custombold{MC01}, \custombold{MC02}, \custombold{MC03}, \custombold{MC04}, \custombold{MC05}, \custombold{MC06}, \custombold{MC07}\\
  \hline
\rowcolor{LightBlue}
\custombold{Sviluppo} & Procedura volta a creare un prodotto software che risponda alle necessità del cliente. & \custombold{MC08} \\
\hline
\rowcolor{LighterBlue}
& \custombold{Processi di supporto} &\\
\hline
\rowcolor{LightBlue}
 \custombold{Verifica} & Procedura mirata a verificare che ogni servizio realizzato soddisfi i requisiti specificati. & \custombold{MC09} \\
 \hline
\rowcolor{LighterBlue}
 \custombold{Gestione della} \custombold{qualità} & Procedura volta a garantire la conformità del prodotto e dei servizi offerti agli standard predefiniti. & \custombold{MC10} \\
\hline
\rowcolor{LightBlue}
& \custombold{Processi organizzativi} &\\
\hline
\rowcolor{LighterBlue}
 \custombold{Gestione} \custombold{organizzativa} & Procedura dedicata a delineare le modalità di coordinamento del team. & \custombold{MC11} \\
 \hline
\end{tabular}
\caption{Tabella dei processi}
\label{tab:processi}
\end{table}

\newpage
\subsection{Processi primari}
\begin{tabular}{|P{5em}| P{18em}| P{8em}| P{8em}|}
\hline
\rowcolor{Blue}
 \custombold{Codice} & \custombold{Nome metrica} & \custombold{Valore} \custombold{accettabile} & \custombold{Valore ottimale}\\
 \hline
\rowcolor{LightBlue}
\multicolumn{4}{|c|}{\custombold{Fornitura}}\\
\hline
\rowcolor{LighterBlue}
\custombold{MC01} & Earned Value (EV) & $>0$ & $\leq EAC$ \\
\hline
\rowcolor{LightBlue}
\custombold{MC02} & Actual Cost (AC) & $\geq0$ & $\leq EAC$ \\
\hline
\rowcolor{LighterBlue}
\custombold{MC03} & Planned Value (PV) & $\geq0$ & $\leq BAC$ \\
\hline
\rowcolor{LightBlue}
\custombold{MC04} & Cost Variance (CV) & $\geq-10\%$ & $\geq0\%$ \\
\hline
\rowcolor{LighterBlue}
\custombold{MC05} & Schedule Variance (SV) & $\geq-10\%$ & $\geq0\%$ \\
\hline
\rowcolor{LightBlue}
\custombold{MC06} & Estimated At Completion (EAC) & $\geq BAC-3\%$; $\leq BAC+3\%$ & $=BAC$ \\
\hline
\rowcolor{LighterBlue}
\custombold{MC07} & Estimate To Complete (ETC) & $\geq0$ & $\leq EAC$ \\
\hline
\rowcolor{LightBlue}
\hline
\multicolumn{4}{|c|}{\custombold{Sviluppo}}\\
\hline
\rowcolor{LighterBlue}
\custombold{MC08} & Requirements Stability Index (RSI) & $\geq80\%$ & $100\%$\\
\hline

\end{tabular}
\captionof{table}{Tabella dei processi primari}
\label{tab:processiprimari}

\subsection{Processi di supporto}
\begin{tabular}{|P{5em}| P{18em}| P{8em}| P{8em}|}
\hline
\rowcolor{Blue}
 \custombold{Codice} & \custombold{Nome metrica} & \custombold{Valore} \custombold{accettabile} & \custombold{Valore ottimale}\\
 \hline
\rowcolor{LightBlue}
\multicolumn{4}{|c|}{\custombold{Verifica}}\\
\hline
\rowcolor{LighterBlue}
\custombold{MC09} & Passed Tests & $\geq80\%$ & $100\%$ \\
\hline
\rowcolor{LightBlue}
\multicolumn{4}{|c|}{\custombold{Gestione della qualità}}\\
\hline
\rowcolor{LighterBlue}
\custombold{MC10} & Metrics Satisfied & $\geq85\%$ & $100\%$ \\
\hline
\end{tabular}

\captionof{table}{Tabella dei processi di supporto}
\label{tab:processisup}

\subsection{Processi organizzativi}
\begin{tabular}{|P{5em} |P{18em}| P{8em} |P{8em}|}
\hline
\rowcolor{Blue}
 \custombold{Codice} & \custombold{Nome metrica} & \custombold{Valore} \custombold{accettabile} & \custombold{Valore ottimale}\\
 \hline
\rowcolor{LightBlue}
\multicolumn{4}{|c|}{\custombold{Gestione organizzativa}}\\
\hline
\rowcolor{LighterBlue}
\custombold{MC11} & Risks Found & $\leq5$ & $0$ \\
\hline
\end{tabular}
\captionof{table}{Tabella dei processi organizzativi}
\label{tab:processiorg}

\section{Qualità di prodotto}
Per assicurare l'elevata qualità del prodotto, è stata adottata come base di riferimento la norma ISO/IEC 12207:1997. In questa sezione vengono presentati i valori ideali e quelli accettabili relativi alle metriche scelte dal team Cyber Sorceres. Per una visione dettagliata delle metriche indicate in seguito, si prega di fare riferimento al documento \textit{Norme di progetto}.
\subsection{Obiettivi}
\begin{itemize}
    \item{Efficienza}
    \item {Usabilità}
    \item {Affidabilità}
    \item {Manutenibilità}
    \item {Portabilità}    
\end{itemize}

\begin{center}
\begin{tabular}{|P{8em}| P{20em}| P{8em}|}
\hline
  \rowcolor{Blue}
    \custombold{Obiettivo} & \custombold{Descrizione} & \custombold{Metriche}\\
    \hline
    \rowcolor{LightBlue}
    \multicolumn{3}{|c|}{Documentazione}\\
    \hline
    \rowcolor{LighterBlue}
    \textbf{Leggibilità documenti} & La documentazione deve essere comprensibile agli utenti. & \textbf{MD01}\\
    \hline
    \rowcolor{LightBlue}
    \textbf{Correttezza linguistica} & Non devono essere presenti errori grammaticali nella documentazione. & \textbf{MD02}\\
    \hline
    \rowcolor{LighterBlue}
    \multicolumn{3}{|c|}{Software}\\
    \hline
    \rowcolor{LightBlue}
    \textbf{Funzionalità} & La capacità del prodotto di fornire tutte le funzioni identificate nell'\textit{Analisi dei requisiti}, perseguendo precisione e idoneità. & \textbf{MS01}, \textbf{MS02}, \textbf{MS03}\\
    \hline
    \rowcolor{LighterBlue}
    \textbf{Usabilità} & La capacità di essere comprensibile al fine di rendere gradevole l'esperienza dell'utente. Le funzionalità devono essere in linea con le aspettative e compatibili con le stesse. & \textbf{MS04}\\
    \hline
    \rowcolor{LightBlue}
    \textbf{Portabilità} & La capacità di operare in vari contesti di esecuzione. Gli obiettivi da raggiungere includono adattabilità e sostituibilità. & \textbf{MS05}, \textbf{MS06}\\
    \hline
    \rowcolor{LighterBlue}
    \textbf{Test\textsubscript{G}} & L'intero codice prodotto sarà soggetto a verifica per assicurare l'implementazione corretta dei requisiti identificati. & \textbf{MS07}, \textbf{MS08}, \textbf{MS09}, \textbf{MS10}\\
    \hline
\end{tabular}
\captionof{table}{Tabella degli obiettivi della qualità di prodotto}
\label{tab:qualitaProd}
\end{center}

\begin{center}
\begin{tabular}{|P{5em} |P{13em} |P{10em}| P{10em}|}
\hline
  \rowcolor{Blue}
    \custombold{Codice} & \custombold{Denominazione metrica} & \custombold{Valore accettabile} & \custombold{Valore ottimale}\\
    \hline
    \rowcolor{LighterBlue}
    \custombold{MD01} & Indice di Gulpease & $\geq 60$ & $\geq 80$\\
    \hline
    \rowcolor{LightBlue}
    \custombold{MD02} & Errori ortografici & 0 & 0 \\
    \hline
    \rowcolor{LighterBlue}
    \custombold{MS01} & Copertura requisiti obbligatori & 100\% & 100\% \\
    \hline
    \rowcolor{LightBlue}
    \custombold{MS02} & Copertura requisiti desiderabili & $\geq 50$\% & $\geq 100$\% \\
    \hline
    \rowcolor{LighterBlue}
    \custombold{MS03} & Copertura requisiti opzionali & $\geq 50$\% & $\geq 100$\% \\
    \hline
    \rowcolor{LightBlue}
    \custombold{MS04} & Facilità utilizzo & 5 click & 4 click \\
    \hline
    \rowcolor{LighterBlue}
    \custombold{MS05} & Versioni browser supportate & $\geq 80$\% & $\geq 100$\% \\
    \hline
    \rowcolor{LightBlue}
    \custombold{MS06} & Versioni VSCode\textsubscript{G} supportate & $\geq 80$\% & $\geq 100$\% \\
    \hline
    \rowcolor{LighterBlue}
    \custombold{MS07} & Solidity Statement Coverage & $\geq 80$\% & $\geq 100$\% \\
    \hline
    \rowcolor{LightBlue}
    \custombold{MS08} & Solidity Branche Coverage & $\geq 80$\% & $\geq 100$\% \\
    \hline
    \rowcolor{LighterBlue}
    \custombold{MS09} & Solidity Function Coverage & $\geq 80$\% & $\geq 100$\% \\
    \hline
    \rowcolor{LightBlue}
    \custombold{MS10} & Solidity Line Coverage & $\geq 80$\% & $\geq 100$\% \\
    \hline
\end{tabular}
\captionof{table}{Metriche per la qualità di prodotto}
\label{tab:metricheQualProd}
\end{center}
\newpage


\section{Test\textsubscript{G} e specifiche}
Nella seguente sezione esporremo le varie metodologie di test, i loro obiettivi e i criteri di successo ottenuti. Per facilitare la fase di validazione e accertamento continuo della correttezza del prodotto il gruppo ha deciso di svolgere una verifica in parallelo allo sviluppo, conformandosi al "modello a V"\textsubscript{G}.  
\begin{figure}[H]
    \centering
    \includegraphics[width=15cm, height=10cm]{./documenti/grafici/modello_a_v.png}
    \caption{Figura 0: Modello a V}
\end{figure}

Ciascuna tipologia di test sarà descritta da apposite tabelle, comprensive
di identificativo, descrizione e stato, per le quali possiamo riportare le seguenti sigle:
\begin{itemize}
    \item{ S per Superato;}
    
    \item { NI per Non Implementato.}    
\end{itemize}



\subsection{Test di Unità}
Il \textit{test di unità} sono una tipologia di testing\textsubscript{G} del software in cui singole unità o componenti del software vengono testate in isolamento. Le unità possono essere singole funzioni, procedure, metodi o classi. L'obiettivo del test di unità è verificare che ciascuna unità funzioni correttamente secondo le specifiche e che produca i risultati attesi. Abbiamo deciso che questo tipo di testing sarà in larga parte automatizzato, per ottimizzare i costi e le tempistiche dedicate a questo processo.

\begin{center}
    \begin{tabular}{P{5em} P{21em} P{6em} P{6em}}
        \rowcolor{Blue}
    \custombold{ID Test\textsubscript{G}} & \custombold{Descrizione} & \custombold{Stato}\\
    \hline
    \rowcolor{LighterBlue}
    TU1 & Si verifica che il componente AddProjectButton venga renderizzato correttamente    &S \\
    \hline
    \rowcolor{LightBlue}
    TU2 &  Si verifica che il componente AddProjectButton apra una casella di inserimento testuale correttamente   &S \\
    \hline
    \rowcolor{LighterBlue}
    TU3 & Si verifica che il  pulsante di chiusura del componente AddProjectButton funzioni correttamente   &S \\
    \hline
    \rowcolor{LightBlue}
    TU4 &  Si verifica che il pulsante di invio del componente AddProjectButton funzioni correttamente  & S\\
    \hline
    \rowcolor{LighterBlue}
    TU5 & Si verifica che il componente BackButton venga renderizzato correttamente  &S \\
    \hline\rowcolor{LightBlue}
    TU6 &  Si verifica che il componente BackButton funzioni correttamente  &S \\
    \hline
    \rowcolor{LighterBlue}
    TU7 & Si verifica che il componente BuisnessRequest venga renderizzato correttamente  &S \\
    \hline\rowcolor{LightBlue}
    TU8 &  Si verifica che il componente BuisnessRequest funzioni correttamente  & S\\
    \hline
    \rowcolor{LighterBlue}
    TU9 & Si verifica che il componente DeleteEpic venga renderizzato correttamente  &S \\
    \hline\rowcolor{LightBlue}
    TU10 &  Si verifica che il componente DeleteEpic funzioni correttamente   & S\\
    \hline
    \rowcolor{LighterBlue}
    TU11 & Si verifica che il componente DeleteUser venga renderizzato correttamente   &S \\
    \hline\rowcolor{LightBlue}
     TU12 &  Si verifica che il componente DeleteUser funzioni correttamente   &S \\
    \hline
    \rowcolor{LighterBlue}
    TU13 & Si verifica che il componente DropdownButton venga renderizzato correttamente   & S\\
    \hline\rowcolor{LightBlue}
     TU14 &  Si verifica che il componente DropdownButton funzioni correttamente &S \\
    \hline
    \rowcolor{LighterBlue}
    TU15 & Si verifica che il componente DropdownMenuContainer venga renderizzato correttamente    & S\\
    \hline\rowcolor{LightBlue}
     TU16 &  Si verifica che il componente DropdownMenuContainer funzioni correttamente    &S \\
    \hline
    \rowcolor{LighterBlue}
    TU17 & Si verifica che il componente DropdownMenuContainer si chiuda correttamente   &S \\
    
    \hline
    \end{tabular}
        \end{center}
        
    \begin{center}
    \begin{tabular}{P{5em} P{21em} P{6em} P{6em}}
    \rowcolor{LightBlue}
    TU18 &  Si verifica che gli item del componente DropdownMenuContainer funzionino correttamente  & S\\
    \hline\rowcolor{LighterBlue}
     TU19 &  Si verifica che il componente DropdownMenu venga renderizzato correttamente   & S\\
    \hline
    \rowcolor{LightBlue}
    TU20 & Si verifica che il componente DropdownMenu venga funzioni correttamente &S \\
    \hline\rowcolor{LighterBlue}
    TU21 &  Si verifica che il componente DropdownMenu si chiuda correttamente   &S \\
    \hline\rowcolor{LightBlue}
     TU22 & Si verifica che gli item del componente DropdownMenu funzionino correttamente  & S\\
    \hline
    \rowcolor{LighterBlue}
    TU23 & Si verifica che il componente EpicStory venga renderizzato correttamente & S\\
    
     \hline\rowcolor{LightBlue}
     TU24 & Si verifica che nel componente EpicStory venga renderizzato il titolo della tabella correttamente    &S \\
    \hline
    
    \rowcolor{LighterBlue}
    TU25 & Si verifica che nel componente EpicStory venga renderizzata la descrizione della tabella correttamente   &S \\
    \hline\rowcolor{LightBlue}
     TU26 & Si verifica che nel componente EpicStory venga renderizzato il bottone per aggiungere un'epic story correttamente   &S \\
    \hline
    \rowcolor{LighterBlue}
    TU27 &  Si verifica che il componente InviteUserButton funzioni correttamente & S  \\
    \hline\rowcolor{LightBlue}
     TU28 & Si verifica che il componente Login venga renderizzato correttamente    &S \\
    \hline
    \rowcolor{LighterBlue}
    TU29&Si verifica che il componente Login funzioni correttamente   &S \\
    \hline\rowcolor{LightBlue}
     TU30 & Si verifica che il componente NavigationBar venga renderizzato correttamente     &S \\
    \hline
    \rowcolor{LighterBlue}
    TU31&Si verifica che il componente NavigationBar funzioni correttamente   &S \\
    \hline\rowcolor{LightBlue}
     TU32 &  Si verifica che il componente NotificationPage venga renderizzato correttamente    &S \\
    \hline
    \rowcolor{LighterBlue}
    TU33& Si verifica che il componente NotificationPage renderizzi il corretto numero di righe   & S\\
    
    \hline
     \end{tabular}
        \end{center}
    \begin{center}
    \begin{tabular}{P{5em} P{21em} P{6em} P{6em}}
    \rowcolor{LightBlue}
     TU34 & Si verifica che il componente Password venga renderizzato correttamente    & \\
    \hline
    \rowcolor{LighterBlue}
    TU35& Si verifica che il toggle della visibilità nel componente Password funzioni correttamente & S\\
    \hline\rowcolor{LightBlue}
     TU36 & Si verifica che il componente Password funzioni correttamente    &S \\
    \hline
    \rowcolor{LighterBlue}
    TU36&Si verifica che il componente PopupFeedback venga renderizzato correttamente & S\\
    \hline\rowcolor{LightBlue}
     TU37 & Si verifica che il componente PopupFeedback funzioni correttamente  & S  \\
    \hline
    \rowcolor{LighterBlue}
    TU38&Si verifica che il componente Registrazione venga renderizzato correttamente  & S\\
    \hline\rowcolor{LightBlue}
     TU39 & Si verifica che il pulsante submit nel componente Registrazione funzioni correttamente    &S \\
    \hline
    \rowcolor{LighterBlue}
    TU40&  Si verifica che il campo di testo email nel componente Registrazione funzioni correttamente& S\\
    \hline\rowcolor{LightBlue}
     TU41 & Si verifica che il campo di testo password nel componente Registrazione funzioni correttamente  &S \\
    \hline
    \rowcolor{LighterBlue}
    TU42& Si verifica che il componente RejectProject funzioni correttamente &S \\
    \hline\rowcolor{LightBlue}
     TU43 & Si verifica che il componente RequireAuth venga renderizzato correttamente    & S\\
    \hline
    \rowcolor{LighterBlue}
    TU44& Si verifica che il componente RequireAuth funzioni correttamente  & S\\
    \hline\rowcolor{LightBlue}
     TU45 & Si verifica che il componente Step1 venga renderizzato correttamente  &S  \\
    \hline
    \rowcolor{LighterBlue}
    TU46& Si verifica che il bottone submit del componente Step1 funzioni correttamente  & S\\
    \hline\rowcolor{LightBlue}
     TU47 & Si verifica che il campo di testo email del componente Step1 funzioni correttamente     & S\\
    \hline
    \rowcolor{LighterBlue}
    TU48& Si verifica che il componente Step2 venga renderizzato correttamente   & S\\
    \hline\rowcolor{LightBlue}
     TU49 & Si verifica che il bottone submit del componente Step2 funzioni correttamente   &S \\
    \hline
    \rowcolor{LighterBlue}
    TU50&  Si verifica che il campo di testo del componente Step2 funzioni correttamente & S\\
     \end{tabular}
        \end{center}
    \begin{center}
    \begin{tabular}{P{5em} P{21em} P{6em} P{6em}}
     \hline\rowcolor{LightBlue}
     TU51 & Si verifica che il componente Table venga renderizzato correttamente  &S   \\
    \hline
    \rowcolor{LighterBlue}
    TU52& Si verifica che il componente Table renderizzi una riga per ogni elemento   &S \\
     \hline\rowcolor{LightBlue}
     TU53 &   Si verifica che il componente Table renderizzi un bottone correttamente  &S \\
     \hline
    \rowcolor{LighterBlue}
    TU54& Si verifica che il bottone del componente Table funzioni correttamente    & \\
    \hline\rowcolor{LightBlue}
     TU55 &   Si verifica che addEpicStory aggiunga le epic story  correttamente   &S \\
     \hline
    \rowcolor{LighterBlue}
    TU56& Si verifica che getProgetti ritorni tutti i progetti correttamente   & S\\
     \hline\rowcolor{LightBlue}
     TU57 &   Si verifica che le notifiche funzionino correttamente  &S \\
     \hline
    \rowcolor{LighterBlue}
    TU58&  Si verifica che getEpicStory ritorni la epic story corretta     &S \\
    \hline\rowcolor{LightBlue}
     TU59 &   Si verifica che getUserStory ritorni la user story corretta    & S\\
     \hline
    \rowcolor{LighterBlue}
    TU60&  Si verifica che getProgetto ritorni il progetto corretto   & S\\
    \hline\rowcolor{LightBlue}
     TU61 &   Si verifica che getProgetto ritorni l'errore giusto quando non si è effettuato il login   & S\\
     \hline
    \rowcolor{LighterBlue}
    TU62&  Si verifica che invite ritorni l'errore giusto quando non si è effettuato il login come Project Manager  & S\\
     \hline\rowcolor{LightBlue}
     TU63 &    Si verifica che invite ritorni l'errore giusto se il corpo dell'invito non è corretto  &S \\
    \end{tabular}
    \captionof{table}{Tabella dei test di unità}
\end{center}


\subsection{Test di Integrazione}
I \textit{test di integrazione} sono una fase del processo di testing\textsubscript{G} in cui le diverse unità o del software vengono combinate e testate insieme come gruppo. L'obiettivo principale è verificare che le singole unità, testate precedentemente in modo isolato tramite i test di unità, funzionino correttamente quando integrate e collegate tra loro. Durante i test di integrazione, vengono identificati e risolti eventuali problemi di interfacciamento tra le diverse unità e vengono verificate le interazioni tra di esse. L'obiettivo finale è garantire che l'intero sistema funzioni come previsto e che tutte le interazioni tra le sue parti siano corrette.

\begin{center}
    \begin{tabular}{P{5em} P{21em} P{6em} P{6em}}
        \rowcolor{Blue}
    \custombold{ID Test\textsubscript{G}} & \custombold{Descrizione}  & \custombold{Stato}\\
    \hline
    \rowcolor{LightBlue}
    TI1 & Verifica che i dati dei progetti vengano recuperati correttamente dal database& S\\
    \hline
    \rowcolor{LighterBlue}
    TI2 &Verifica che i dati delle epic stories vengano recuperati correttamente dal database & S\\
    \hline
    \rowcolor{LightBlue}
    TI3 & Verifica che i dati delle user stories vengano recuperati corretttamente dal database& S\\
   \hline
    \rowcolor{LighterBlue}
    TI4 &Verifica che il project manager possa aggiungere progetti correttamente & S\\
    \hline
    \rowcolor{LightBlue}
    TI5 & Verifica che l'itelligenza artificiale selezionata generi correttamente le epic user sotries a partire dalle richieste di buisness& S\\
    \hline
    \rowcolor{LighterBlue}
    TI6 &Verifica che l'itelligenza artificiale selezionata generi correttamente i test a partire dal codice fornito& S\\
    \hline
    \rowcolor{LightBlue}
    TI7&  Verifica che iclineti possano inviare correttamente le richieste di buisness & S\\
    \hline
    \rowcolor{LighterBlue}
    TI8&Verifica che si possano  eliminare le epic stories e tutte le loro user stories correttamente& S\\
    \hline
    \rowcolor{LightBlue}
    TI9&  Verifica che i clienti possano inviare correttamente le richieste di buisness & S\\
    \end{tabular}
\end{center}
\subsection{Test di Sistema}
I \textit{test di sistema} sono una fase del processo che si concentra sull'analisi e la verifica del sistema nel suo complesso rispetto ai requisiti specificati. L'obiettivo principale è garantire che il sistema soddisfi tutte le funzionalità e i requisiti richiesti dal cliente o specificati nel documento di specifica dei requisiti. Durante i test di sistema, vengono eseguiti scenari di test realistici per simulare l'utilizzo del software in un ambiente di produzione. I risultati dei test di sistema sono utilizzati per valutare se il sistema è pronto per il rilascio o se sono necessari ulteriori miglioramenti e correzioni.

\begin{center}
    \begin{tabular}{P{5em} P{21em} P{6em} P{6em}}
    \rowcolor{Blue}
    \custombold{ID Test\textsubscript{G}} & \custombold{Descrizione} &  \custombold{Stato}\\
    \hline
    \rowcolor{LighterBlue}
    TS1 & Verificare che sia possibile accedere alla web-app\textsubscript{G} tramite email e password  & NI\\
    \hline
    \rowcolor{LightBlue}
    TS2 & Verificare che sia possibile scrivere le richieste di business  & NI\\
    \hline
    \rowcolor{LighterBlue}
    TS3 & Verificare che sia possibile inviare le richieste di business all'IA\textsubscript{G}  & NI\\
    \hline
    \rowcolor{LightBlue}
    TS4 & Verificare che sia possibile visualizzare l'andamento del progetto dalla web-app\textsubscript{G} & NI\\
    \hline
    \rowcolor{LighterBlue}
    TS5 & Verificare che sia possibile per l'attore\textsubscript{G} Cliente inviare un feedback riguardo l'implementazione di una user story\textsubscript{G} generata & NI\\
    \hline
    \rowcolor{LightBlue}
    TS6 & Verificare che si riceva una notifica una volta che una user story\textsubscript{G} è contrassegnata come completata  & NI\\
    \hline
    \rowcolor{LighterBlue}
    TS7 & Verificare che sia possibile il tag\textsubscript{G} del codice utilizzando il plug-in\textsubscript{G} & NI\\
    \hline
    \rowcolor{LightBlue}
    TS8 & Verificare che sia possibile per l'attore\textsubscript{G} Sviluppatore il poter vedere la lista di user/epic stories\textsubscript{G} che gli sono state assegnate  & NI\\
    \hline
    \rowcolor{LighterBlue}
    TS9 & Verificare che quando viene assegnata una epic/user story\textsubscript{G} l'attore Sviluppatore venga notificato tramite la web-app\textsubscript{G}  & NI\\
    \hline
    \rowcolor{LightBlue}
    TS10 & Verificare che sia possibile inviare il codice all'IA\textsubscript{G} per la verifica  & NI\\
    \hline
    \end{tabular}
\end{center}
\begin{center}
    \begin{tabular}{P{5em} P{21em} P{6em} P{6em}}
    \rowcolor{Blue}
    \custombold{ID Test\textsubscript{G}} & \custombold{Descrizione} & \custombold{Stato}\\
    \hline
    \rowcolor{LighterBlue}
    TS11 & Verificare che sia possibile visualizzare le epic/user stories\textsubscript{G} generate  & NI\\
    \hline
    \rowcolor{LightBlue}
    TS12 & Verificare che sia possibile per l'attore Project Manager\textsubscript{G} l'invio di feedback riguardante le epic/user stories generate dall'IA\textsubscript{G} & NI\\
    \hline
    \rowcolor{LighterBlue}
    TS13 & Verificare che sia possibile per l'attore Project Manager richiedere la divisione delle user stories valutate come troppo grandi  & NI\\
    \hline
    \rowcolor{LightBlue}
    TS14 & Verificare che sia possibile per l'attore Project Manager assegnare le epic/user stories a gli attori Sviluppatore  & NI\\
    \hline
    \rowcolor{LighterBlue}
    TS15 & Verificare che venga inviata una notifica all'attore Project Manager quando vengono generate le epic/user stories  & NI\\
    \hline
    \rowcolor{LightBlue}
    TS16 & Verificare che sia possibile per l'attore Project Manager richiedere la modifica di epic/user stories all'IA  & NI\\
    \hline
    \rowcolor{LighterBlue}
    TS17 & Verificare che sia possibile per l'attore Sviluppatore visualizzare l'andamento delle epic/user stories assegnate & NI\\
    \hline
    \rowcolor{LightBlue}
    TS18 & Verificare che il plug-in\textsubscript{G} supporti i linguaggi Typescript\textsubscript{G} e Javascript\textsubscript{G}  & NI\\
    \hline
    \rowcolor{LighterBlue}
    TS19 & Verificare che il plug-in supporti i linguaggi Kotlin\textsubscript{G} e Swift\textsubscript{G}  & NI\\
    \hline
    \end{tabular}
    \captionof{table}{Tabella dei test di sistema}
\label{tab:testSistema}
\end{center}

\subsection{Test di Accettazione}
I \textit{test di accettazione} sono una fase finale in cui il sistema viene valutato dal cliente per determinare se soddisfa i requisiti concordati e se è pronto per il rilascio. Questi test sono orientati a verificare che il sistema sia conforme alle aspettative e alle necessità degli utenti e che sia in grado di svolgere le funzioni previste in modo efficace ed efficiente. L'obiettivo principale è confermare che il software sia pronto per essere messo in produzione e che risponda alle aspettative del cliente. I risultati dei test di accettazione sono fondamentali per prendere decisioni riguardanti il rilascio del prodotto e possono influenzare eventuali modifiche o miglioramenti futuri.
\begin{center}
    \begin{tabular}{P{5em} P{21em} P{6em} P{6em}}
    \rowcolor{Blue}
    \custombold{ID Test\textsubscript{G}} & \custombold{Descrizione}  & \custombold{Stato}\\
    \hline
    \rowcolor{LighterBlue}
    TA1 & Si verifica che la web-app\textsubscript{G} sia in grado di gestire il recupero password, inviando opportunamente una mail di
reimpostazione all’utente qualora abbia perso le proprie credenziali  & S\\
    \hline
    \rowcolor{LightBlue}
    TA2 & SI verifica che il Project Manager\textsubscript{G} possa aggiungere nuovi clienti correttamente   & S\\
    \hline
    \rowcolor{LighterBlue}
    TA3 & Si verifica che la web-app\textsubscript{G} sia in grado di gestire correttamente gli eventi di autenticazione dei propri tipi di utenti, inclusi login e logout  & S\\
    \hline
    \rowcolor{LightBlue}
    TA4 & Si verifica che l'applicazione permetta di visualizzare l'andamento dei progetti in base allo sviluppo di epic/user stories    & S\\
    \hline
    \rowcolor{LighterBlue}
    TA5 & Si verifica che il Project Manger sia in grado di invitare nuovi clienti fornendo password provvisorie   & S\\
    \hline
    \rowcolor{LightBlue}
    TA16 & Si verifica che il Project Manager possa creare nuovi progetti    &S\\
    \hline
    \rowcolor{LighterBlue}
    TA7 & Si verifica che il cliente possa inviare richieste di buisness   & S\\
    \hline
    \rowcolor{LightBlue}
    TA8 & Si verifica che il cliente possa richiedere modifiche  alle user stories sviluppate & S\\
    \hline
    \rowcolor{LighterBlue}
    TA9 & Si verifica che il Project Manager possa rifiutare una proposta di progetto   & S\\
    \hline
    \rowcolor{LightBlue}
    TA10 & Si verifica che il  Project Manager possa eliminare epic stories  & S\\
    \hline
    \rowcolor{LighterBlue}
    TA11 & Si verifica che il sistema di notifiche sia funzionante   & S\\
    \hline
    \end{tabular}
    \captionof{table}{Tabella dei test di accettazione}
\label{tab:testSistema}
\end{center}
\subsection{Test di Regressione}
I \textit{test di regressione} mirano a verificare che le modifiche apportate al codice sorgente o al sistema non abbiano introdotto nuovi difetti o rotto funzionalità esistenti. Questi test vengono eseguiti dopo ogni modifica al software, come aggiornamenti, correzioni di bug o nuove implementazioni. L'obiettivo è assicurarsi che le modifiche non abbiano impatti indesiderati sul comportamento del sistema, specialmente su funzionalità precedentemente testate e funzionanti correttamente. L'obiettivo del gruppo è raggiungere la massima automazione possibile dei test di regressione, al fine di ridurre i tempi di esecuzione e garantire una copertura completa dei test.

\begin{center}
    \begin{tabular}{P{5em} P{21em} P{6em} P{6em}}
        \rowcolor{Blue}
    \custombold{ID Test\textsubscript{G}} & \custombold{Descrizione}  & \custombold{Stato}\\
    \hline
    \rowcolor{LighterBlue}
    TR1 &Verifica che l'applicazione sia compatibile con Chrome a partire dalla versione 123.0 & S\\
    \hline
    \rowcolor{LightBlue}
    TR2 & Verifica che l'applicazione sia compatibile con Firefox a partire dalla versione 124.0& S\\
   \hline
    \rowcolor{LighterBlue}
    TR3 &Verifica che l'applicazione sia compatibile con Safari a partire dalla versione 17.0 & S\\
    \hline
    \rowcolor{LightBlue}
    TR4 & Verifica che i dati dei progetti vengano recuperati correttamente dal database& S\\
    \hline
    \rowcolor{LighterBlue}
    TR5 &Verifica che i dati delle epic stories vengano recuperati correttamente dal database & S\\
    \hline
    \rowcolor{LightBlue}
    TR6 & Verifica che i dati delle user stories vengano recuperati corretttamente dal database& S\\
   \hline
    \rowcolor{LighterBlue}
    TR7 &Verifica che il project manager possa aggiungere progetti correttamente & S\\
    \hline
    \rowcolor{LightBlue}
    TR8 & Verifica che l'itelligenza artificiale selezionata generi correttamente le epic user sotries a partire dalle richieste di buisness& S\\
    \hline
    \rowcolor{LighterBlue}
    TR9 &Verifica che l'itelligenza artificiale selezionata generi correttamente i test a partire dal codice fornito& S\\
    \hline
    \rowcolor{LightBlue}
    TR10&  Verifica che iclineti possano inviare correttamente le richieste di buisness & S\\
    \hline
    \rowcolor{LighterBlue}
    TR11&Verifica che si possano  eliminare le epic stories e tutte le loro user stories correttamente& S\\
    \hline
    \rowcolor{LightBlue}
    TR12&  Verifica che i clienti possano inviare correttamente le richieste di buisness & S\\
    \end{tabular}
\end{center}
\section{Resoconto attività di verifica}
\subsection{Verifica documenti}
\subsubsection{Indice di Gulpease}

    \begin{figure}[H]
    \centering
    \includegraphics[width=\textwidth]{documenti/grafici/IndiceDiGulpeasePB.png}
     \caption{Figura 1: indice di Gulpease per periodo}
    \end{figure}

    L'indice di Gulpease valuta la leggibilità dei documenti prodotti durante il progetto. Un punto di forza è il miglioramento della leggibilità nel tempo, come indicato dall'aumento dell'indice. Tuttavia, la diminuzione dell'indice durante la fase di Codifica PoC evidenzia una potenziale complessità aggiuntiva nei documenti tecnici, che potrebbe richiedere ulteriori sforzi per semplificare e migliorare la comprensione.



\subsubsection{Errori ortografici}
   \begin{figure}[H]
    \centering
    \includegraphics[width=\textwidth]{documenti/grafici/ErroriOrtograficiPB.png}
     \caption{Figura 2: errori ortografici per periodo}
    \end{figure}
     Il grafico evidenzia il numero di errori ortografici rilevati nei documenti del progetto attraverso le diverse fasi. Il calo costante degli errori rappresenta un miglioramento significativo nella qualità della documentazione, un chiaro punto di forza. Tuttavia, l'alto numero iniziale di errori ortografici durante l'Analisi e la Progettazione TB solleva preoccupazioni riguardo la qualità iniziale della revisione e la necessità di maggiore attenzione nei primi stadi del progetto.
\subsection{Verifica processi}
\subsubsection{Stima al completamento}
\begin{figure}[H]
    \centering
    \includegraphics{documenti/grafici/CostoEffettivoEStimaAlCompletamentoPB.png}
    \caption{Figura 3: Revisione del valore stimato per la realizzazione del progetto}
    \end{figure}
    Questo grafico mostra il budget totale stanziato, il budget stimato per completare le attività rimanenti e i soldi spesi fino al momento del calcolo per ogni fase del progetto. Un punto di forza evidente è il controllo accurato delle spese rispetto al budget totale. Tuttavia, si nota un problema durante la fase di Codifica PoC dove il budget stimato per completare le attività rimanenti è significativamente inferiore al budget stanziato, suggerendo possibili sottostime o risparmi imprevisti.
\subsubsection{Valore guadagnato \& valore previsto}
\begin{figure}[H]
    \centering
    \includegraphics[width=\textwidth]{documenti/grafici/ValoreGuadagnatoEPrevistoPB1.png}
    \caption{Figura 4: Valore delle attività realizzate e costo pianificato per realizzare le rimanenti}
    \end{figure}
    Questo grafico confronta il valore guadagnato (Serie1) con il valore previsto (Serie2) per ciascuna fase del progetto. La corrispondenza quasi perfetta tra i due valori è un punto di forza, dimostrando una pianificazione accurata e una realizzazione conforme. Tuttavia, eventuali discrepanze minime possono indicare aree dove l'efficienza potrebbe essere ulteriormente migliorata.
\subsubsection{Costo e stima al completamento}
    
    \begin{figure}[H]
    \centering
    \includegraphics{documenti/grafici/CostoEStimaAlCompletamentoPB.png}
    \caption{Figura 5: Costo effettivamente sostenuto e valore stimato per la realizzazione delle rimanenti attività}
\end{figure}
Questo grafico rappresenta il budget totale (Budget at completion), il budget stanziato (EAC) e i valori superiori e inferiori di riferimento per ogni fase del progetto. La stabilità dei valori totali e la vicinanza delle stime ai valori effettivi sono punti di forza, indicando una buona gestione del budget. Tuttavia, il leggero aumento nel periodo di Progettazione TB indica una possibile sottovalutazione iniziale dei costi in quella fase.

\subsubsection{Variazione programmazione \& variazione costi}
\begin{figure}[H]
    \centering
    \includegraphics[width=\textwidth]{documenti/grafici/VariazioneProgrammazioneECostiPB.png}
    \caption{Figura 6: Variazione programmazione e variazione costo per periodo}
    \end{figure}
    Il grafico mostra la variazione della programmazione (Schedule Variance) e dei costi (Cost Variance) per ogni fase del progetto. I punti di forza includono la capacità di mantenere le variazioni di costi relativamente contenute. Tuttavia, le significative variazioni negative nella programmazione durante la fase di Progettazione TB e Codifica PoC suggeriscono problemi di scheduling che potrebbero derivare da una pianificazione iniziale troppo ottimistica o da imprevisti non considerati adeguatamente.
\subsubsection{Stabilità indice dei requisiti}
\begin{figure}[H]
    \centering
    \includegraphics{documenti/grafici/IndiceStabilitàRequisitiRSIPB.png}
    \caption{Figura 7: Variazione del numero di requisiti}
    \end{figure}
     Il grafico mostra l'indice di stabilità dei requisiti (RSI) durante le diverse fasi del progetto. Un punto di forza è l'aumento costante dell'indice, che indica una crescente stabilità e chiarezza dei requisiti nel tempo. Tuttavia, l'iniziale basso valore di RSI durante la fase di Analisi e Progettazione TB suggerisce che i requisiti iniziali erano vaghi o soggetti a cambiamenti, evidenziando la necessità di un'analisi più approfondita all'inizio del progetto.
\subsubsection{Attualizzazione rischi}
\begin{figure}[H]
    \centering
    \includegraphics{documenti/grafici/AttualizzazioneDeiRischiPB1.png}
    \caption{Figura 8: Rischi verificati per periodo}
    \end{figure}
    Questo grafico rappresenta l'aggiornamento dei rischi durante le varie fasi del progetto. Un punto di forza evidente è il picco nella fase di Progettazione TB, dove i rischi sono stati identificati e gestiti in modo proattivo. La gestione continua dei rischi è dimostrata dall'andamento fluttuante, suggerendo una risposta attiva ai nuovi rischi man mano che emergono.
\subsubsection{Metriche di qualità soddisfatte}
\begin{figure}[H]
    \centering
    \includegraphics{documenti/grafici/MetricheDiQualitàSoddisfattePB1.png}
    \caption{Figura 9: Metriche soddisfatte per periodo}
    \end{figure}
    Questo grafico rappresenta il numero di metriche di qualità soddisfatte durante le varie fasi del progetto. Un punto di forza è il miglioramento costante e significativo nella soddisfazione delle metriche di qualità, specialmente durante le fasi di Sviluppo Requisiti Obbligatori e Opzionali. Tuttavia, il numero di metriche soddisfatte rimane stabile e basso nelle fasi iniziali, indicando una mancanza di attenzione alla qualità sin dall'inizio, che ha richiesto sforzi correttivi nelle fasi successive.

\section{Valutazioni per il miglioramento}
In questo paragrafo, esamineremo le sfide incontrate fino alla consegna del progetto e valuteremo le soluzioni adottate dal gruppo. Nella tabella allegata, nella colonna delle soluzioni, sono elencate le strategie che abbiamo identificato per affrontare le sfide riscontrate. È importante sottolineare che il processo di miglioramento è continuo e dinamico, seguendo il principio del ciclo PDCA\textsubscript{G}. Ciò significa che le migliorie vengono implementate in corso d'opera non appena vengono individuate criticità, consentendo al team di adattarsi e migliorare costantemente durante lo sviluppo del progetto.

\subsection{Valutazione sull'organizzazione}
\begin{center}
\begin{tabular}{|P{10em} |P{13em}| P{4em}| P{13em}|}
\hline
    \rowcolor{Blue}
    \custombold{Criticità} & \custombold{Descrizione} & \custombold{Gravità} &
    \custombold{Soluzione}\\
    \hline
    \rowcolor{LighterBlue}
     Iniziale carenza di comunicazione con il cliente & Durante le prime fasi di sviluppo abbiamo avuto difficoltà ad ottenere le credenziali per utilizzare gli strumenti da loro richiesti. & Bassa & Abbiamo scelto di focalizzare il lavoro, durante l'attesa, nella redazione dei documenti e aprire un canale di comunicazione più veloce delle mail. Si è scelto Slack per la comunicazione esterna, più rapido e molto utilizzato dall''azienda. è fondamentale sviluppare da parte del team una rapida capacità di reazione.\\ 
     \hline
    \rowcolor{LightBlue}
     Disparità di impegno tra i membri& Alcuni membri, avendo più impegni accademici o lavorativi, sono stati meno presenti agli incontri o per la realizzazione del progetto. & Media & Ci siamo imegnati nell'autostimare concretamente il tempo che si ha disponibile, comunicando in modo trasparente i nostri impegni settimanali (creando un calendario condiviso su Google Drive) agli altri memebri, e assegnare i compiti quanto più in maniera equa e realistica. \\ 
     \hline
\end{tabular}
\captionof{table}{Criticità sull'organizzazione}
\label{tab:org}
\end{center}

\subsection{Valutazione sugli strumenti utilizzati}
\begin{center}
\begin{tabular}{|P{10em}| P{13em}| P{4em}| P{13em}|}
\hline
    \rowcolor{Blue}
    \custombold{Criticità} & \custombold{Descrizione} & \custombold{Gravità} &
    \custombold{Soluzione}\\
    \hline
    \rowcolor{LighterBlue}
     Complessità nell'integrazione del plug-in\textsubscript{G} & Non avendo mai sviluppato un plug-in è stata difficoltosa la fase di integrazione. & Bassa & Abbiamo scelto di focalizzarci sull'autoapprendimento di tale tecnologia e aggiungere uno Sviluppatore a discapito di altri ruoli più marginali in quella fase. \\ 
     \hline
    \rowcolor{LightBlue}
     Repository\textsubscript{G} & Difficoltà nel mantenimento dell'ordine, della linea temporale e della versioni\textsubscript{G} dei documenti. & Media & Abbiamo focalizzato una delle fasi di verifica del Verificatore proprio sul controllo della Repository e di fare delle sedute di formazione interna per chi avesse difficoltà nell'uso delle funzionalità più utilizzate dello strumento. \\ 
     \hline
     \rowcolor{LighterBlue}
     Amazon AWS\textsubscript{G}  & Le librerie di Amazon AWS oltre ad essere moltissime, hanno tutte un prezzo diverso. & Alta & Abbiamo fissato un incontro di formazione da parte del proponente per scegliere in maniera mirata le librerie, in modo tale da essere conformi alle esigenze di costo e non perderci nella fase di analisi, studio dello strumento e scelta delle librerie. Ciò ha ottimizzato a livello di tempistica potenzialmente molto critica per un team inesperto. \\
     \hline
\end{tabular}
\captionof{table}{Criticità negli strumenti utilizzati}
\label{tab:strum}
\end{center}

\subsection{Valutazione sui ruoli}
\begin{center}
\begin{tabular}{|P{10em}| P{13em}| P{4em}| P{13em}|}
\hline
    \rowcolor{Blue}
    \custombold{Criticità} & \custombold{Descrizione} & \custombold{Gravità} &
    \custombold{Soluzione}\\
    \hline
    \rowcolor{LighterBlue}
    Verifica superficiale da parte del verificatore.& Alcuni errori sono sfuggiti durante la fase di verifica a causa di una valutazione superficiale. & Media  & Abbiamo implementato checklist di verifica più dettagliate e stringenti.\\ 
    \hline
    \rowcolor{LightBlue}
     Sviluppatori non allineati agli standard di codifica. & Non sempre il codice è stato conforme agli standard richiesti.&  Media &  Abbiamo svolto sessioni di formazione sui codici di stile e revisione del codice condiviso. Abbiamo condiviso tra membri del team materiale presente sul web fruibile e adatto allo studio individuale. \\ 
     \hline
     \rowcolor{LighterBlue}
     Inesperienza dell'analista. & Gli analisti, non avendo mai lavorato a un progetto di tale portata, hanno fatto fatica inizialmente ad individuare tutti i requisiti necessari dalle prime sedute. & Media & Abbiamo svolto sessioni  di brainstorming interne e con i proponenti, oltre che col Prof Cardin. \\
     \hline
    \rowcolor{LightBlue}
    Pianificazione poco realistica da parte del Responsabile. & Data l'inesperienza nell'ambito, la pianificazione e le aspettative sul carico di lavoro non sono state conformi alla realtà.& Bassa & Abbiamo definito sprint più frequenti della media, di due settimane, per mantenere sempre alta la produttività e favorire l'interazione tra i membri del team.\\
    \hline     
\end{tabular}
\captionof{table}{Criticità dei ruoli}
\label{tab:ruoli}

\end{center}

Questo approccio consente una valutazione delle sfide affrontate e delle relative soluzioni, fornendo una base più solida per il miglioramento continuo durante lo sviluppo del progetto.


\end{document}