\documentclass{article}
\usepackage[utf8]{inputenc}
\usepackage[includeheadfoot, margin=1em,headheight=2em]{geometry}
\usepackage{titling}
\geometry{a4paper, left=2cm, right=2cm, top=2cm, bottom=2cm}
\usepackage{graphicx}
\usepackage{enumitem}
\usepackage{array}
\newcolumntype{P}[1]{>{\centering\arraybackslash}p{#1}}
\renewcommand{\arraystretch}{1.5} % Default value: 1
\setlength{\droptitle}{-6em}

%font
\usepackage[defaultfam,tabular,lining]{montserrat}
\usepackage[T1]{fontenc}
\renewcommand*\oldstylenums[1]{{\fontfamily{Montserrat-TOsF}\selectfont #1}}

%custom bold 
\usepackage[outline]{contour}
\usepackage{xcolor}
\newcommand{\custombold}{\contour{black}}

%table colors
\usepackage{color, colortbl}
\definecolor{Blue}{rgb}{0.51,0.68,0.79}
\definecolor{LightBlue}{rgb}{0.82,0.87,0.90}
\definecolor{LighterBlue}{rgb}{0.93,0.95,0.96}

%Header
\usepackage{fancyhdr, xcolor}
\pagestyle{fancy}
\let\oldheadrule\headrule% Copy \headrule into \oldheadrule
\renewcommand{\headrule}{\color{Blue}\oldheadrule}% Add colour to \headrule
\renewcommand{\headrulewidth}{0.2em}
\fancyhead[L]{Piano di qualifica}
\fancyhead[C]{}
\fancyhead[R]{}

\title{\Huge{\textbf{Piano di Qualifica}}\vspace{-1em}}
\date{}
\begin{document}
\maketitle
\vspace{-3em}
\begin{figure}[h]
  \centering
  \includegraphics[width=6cm, height=6cm]{logo rotondo.png}
  \label{fig:immagine}
\end{figure}

\vspace{6em}
\large{

\begin{center}
    \begin{tabular}{l c c}
        \rowcolor{Blue} 
        Informazioni sul documento & &\\ [1 ex]
        \rowcolor{LighterBlue}
        Redattori: &  & \\ [1 ex]
        \rowcolor{LightBlue}
        Verificatore: &   & \\ [1 ex]
        \rowcolor{LighterBlue}
        Destinatari: & Prf. Tullio Vardanega & Prf. Riccardo Cardin\\ [1 ex]


    \end{tabular}
\end{center}}
\newpage
Registro dei Cambiamenti - Changelog

\begin{center}
\begin{tabular}{P{4em} P{6em} P{8em} P{8em} P{10em}} 
  \rowcolor{Blue}
    \custombold{Versione} & \custombold{Data} & \custombold{Autore} &
    \custombold{ Verificatore} & \custombold{Dettaglio}\\
    \rowcolor{LighterBlue}
     0.0.1& 02/12/2023 & Sabrina Caniato & Samuele Vignotto& Definizione struttura del documento e scheletro delle sezioni. Scrittura introduzione ed obiettivi delle diverse sezioni.\\ 
     \rowcolor{LightBlue}
      0.2.0& 14/12/2023 & Giulia Dentone & Samuele Vignotto& Aggiunta dei riferimenti.\\ 
     0.2.1& 30/12/2023 & Samuele Vignotto  & Sabrina Caniato & Aggiunta sezione "Qualità di prodotto"\\ 
\end{tabular}
\end{center}
\newpage
\section{Introduzione e scopo}
Il Piano di Qualifica è un documento che ci prefissiamo di aggiornare periodicamente dato che definisce l'approccio e le strategie per garantire la qualità di un progetto software. Questo piano è parte integrante del processo di gestione della qualità e fornisce una roadmap dettagliata su come il controllo e l'assicurazione della qualità verranno implementati durante l'intero ciclo di vita del progetto.\\
In Questo documento cercheremo di definire delle metriche di misurazione dell'efficacia e dell'efficenza del progetto, in base anche agli accorgimenti fornitici dal proponente.\\

Il piano di qualifica conterrà:

\begin{itemize}
    \item Definizione chiara degli obiettivi e delle metriche di qualità che il progetto propone di raggiungere.
        
    \item Specifica dei criteri che determineranno se il prodotto soddisfa gli standard di qualità stabiliti.
    
    \item Descrizione dettagliata dei processi di test che saranno implementati e la definizione delle strategie utilizzate per l'esecuzione di essi.

    \item Procedure per gestire eventuali deviazioni rispetto agli standard di qualità pianificati.

\end{itemize}

\subsection{Glossario}
All'interno del documento saranno presenti termini specifici, indicati con una G come apice. Al fine di garantirne la comprensione abbiamo stilato un documento nominato "Glossario" in cui si potrà trovare una breve spiegazione del termine. 

\subsection{Riferimenti}
\textbf{Riderimenti normativi}
\begin{itemize}
    \item Capitolato C7 - 	ChatGPT vs BedRock developer analysis: https://www.math.unipd.it/~tullio/IS-1/2023/Progetto/Capitolati.html
\end{itemize}
\textbf{Riferimenti informativi}
\begin{itemize}
    \item Argomento T7 - Qualità del software: https://www.math.unipd.it/~tullio/IS-1/2023/Dispense/T7.pdf
    \item Argomento T8 - Qualità di processo:https://www.math.unipd.it/~tullio/IS-1/2023/Dispense/T8.pdf
    \item Argomento T9 - Verifica e validazione: https://www.math.unipd.it/~tullio/IS-1/2023/Dispense/T9.pdf
    \item - ISOG/IECG 9126:2001 SWE Product Quality;
    \item ISO/IEC 14598:1999 SW Product Evaluation;
    \item  ISO/IEC 25000:2005 SQuaRE: Systems and software Quality Requirements and Evaluation:
        \begin{itemize}
        \item 25010:2011 Quality model;
        \item 25020:2019 Quality measurement framework;
        \item 25030:2007 Quality requirements;
        \item 25040:2011 Quality evaluation.
    \end{itemize}
    \item ISO 9000:2015;
    \item ISO 9004:2018;
    \item ISO/IEC 33020:2019.
\end{itemize}

\section{Qualità di processo}
\subsection{Scopo ed obiettivi}
La qualità è determinata univocamente dai processi che compongono un prodotto, misurata attraverso che permettano di valutare tali processi e accertarsi che siano conformi agli obiettivi di qualità previsti. Da mettere in atto è Ciclo PDCA (Plan - Do - Check- Act), che garantisce un miglioramento continuo nell’utilizzo dei processi e delle risorse tramite una prima fase di pianificazione, seguita da una verifica con le metriche previste e infine un'integrazione o correzione del prodotto in base ai risultati precedentemente ottenuti.
\subsection{Processi primari}
\subsection{Processi di supporto}

\section{Qualità di prodotto}
Per assicurare l'elevata qualità del prodotto, è stata adottata come base di riferimento la norma ISO/IEC 12207:1997. In questa sezione vengono presentati i valori ideali e quelli accettabili relativi alle metriche scelte dal team Cyber Sorceres. Per una visione dettagliata delle metriche indicate in seguito, si prega di fare riferimento al documento \textit{Norme di progetto}.
\subsection{Obiettivi}
\begin{itemize}
    \item{Efficienza}
    \item {Usabilità}
    \item {Affidabilità}
    \item {Manutenibilità}
    \item {Portabilità}    
\end{itemize}

\begin{center}
\begin{tabular}{P{8em} P{20em} P{8em}} 
  \rowcolor{Blue}
    \custombold{Obiettivo} & \custombold{Descrizione} & \custombold{Metriche}\\
    \hline
    \rowcolor{LightBlue}
    &\custombold{Documentazione}&\\
    \hline
    \rowcolor{LighterBlue}
    \textbf{Leggibilità documenti} & La documentazione deve essere comprensibile agli utenti. & \textbf{MD01}\\
    \rowcolor{LightBlue}
    \textbf{Correttezza linguistica} & Non devono essere presenti errori grammaticali nella documentazione. & \textbf{MD02}\\
    \hline
    \rowcolor{LighterBlue}
    &\custombold{Software}&\\
    \hline
    \rowcolor{LightBlue}
    \textbf{Funzionalità} & La capacità del prodotto di fornire tutte le funzioni identificate nell'\textit{Analisi dei requisiti}, perseguendo precisione e idoneità. & \textbf{MS01}, \textbf{MS02}, \textbf{MS03}\\
    \rowcolor{LighterBlue}
    \textbf{Usabilità} & La capacità di essere comprensibile al fine di rendere gradevole l'esperienza dell'utente. Le funzionalità devono essere in linea con le aspettative e compatibili con le stesse. & \textbf{MS04}\\
    \rowcolor{LightBlue}
    \textbf{Portabilità} & La capacità di operare in vari contesti di esecuzione. Gli obiettivi da raggiungere includono adattabilità e sostituibilità. & \textbf{MS05}, \textbf{MS06}\\
    \rowcolor{LighterBlue}
    \textbf{Test} & L'intero codice prodotto sarà soggetto a verifica per assicurare l'implementazione corretta dei requisiti identificati. & \textbf{MS07}, \textbf{MS08}, \textbf{MS09}, \textbf{MS10}\\
\end{tabular}
\end{center}

\begin{center}
\begin{tabular}{P{5em} P{13em} P{10em} P{10em}} 
  \rowcolor{Blue}
    \custombold{Codice} & \custombold{Denominazione metrica} & \custombold{Valore accettabile} & \custombold{Valore ottimale}\\
    \hline
    \rowcolor{LighterBlue}
    \custombold{MD01} & Indice di Gulpease & $\geq 60$ & $\geq 80$\\
    \rowcolor{LightBlue}
    \custombold{MD02} & Errori ortografici & 0 & 0 \\
    \hline
    \rowcolor{LighterBlue}
    \custombold{MS01} & Copertura requisiti obbligatori & 100\% & 100\% \\
    \rowcolor{LightBlue}
    \custombold{MS02} & Copertura requisiti desiderabili & $\geq 50$\% & $\geq 100$\% \\
    \rowcolor{LighterBlue}
    \custombold{MS03} & Copertura requisiti opzionali & $\geq 50$\% & $\geq 100$\% \\
    \rowcolor{LightBlue}
    \custombold{MS04} & Facilità utilizzo & 5 click & 4 click \\
    \rowcolor{LighterBlue}
    \custombold{MS05} & Versioni browser supportate & $\geq 80$\% & $\geq 100$\% \\
    \rowcolor{LightBlue}
    \custombold{MS06} & Versioni VSCode supportate & $\geq 80$\% & $\geq 100$\% \\
    \rowcolor{LighterBlue}
    \custombold{MS07} & Solidity Statement Coverage & $\geq 80$\% & $\geq 100$\% \\
    \rowcolor{LightBlue}
    \custombold{MS08} & Solidity Branche Coverage & $\geq 80$\% & $\geq 100$\% \\
    \rowcolor{LighterBlue}
    \custombold{MS09} & Solidity Function Coverage & $\geq 80$\% & $\geq 100$\% \\
    \rowcolor{LightBlue}
    \custombold{MS10} & Solidity Line Coverage & $\geq 80$\% & $\geq 100$\% \\
\end{tabular}
\end{center}
\newpage


\section{Test e specifiche}
\subsection{Test di Unità}
\subsection{Test di Integrazione}
\subsection{Test di Sistema}
\subsection{Test di Accettazione}
\subsection{Test di Regressione}

\section{Valutazioni per il miglioramento}
In questo paragrafo cercheremo di analizzare le difficoltà che abbiamo avuto fino alla consegna e valutarne le rispettive soluzioni e miglioramenti adottati dal gruppo.
\subsection{Valutazione sull'organizzazione}
\begin{center}
\begin{tabular}{P{10em} P{13em} P{4em} P{13em}} 
    \rowcolor{Blue}
    \custombold{Criticità} & \custombold{descrizione} & \custombold{gravità} &
    \custombold{ Soluzione}\\
    \rowcolor{LighterBlue}
     & &  &  \\ 
    \rowcolor{LightBlue}
     & &  &  \\ 
\end{tabular}
\end{center}
\subsection{Valutazione sugli strumenti utilizzati}
\begin{center}
\begin{tabular}{P{10em} P{13em} P{4em} P{13em}} 
      \rowcolor{Blue}
    \custombold{Criticità} & \custombold{descrizione} & \custombold{gravità} &
    \custombold{ Soluzione}\\
    \rowcolor{LighterBlue}
     & &  &  \\ 
    \rowcolor{LightBlue}
     & &  &  \\ 
\end{tabular}
\end{center}
\subsection{Valutazione sui ruoli}
\begin{center}
\begin{tabular}{P{10em} P{13em} P{4em} P{13em}} 
  \rowcolor{Blue}
    \custombold{Criticità} & \custombold{descrizione} & \custombold{gravità} &
    \custombold{ Soluzione}\\
    \rowcolor{LighterBlue}
     & &  &  \\ 
    \rowcolor{LightBlue}
     & &  &  \\ 
\end{tabular}
\end{center}
\end{document}