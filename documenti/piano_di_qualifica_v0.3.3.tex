\documentclass{article}
\usepackage[utf8]{inputenc}
\usepackage[includeheadfoot, margin=1em,headheight=2em]{geometry}
\usepackage{titling}
\usepackage{hyperref}
\geometry{a4paper, left=2cm, right=2cm, top=2cm, bottom=2cm}
\usepackage{graphicx}
\providecommand{\versionnumber}{1.0.0}
\usepackage{enumitem}
\usepackage{array}
\newcolumntype{P}[1]{>{\centering\arraybackslash}p{#1}}
\renewcommand{\arraystretch}{1.5} % Default value: 1
\setlength{\droptitle}{-6em}
\usepackage{capt-of}
\usepackage{setspace}

%font
\usepackage[defaultfam,tabular,lining]{montserrat}
\usepackage[T1]{fontenc}
\renewcommand*\oldstylenums[1]{{\fontfamily{Montserrat-TOsF}\selectfont #1}}

%custom bold 
\usepackage[outline]{contour}
\usepackage{xcolor}
\newcommand{\custombold}{\contour{black}}

%table colors
\usepackage{color, colortbl}
\definecolor{Blue}{rgb}{0.51,0.68,0.79}
\definecolor{LightBlue}{rgb}{0.82,0.87,0.90}
\definecolor{LighterBlue}{rgb}{0.93,0.95,0.96}

\usepackage{caption}
\captionsetup[figure]{labelformat=empty}

%Header
\usepackage{fancyhdr, xcolor}
\pagestyle{fancy}
\let\oldheadrule\headrule% Copy \headrule into \oldheadrule
\renewcommand{\headrule}{\color{Blue}\oldheadrule}% Add colour to \headrule
\renewcommand{\headrulewidth}{0.2em}
\fancyhead[L]{Piano di qualifica}
\fancyhead[C]{Cybersorceres}
\fancyhead[R]{versione \versionnumber}

\title{\Huge{\textbf{Piano di Qualifica}}\vspace{-1em}}
\author{CyberSorcerers Team}
\date{}
\begin{document}
\maketitle
\vspace{-3em}
\begin{figure}[h]
  \centering
  \includegraphics[width=6cm, height=6cm]{documenti/logo rotondo.png}
  \label{fig:immagine}
\end{figure}

\vspace{6em}
\large{
\begin{center}
    \begin{tabular}{P{24em}}
        \rowcolor{Blue}
        \textbf{Membri del team:}\\
        \rowcolor{LighterBlue}
        \custombold{Sabrina Caniato}\\
        \rowcolor{LightBlue}
        \custombold{Giulia Dentone}\\
        \rowcolor{LighterBlue}
        \custombold{Nicola Lazzarin}\\
        \rowcolor{LightBlue}
        \custombold{Giovanni Moretti}\\
        \rowcolor{LighterBlue}
        \custombold{Andrea Rezzi}\\
        \rowcolor{LightBlue}
        \custombold{Samuele Vignotto}\\
    \end{tabular}
\end{center}

\begin{center}
    \begin{tabular}{l c c}
        \rowcolor{Blue} 
        \textbf{Informazioni sul documento} & &\\ [1 ex]
        \rowcolor{LighterBlue}
        Destinatari: & Prf. Tullio Vardanega & Prf. Riccardo Cardin\\ [1 ex]
        \rowcolor{LightBlue}
        G al pedice: & Consultare il Glossario & \\ [1 ex]
    \end{tabular}
\end{center}
    
\newpage

\textbf{Registro dei Cambiamenti - Changelog}
\begin{center}
\begin{tabular}{P{4em} P{6em} P{8em} P{8em} P{10em}} 
  \rowcolor{Blue}
    \custombold{Versione} & \custombold{Data} & \custombold{Autore} &
    \custombold{ Verificatore} & \custombold{Dettaglio}\\
    \rowcolor{LightBlue}
    0.4.0 & 08/03/2024 & Giulia Dentone & Samuele Vignotto & Inizio stesura della sezione testing\\
    \rowcolor{LighterBlue}
    0.3.3 & 22/01/2024 & Sabrina Caniato & Giulia Dentone & Update della qualità di processo\\
    \rowcolor{LightBlue}
    0.3.2 & 09/01/2024 & Giovanni Moretti & Nicola Lazzarin & Update delle valutazioni per il miglioramento\\
    \rowcolor{LighterBlue}
    0.3.1 & 02/01/2024 & Andrea Rezzi & Sabrina Caniato & Update delle valutazioni per il miglioramento \\
    \rowcolor{LightBlue}
    0.3.0 & 21/12/2023 & Andrea Rezzi & Sabrina Caniato & Aggiunta delle valutazioni per il miglioramento \\
    \rowcolor{LighterBlue}
    0.2.1 & 18/12/2023 & Giulia Dentone & Samuele Vignotto & Update della qualità di prodotto \\    
    \rowcolor{LightBlue}
    0.2.0 & 18/12/2023 & Samuele Vignotto & Giovanni Moretti & Descrizione della qualità di prodotto \\
    \rowcolor{LighterBlue}
    0.1.0 & 17/12/2023 & Sabrina Caniato & Andrea Rezzi & Descrizione della qualità di processo \\
    \rowcolor{LightBlue}
    0.0.1 & 14/12/2023 & Giovanni Moretti & Nicola Lazzarin &  Definizione struttura del documento e scheletro delle sezioni. Scrittura introduzione ed obiettivi delle diverse sezioni\\
\end{tabular}
\end{center}
\newpage
\tableofcontents
\newpage
\section{Introduzione e scopo}
Il Piano di Qualifica è un documento che ci prefissiamo di aggiornare periodicamente dato che definisce l'approccio e le strategie per garantire la qualità di un progetto software. Questo piano è parte integrante del processo di gestione della qualità e fornisce una linea guida dettagliata su come il controllo e l'assicurazione della qualità verranno implementati durante l'intero ciclo di vita\textsubscript{G} del progetto.\\
In questo documento cercheremo di definire delle metriche di misurazione dell'efficacia e dell'efficienza del progetto, in base anche agli accorgimenti forniti dal proponente.\\

Il piano di qualifica conterrà:

\begin{itemize}
    \item Definizione chiara degli obiettivi e delle metriche di qualità che il progetto propone di raggiungere.
        
    \item Specifica dei criteri che determineranno se il prodotto soddisfa gli standard di qualità stabiliti.
    
    \item Descrizione dettagliata dei processi di test che saranno implementati e la definizione delle strategie utilizzate per l'esecuzione di essi.

    \item Procedure per gestire eventuali deviazioni rispetto agli standard di qualità pianificati.

\end{itemize}

\subsection{Glossario}
I termini impiegati in questo testo potrebbero suscitare incertezze circa il loro significato, rendendo quindi necessaria una definizione per evitare ambiguità. Tali termini sono identificati da una lettera "G" maiuscola posta in pedice alla parola, e la loro spiegazione è fornita nel Glossario v1.0.0.

\subsection{Riferimenti}
\textbf{Riferimenti normativi}
\begin{itemize}
    \item \href{https://www.math.unipd.it/~tullio/IS-1/2023/Progetto/C7.pdf}{C7.pdf}
\end{itemize}
\textbf{Riferimenti informativi}
\begin{itemize}
    \item Argomento T7 - Qualità del software
    \item Argomento T8 - Qualità di processo
    \item Argomento T9 - Verifica e validazione
    \item - ISOG/IECG 9126:2001 SWE Product Quality;
    \item ISO/IEC 14598:1999 SW Product Evaluation;
    \item  ISO/IEC 25000:2005 SQuaRE: Systems and software Quality Requirements and Evaluation:
        \begin{itemize}
        \item 25010:2011 Quality model;
        \item 25020:2019 Quality measurement framework;
        \item 25030:2007 Quality requirements;
        \item 25040:2011 Quality evaluation.
    \end{itemize}
    \item ISO 9000:2015;
    \item ISO 9004:2018;
    \item ISO/IEC 33020:2019.
\end{itemize}

\section{Qualità di processo}
\subsection{Scopo ed obiettivi}
La qualità è determinata univocamente dai processi che compongono un prodotto, misurata attraverso che permettano di valutare tali processi e accertarsi che siano conformi agli obiettivi di qualità previsti. Da mettere in atto è Ciclo PDCA (Plan - Do - Check- Act)\textsubscript{G}, che garantisce un miglioramento continuo nell’utilizzo dei processi e delle risorse tramite una prima fase di pianificazione, seguita da una verifica con le metriche previste e infine un'integrazione o correzione del prodotto in base ai risultati precedentemente ottenuti.\\
\\
\begin{table}[h]
\centering
\begin{tabular}{P{10em} P{23em} P{10em}}
\rowcolor{Blue}
 \custombold{Obiettivo} & \custombold{Descrizione} & \custombold{Metriche} \\
 \hline
 \rowcolor{LightBlue}
 & \custombold{Processi primari} &\\
 \hline
 \rowcolor{LighterBlue}
  \custombold{Fornitura} & Procedura che implica la selezione di metodologie e risorse adeguate per soddisfare le esigenze del cliente. & \custombold{MC01}, \custombold{MC02}, \custombold{MC03}, \custombold{MC04}, \custombold{MC05}, \custombold{MC06}, \custombold{MC07}\\
\rowcolor{LightBlue}
\custombold{Sviluppo} & Procedura volta a creare un prodotto software che risponda alle necessità del cliente. & \custombold{MC08} \\
\hline
\rowcolor{LighterBlue}
& \custombold{Processi di supporto} &\\
\hline
\rowcolor{LightBlue}
 \custombold{Verifica} & Procedura mirata a verificare che ogni servizio realizzato soddisfi i requisiti specificati. & \custombold{MC09} \\
\rowcolor{LighterBlue}
 \custombold{Gestione della} \custombold{qualità} & Procedura volta a garantire la conformità del prodotto e dei servizi offerti agli standard predefiniti. & \custombold{MC10} \\
\hline
\rowcolor{LightBlue}
& \custombold{Processi organizzativi} &\\
\hline
\rowcolor{LighterBlue}
 \custombold{Gestione} \custombold{organizzativa} & Procedura dedicata a delineare le modalità di coordinamento del team. & \custombold{MC11} \\
\end{tabular}
\caption{Tabella dei processi}
\label{tab:processi}
\end{table}

\newpage
\subsection{Processi primari}
\begin{tabular}{P{5em} P{18em} P{8em} P{8em}}
\rowcolor{Blue}
 \custombold{Codice} & \custombold{Nome metrica} & \custombold{Valore} \custombold{accettabile} & \custombold{Valore ottimale}\\
 \hline
\rowcolor{LightBlue}
\multicolumn{4}{c}{\custombold{Fornitura}}\\
\hline
\rowcolor{LighterBlue}
\custombold{MC01} & Earned Value (EV) & $>0$ & $\leq EAC$ \\
\rowcolor{LightBlue}
\custombold{MC02} & Actual Cost (AC) & $\geq0$ & $\leq EAC$ \\
\rowcolor{LighterBlue}
\custombold{MC03} & Planned Value (PV) & $\geq0$ & $\leq BAC$ \\
\rowcolor{LightBlue}
\custombold{MC04} & Cost Variance (CV) & $\geq-10\%$ & $\geq0\%$ \\
\rowcolor{LighterBlue}
\custombold{MC05} & Schedule Variance (SV) & $\geq-10\%$ & $\geq0\%$ \\
\rowcolor{LightBlue}
\custombold{MC06} & Estimated At Completion (EAC) & $\geq BAC-3\%$; $\leq BAC+3\%$ & $=BAC$ \\
\rowcolor{LighterBlue}
\custombold{MC07} & Estimate To Complete (ETC) & $\geq0$ & $\leq EAC$ \\
\rowcolor{LightBlue}
\hline
\multicolumn{4}{c}{\custombold{Sviluppo}}\\
\hline
\rowcolor{LighterBlue}
\custombold{MC08} & Requirements Stability Index (RSI) & $\geq80\%$ & $100\%$\\

\end{tabular}
\captionof{table}{Tabella dei processi primari}
\label{tab:processiprimari}

\subsection{Processi di supporto}
\begin{tabular}{P{5em} P{18em} P{8em} P{8em}}
\rowcolor{Blue}
 \custombold{Codice} & \custombold{Nome metrica} & \custombold{Valore} \custombold{accettabile} & \custombold{Valore ottimale}\\
 \hline
\rowcolor{LightBlue}
\multicolumn{4}{c}{\custombold{Verifica}}\\
\hline
\rowcolor{LighterBlue}
\custombold{MC09} & Passed Tests & $\geq80\%$ & $100\%$ \\
\hline
\rowcolor{LightBlue}
\multicolumn{4}{c}{\custombold{Gestione della qualità}}\\
\hline
\rowcolor{LighterBlue}
\custombold{MC10} & Metrics Satisfied & $\geq85\%$ & $100\%$ \\
\end{tabular}

\captionof{table}{Tabella dei processi di supporto}
\label{tab:processisup}

\subsection{Processi organizzativi}
\begin{tabular}{P{5em} P{18em} P{8em} P{8em}}
\rowcolor{Blue}
 \custombold{Codice} & \custombold{Nome metrica} & \custombold{Valore} \custombold{accettabile} & \custombold{Valore ottimale}\\
 \hline
\rowcolor{LightBlue}
\multicolumn{4}{c}{\custombold{Gestione organizzativa}}\\
\hline
\rowcolor{LighterBlue}
\custombold{MC11} & Risks Found & $\leq5$ & $0$ \\
\end{tabular}
\captionof{table}{Tabella dei processi organizzativi}
\label{tab:processiorg}

\section{Qualità di prodotto}
Per assicurare l'elevata qualità del prodotto, è stata adottata come base di riferimento la norma ISO/IEC 12207:1997. In questa sezione vengono presentati i valori ideali e quelli accettabili relativi alle metriche scelte dal team Cyber Sorceres. Per una visione dettagliata delle metriche indicate in seguito, si prega di fare riferimento al documento \textit{Norme di progetto}.
\subsection{Obiettivi}
\begin{itemize}
    \item{Efficienza}
    \item {Usabilità}
    \item {Affidabilità}
    \item {Manutenibilità}
    \item {Portabilità}    
\end{itemize}

\begin{center}
\begin{tabular}{P{8em} P{20em} P{8em}} 
  \rowcolor{Blue}
    \custombold{Obiettivo} & \custombold{Descrizione} & \custombold{Metriche}\\
    \hline
    \rowcolor{LightBlue}
    &\custombold{Documentazione}&\\
    \hline
    \rowcolor{LighterBlue}
    \textbf{Leggibilità documenti} & La documentazione deve essere comprensibile agli utenti. & \textbf{MD01}\\
    \rowcolor{LightBlue}
    \textbf{Correttezza linguistica} & Non devono essere presenti errori grammaticali nella documentazione. & \textbf{MD02}\\
    \hline
    \rowcolor{LighterBlue}
    &\custombold{Software}&\\
    \hline
    \rowcolor{LightBlue}
    \textbf{Funzionalità} & La capacità del prodotto di fornire tutte le funzioni identificate nell'\textit{Analisi dei requisiti}, perseguendo precisione e idoneità. & \textbf{MS01}, \textbf{MS02}, \textbf{MS03}\\
    \rowcolor{LighterBlue}
    \textbf{Usabilità} & La capacità di essere comprensibile al fine di rendere gradevole l'esperienza dell'utente. Le funzionalità devono essere in linea con le aspettative e compatibili con le stesse. & \textbf{MS04}\\
    \rowcolor{LightBlue}
    \textbf{Portabilità} & La capacità di operare in vari contesti di esecuzione. Gli obiettivi da raggiungere includono adattabilità e sostituibilità. & \textbf{MS05}, \textbf{MS06}\\
    \rowcolor{LighterBlue}
    \textbf{Test} & L'intero codice prodotto sarà soggetto a verifica per assicurare l'implementazione corretta dei requisiti identificati. & \textbf{MS07}, \textbf{MS08}, \textbf{MS09}, \textbf{MS10}\\
\end{tabular}
\captionof{table}{Tabella degli obiettivi della qualità di prodotto}
\label{tab:qualitaProd}
\end{center}

\begin{center}
\begin{tabular}{P{5em} P{13em} P{10em} P{10em}} 
  \rowcolor{Blue}
    \custombold{Codice} & \custombold{Denominazione metrica} & \custombold{Valore accettabile} & \custombold{Valore ottimale}\\
    \hline
    \rowcolor{LighterBlue}
    \custombold{MD01} & Indice di Gulpease & $\geq 60$ & $\geq 80$\\
    \rowcolor{LightBlue}
    \custombold{MD02} & Errori ortografici & 0 & 0 \\
    \hline
    \rowcolor{LighterBlue}
    \custombold{MS01} & Copertura requisiti obbligatori & 100\% & 100\% \\
    \rowcolor{LightBlue}
    \custombold{MS02} & Copertura requisiti desiderabili & $\geq 50$\% & $\geq 100$\% \\
    \rowcolor{LighterBlue}
    \custombold{MS03} & Copertura requisiti opzionali & $\geq 50$\% & $\geq 100$\% \\
    \rowcolor{LightBlue}
    \custombold{MS04} & Facilità utilizzo & 5 click & 4 click \\
    \rowcolor{LighterBlue}
    \custombold{MS05} & Versioni browser supportate & $\geq 80$\% & $\geq 100$\% \\
    \rowcolor{LightBlue}
    \custombold{MS06} & Versioni VSCode supportate & $\geq 80$\% & $\geq 100$\% \\
    \rowcolor{LighterBlue}
    \custombold{MS07} & Solidity Statement Coverage & $\geq 80$\% & $\geq 100$\% \\
    \rowcolor{LightBlue}
    \custombold{MS08} & Solidity Branche Coverage & $\geq 80$\% & $\geq 100$\% \\
    \rowcolor{LighterBlue}
    \custombold{MS09} & Solidity Function Coverage & $\geq 80$\% & $\geq 100$\% \\
    \rowcolor{LightBlue}
    \custombold{MS10} & Solidity Line Coverage & $\geq 80$\% & $\geq 100$\% \\
\end{tabular}
\captionof{table}{Metriche per la qualità di prodotto}
\label{tab:metricheQualProd}
\end{center}
\newpage


\section{Test e specifiche}
Nella seguente sezione esporremo le varie metodologie di test, i loro obiettivi e i criteri di successo ottenuti. Per facilitare la fase di validazione e accertamento continuo della correttezza del prodotto il gruppo ha deciso di svolgere una verifica in parallelo allo sviluppo, conformandosi al "modello a V"\textsubscript{G}.  
\begin{center}
\includegraphics[width=15cm, height=10cm]{documenti/modello_a_v.png}
\end{center}

\subsection{Test di Unità}
Il \textit{test di unità} sono una tipologia di testing del software in cui singole unità o componenti del software vengono testate in isolamento. Le unità possono essere singole funzioni, procedure, metodi o classi. L'obiettivo del test di unità è verificare che ciascuna unità funzioni correttamente secondo le specifiche e che produca i risultati attesi. Abbiamo deciso che questo tipo di testing sarà in larga parte automatizzato, per ottimizzare i costi e le tempistiche dedicate a questo processo.

\subsection{Test di Integrazione}
I \textit{test di integrazione} sono una fase del processo di testing in cui le diverse unità o del software vengono combinate e testate insieme come gruppo. L'obiettivo principale è verificare che le singole unità, testate precedentemente in modo isolato tramite i test di unità, funzionino correttamente quando integrate e collegate tra loro. Durante i test di integrazione, vengono identificati e risolti eventuali problemi di interfacciamento tra le diverse unità e vengono verificate le interazioni tra di esse. L'obiettivo finale è garantire che l'intero sistema funzioni come previsto e che tutte le interazioni tra le sue parti siano corrette.

\subsection{Test di Sistema}
I \textit{test di sistema} sono una fase del processo che si concentra sull'analisi e la verifica del sistema nel suo complesso rispetto ai requisiti specificati. L'obiettivo principale è garantire che il sistema soddisfi tutte le funzionalità e i requisiti richiesti dal cliente o specificati nel documento di specifica dei requisiti. Durante i test di sistema, vengono eseguiti scenari di test realistici per simulare l'utilizzo del software in un ambiente di produzione. I risultati dei test di sistema sono utilizzati per valutare se il sistema è pronto per il rilascio o se sono necessari ulteriori miglioramenti e correzioni.

\subsection{Test di Accettazione}
I \textit{test di accettazione} sono una fase finale in cui il sistema viene valutato dal cliente per determinare se soddisfa i requisiti concordati e se è pronto per il rilascio. Questi test sono orientati a verificare che il sistema sia conforme alle aspettative e alle necessità degli utenti e che sia in grado di svolgere le funzioni previste in modo efficace ed efficiente. L'obiettivo principale è confermare che il software sia pronto per essere messo in produzione e che risponda alle aspettative del cliente. I risultati dei test di accettazione sono fondamentali per prendere decisioni riguardanti il rilascio del prodotto e possono influenzare eventuali modifiche o miglioramenti futuri.

\subsection{Test di Regressione}
I \textit{test di regressione} mirano a verificare che le modifiche apportate al codice sorgente o al sistema non abbiano introdotto nuovi difetti o rotto funzionalità esistenti. Questi test vengono eseguiti dopo ogni modifica al software, come aggiornamenti, correzioni di bug o nuove implementazioni. L'obiettivo è assicurarsi che le modifiche non abbiano impatti indesiderati sul comportamento del sistema, specialmente su funzionalità precedentemente testate e funzionanti correttamente. L'obiettivo del gruppo è raggiungere la massima automazione possibile dei test di regressione, al fine di ridurre i tempi di esecuzione e garantire una copertura completa dei test.

\section{Valutazioni per il miglioramento}
In questo paragrafo cercheremo di analizzare le difficoltà che abbiamo avuto fino alla consegna e valutarne le rispettive soluzioni e miglioramenti adottati dal gruppo.

\subsection{Valutazione sull'organizzazione}
\begin{center}
\begin{tabular}{P{10em} P{13em} P{4em} P{13em}} 
    \rowcolor{Blue}
    \custombold{Criticità} & \custombold{Descrizione} & \custombold{Gravità} &
    \custombold{Soluzione}\\
    \rowcolor{LighterBlue}
     Iniziale carenza di comunicazione con il cliente& Durante le prime fasi di sviluppo abbiamo avuto difficoltà ad ottenere le credenziali per utilizzare gli stumenti da loro richiesti & Bassa & Focalizzare il lavoro, durante l'attesa, nella redazione dei documenti e aprire un canale di comunicazione più veloce delle mail  \\ 
    \rowcolor{LightBlue}
     Disparità di impegno tra i membri& Alcuni membri, avendo magari più impegni accademici o lavorativi, sono stati meno presenti agli incontri o per la realizzazione del progetto & Media & Assegnare i compiti quanto più in maniera equa e realistica \\ 
\end{tabular}
\captionof{table}{Criticità sull'organizzazione}
\label{tab:org}
\end{center}

\subsection{Valutazione sugli strumenti utilizzati}
\begin{center}
\begin{tabular}{P{10em} P{13em} P{4em} P{13em}} 
    \rowcolor{Blue}
    \custombold{Criticità} & \custombold{Descrizione} & \custombold{Gravità} &
    \custombold{Soluzione}\\
    \rowcolor{LighterBlue}
     Complessità nell'integrazione del plugin & Non avendo mai sviluppato un plug in è stata difficoltosa la fase di integrazione & Bassa & Focalizzarsi sull'autoapprendimento e aggiungere uno Sviluppatore a discapito di altri ruoli più marginali in quella fase  \\ 
    \rowcolor{LightBlue}
     Repository & Difficoltà nel mantenimento dell'ordine, della linea temporale e della versioni dei documenti & Media & Focalizzare una delle fasi di verifica del Verificatore proprio sul controllo della Repository e fare delle sedute di formazione interna per chi avesse difficoltà nell'uso delle funzionalità più utilizzate dello strumento \\ 
     \rowcolor{LighterBlue}
     Amazon AWS  & Le librerie di Amazon AWS oltre ad essere moltissime, hanno tutte un prezzo diverso & Alta & Fissare un incontro di formazione da parte del proponente per scegliere in maniera mirata le librerie, in modo tale da essere conformi alle esigenze di costo e non perderci nella fase di analisi, studio dello strumento e scelta delle librerie
\end{tabular}
\captionof{table}{Criticità negli strumenti utilizzati}
\label{tab:strum}
\end{center}

\subsection{Valutazione sui ruoli}
\begin{center}
\begin{tabular}{P{10em} P{13em} P{4em} P{13em}} 
    \rowcolor{Blue}
    \custombold{Criticità} & \custombold{Descrizione} & \custombold{Gravità} &
    \custombold{Soluzione}\\
    \rowcolor{LighterBlue}
    Verifica superficiale da parte del verificatore& Alcuni errori sono sfuggiti durante la fase di verifica a causa di una valutazione superficiale & Media  & Implementazione di checklist di verifica più dettagliate\\ 
    \rowcolor{LightBlue}
     Sviluppatori non allineati agli standard di codifica & Non sempre il codice è stato conforme agli standard richiesti&  Media &  Sessioni di formazione sui codici di stile e revisione del codice condiviso\\ 
     \rowcolor{LighterBlue}
     Inesperienza dell'analista & Gli analisti, non avendo mai lavorato a un progetto di tale portata, hanno fatto fatica inizialmente ad individuare tutti i requisiti necessari dalle prime sedute & Media & Sessioni di brainstorming interne e con i proponenti \\
    \rowcolor{LightBlue}
    Pianificazione poco realistica da parte del Responsabile & Data l'inesperienza nell'ambito, la pianificazione e le aspettative sul carico di lavoro non sono state conformi alla realtà & Bassa & Ridefinire gli sprint ed effettuarne più frequentemente ma ognuno con un minor carico di lavoro per mantenere sempre alta la produttività dei membri
     
\end{tabular}
\captionof{table}{Criticità dei ruoli}
\label{tab:ruoli}
\end{center}
\end{document}
