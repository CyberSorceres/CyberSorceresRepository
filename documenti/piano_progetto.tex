\documentclass{article}
\usepackage[utf8]{inputenc}
\usepackage[includeheadfoot, margin=1em,headheight=2em]{geometry}
\usepackage{titling}
\geometry{a4paper, left=2cm, right=2cm, top=2cm, bottom=2cm}
\usepackage{graphicx}
\usepackage{enumitem}
\usepackage{array}
\usepackage[italian]{babel}
\newcolumntype{P}[1]{>{\centering\arraybackslash}p{#1}}
\renewcommand{\arraystretch}{1.5} % Default value: 1
\setlength{\droptitle}{-6em}

\providecommand{\versionnumber}{2.0.0}

%font
\usepackage[defaultfam,tabular,lining]{montserrat}
\usepackage[T1]{fontenc}
\renewcommand*\oldstylenums[1]{{\fontfamily{Montserrat-TOsF}\selectfont #1}}

%custom bold 
\usepackage[outline]{contour}
\usepackage{xcolor}
\newcommand{\custombold}{\contour{black}}

%table colors
\usepackage{color, colortbl}
\definecolor{Blue}{rgb}{0.51,0.68,0.79}
\definecolor{LightBlue}{rgb}{0.82,0.87,0.90}
\definecolor{LighterBlue}{rgb}{0.93,0.95,0.96}

%Header
\usepackage{fancyhdr, xcolor}
\pagestyle{fancy}
\let\oldheadrule\headrule% Copy \headrule into \oldheadrule
\renewcommand{\headrule}{\color{Blue}\oldheadrule}% Add colour to \headrule
\renewcommand{\headrulewidth}{0.2em}
\fancyhead[L]{Prevetivo costi e assunzione impegni}
\fancyhead[C]{}
\fancyhead[R]{}


\title{\Huge{\textbf{Prevetivo costi e assunzione impegni}}\vspace{-1em}}

\author{CyberSorcerers Team}
\date{}
\begin{document}
\maketitle

\vspace{-3em}
\begin{figure}[h]
  \centering
  \includegraphics[width=6cm, height=6cm]{documenti/logo rotondo.png}
  \label{fig:immagine}
\end{figure}

\vspace{6em}
\large{

\begin{center}
    \begin{tabular}{l c c}
        \rowcolor{Blue} 
        Informazioni sul documento & &\\ [1 ex]
        \rowcolor{LighterBlue}
        Redattori: & Sabrina Caniato & \\ [1 ex]
        \rowcolor{LightBlue}
        Verificatore: &  Nicola Lazzarin & \\ [1 ex]
        \rowcolor{LighterBlue}
        Destinatari: & Prf. Tullio Vardanega & Prf. Riccardo Cardin\\ [1 ex]


    \end{tabular}
\end{center}}
\newpage

\section{Dichiarazione impegni}

\subsection{Totale ore produttive assegnate}
La tabella seguente illustra la suddivisione delle ore per ciascun membro del gruppo:\\ \\
\begin{tabular}{|c|c|c|c|c|}
\hline
\rowcolor{LightBlue}
Ruolo & Costo orario & Ore per ruolo & Ore per membro & Costo totale \\
\hline
\hline
\rowcolor{LighterBlue}
Responsabile & 30 & 72 & 12 & 2160 \\
Amministratore & 20 & 72 & 12 & 1440 \\
\rowcolor{LighterBlue}

Analista & 25 & 108 & 18 & 2700 \\
Progettista & 25 & 102 & 17 & 2550 \\
\rowcolor{LighterBlue}

Programmatore & 15 & 126 & 21 & 1890 \\
Verificatore & 15 & 90 & 15 & 1350 \\
\hline
\hline
\rowcolor{LightBlue}
Totale &  & 570 & 95 & 12090 \\
\hline
\end{tabular} 
\\ \\ \\ Le ore assegnate per ciascun ruolo sono state distribuite in modo equo tra i membri del gruppo. \\
\begin{itemize}
  \item Il Responsabile risulta essere una figura fondamentale e di riferimento che servirà per tutto periodo del progetto. Ciò aiuta a garantire che ognuno svolga il proprio lavore nei tempi previsti.
  \item L'Analista e il Progettista ci serviranno per gran parte del progetto dato che l'analisi dei requisiti e il design sono punti chiavi per ottenere migliori risultati al momento dell'implementazione del codice.
  \item Il Programmatore sarà i ruolo principale avendo come obiettivo del lavoro implementare un software
  \item Il Verificatore dovrà vagliare tutte le ultime versioni dei documenti o delle parti di codice, indicando se soddisfano i requisiti richiesti.
\end{itemize}


\subsection{Preventivo dei costi finali}
Il costo finale del progetto ammonta a 12090 €, in base alla tabella mostrata sopra.

\subsection{Scadenza ultima di consegna prevista}
Il gruppo si impegna a consegnare il prodotto finito relativo al capitolato "ChatGPT vs BedRock developer Analysis" proposto dall'azienda Zero12 entro il \custombold{ 7 Aprile 2024}.\\
Per stimare questa data abbiamo considerato le 20 settimane consigliate per gruppi a disponibilità alta, lasciandoci un piccolo margine derivante da piccole problematiche o criticità che si potranno presentare. 


\end{document}
