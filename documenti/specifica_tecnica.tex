\documentclass{article}
\usepackage{graphicx}
\usepackage{float}  
\usepackage[utf8]{inputenc}
\usepackage[includeheadfoot, margin=1em,headheight=2em]{geometry}
\usepackage{titling}
\usepackage{hyperref}
\geometry{a4paper, left=2cm, right=2cm, top=2cm, bottom=2cm}
\usepackage{graphicx}
\providecommand{\versionnumber}{0.1.1}
\usepackage{enumitem}
\usepackage{array}
\newcolumntype{P}[1]{>{\centering\arraybackslash}p{#1}}
\renewcommand{\arraystretch}{1.5} % Default value: 1
\setlength{\droptitle}{-6em}
\usepackage{capt-of}
\usepackage{setspace}

%font
\usepackage[defaultfam,tabular,lining]{montserrat}
\usepackage[T1]{fontenc}
\renewcommand*\oldstylenums[1]{{\fontfamily{Montserrat-TOsF}\selectfont #1}}

%custom bold 
\usepackage[outline]{contour}
\usepackage{xcolor}
\newcommand{\custombold}{\contour{black}}

%table colors
\usepackage{color, colortbl}
\definecolor{Blue}{rgb}{0.51,0.68,0.79}
\definecolor{LightBlue}{rgb}{0.82,0.87,0.90}
\definecolor{LighterBlue}{rgb}{0.93,0.95,0.96}

\usepackage{caption}
\captionsetup[figure]{labelformat=empty}

%Header
\usepackage{fancyhdr, xcolor}
\pagestyle{fancy}
\let\oldheadrule\headrule% Copy \headrule into \oldheadrule
\renewcommand{\headrule}{\color{Blue}\oldheadrule}% Add colour to \headrule
\renewcommand{\headrulewidth}{0.2em}
\fancyhead[L]{Specifica Tecnica}
\fancyhead[C]{Cybersorceres}
\fancyhead[R]{versione \versionnumber}

\title{\Huge{\textbf{Specifica Tecnica}}\vspace{-1em}}
\author{CyberSorcerers Team}
\date{}
\begin{document}
\maketitle
\vspace{-3em}
\begin{figure}[h]
  \centering
  \includegraphics[width=6cm, height=6cm]{documenti/logo rotondo.png}
  \label{fig:immagine}
\end{figure}

\vspace{6em}
\large{
\begin{center}
    \begin{tabular}{P{24em}}
        \rowcolor{Blue}
        \textbf{Membri del team:}\\
        \rowcolor{LighterBlue}
        \custombold{Sabrina Caniato}\\
        \rowcolor{LightBlue}
        \custombold{Giulia Dentone}\\
        \rowcolor{LighterBlue}
        \custombold{Nicola Lazzarin}\\
        \rowcolor{LightBlue}
        \custombold{Giovanni Moretti}\\
        \rowcolor{LighterBlue}
        \custombold{Andrea Rezzi}\\
        \rowcolor{LightBlue}
        \custombold{Samuele Vignotto}\\
    \end{tabular}
\end{center}

\begin{center}
    \begin{tabular}{l c c}
        \rowcolor{Blue} 
        \textbf{Informazioni sul documento} & &\\ [1 ex]
        \rowcolor{LighterBlue}
        Destinatari: & Prof Tullio Vardanega & Prof Riccardo Cardin\\ [1 ex]
        \rowcolor{LightBlue}
        G al pedice: & Consultare il Glossario & \\ [1 ex]
    \end{tabular}
\end{center}
    
\newpage

\textbf{Registro dei Cambiamenti - Changelog\textsubscript{G}}
\begin{center}
\begin{tabular}{P{4em} P{6em} P{8em} P{8em} P{10em}} 
    \rowcolor{LightBlue}
    0.2.0 & 23/05/2024 & Samuele Vignotto & Giovanni Moretti & Stesura della sezione 'Requisiti soddisfatti'.\\
    \rowcolor{LighterBlue}
    0.1.1 & 23/05/2024 & Giulia Dentone & Sabrina Caniato & Stesura della sezione 'Tecnologie'.\\
    \rowcolor{LightBlue}
    0.0.1 & 03/05/2024 & Giulia Dentone & Samuele Vignotto &  Definizione struttura del documento e scheletro delle sezioni. Scrittura introduzione ed obiettivi delle diverse sezioni.\\
\end{tabular}
\end{center}
\newpage
\tableofcontents
\newpage

\section{Introduzione}
\subsection{Scopo del documento}
Questo documento ha lo scopo di delineare e giustificare le decisioni architetturali prese durante le fasi di progettazione e sviluppo del prodotto. Sono presentati i diagrammi dei componenti React e dei pacchetti per illustrare le scelte dei pattern architetturali adottati per realizzare la struttura finale del prodotto. Inoltre, viene fornita una sezione dedicata ai requisiti soddisfatti dal team, offrendo così una panoramica completa dello stato di avanzamento del lavoro.
\subsection{Scopo del prodotto}
L'azienda proponente ha richiesto la creazione di una web app\textsubscript{G} che, tramite l'uso di IA\textsubscript{G} (in questo caso ChatGPT4 e Bedrock) è in grado di creare epic user stories\textsubscript{G} a partire dalle richieste del cliente e confrontarle con il codice sviluppato in modo da informare il cliente dello stato di avanzamento dello sviluppo del prodotto. Inoltre deve essere possibile, sia per il Project Manager\textsubscript{G}, sia per il cliente rilasciare dei feedback (nel primo caso riguardanti l'adeguatezza delle stories, nel secondo caso riguardanti il prodotto finale) al fine di migliorare l'IA\textsubscript{G}. È inoltre richiesta un' analisi comparativa tra le due IA\textsubscript{G} utilizzate e lo sviluppo di un plug-in\textsubscript{G} utile agli sviluppatori e al Project Manager\textsubscript{G}.

\subsection{Glossario}
Alcuni termini presenti nel documento potrebbero essere ambigui, pertanto verranno inseriti nel Glossario v.1.0.0. La loro presenza all'interno di esso sarà indicata tramite una G maiuscola a pedice.

\section{Riferimenti}
\subsection{Riferimenti normativi}
\begin{itemize}
    \item Capitolato \textbf{C7 - ChatGPT vs BedRock developer Analysis}
    \\ \\
       \href{https://github.com/CyberSorceres/CyberSorceresRepository}{https://github.com/CyberSorceres/CyberSorceresRepository} 
    \item Norme del way of working v 1.0.0
    \item Regolamento del progetto didattico \\ \\ \href{https://www.math.unipd.it/~tullio/IS-1/2023/Dispense/PD2.pdf} 
    {https://www.math.unipd.it/~tullio/IS-1/2023/Dispense/PD2.pdf}
\end{itemize}
\subsection{Riferimenti informativi}
\begin{itemize}
    \item Slide del corso di Ingegneria del Software - Analisi dei requisiti \\ \\
    \href{https://www.math.unipd.it/~tullio/IS-1/2023/Dispense/T5.pdf}{https://www.math.unipd.it/~tullio/IS-1/2023/Dispense/T5.pdf}
    \item Slide del corso di Ingegneria del Software - Progettazione e programmazione: Diagrammi delle classi \\ \\
\href{https://www.math.unipd.it/~rcardin/swea/2023/Diagrammi%20delle%20Classi.pdf}{https://www.math.unipd.it/~rcardin/swea/2023/Diagrammi\%20delle\%20Classi.pdf}
    \item Slide del corso di Ingegneria del Software - Solid Programming \\ \\
\href{https://www.math.unipd.it/~rcardin/swea/2021/SOLID\%20Principles\%20of\%20Object-Oriented\%20Design_4x4.pdf}{\texttt{https://www.math.unipd.it/~rcardin/swea/2021/SOLID\%20Principles\%20of\%20Object-Oriented\%20Design\_4x4.pdf}}
\end{itemize}
\subsection{Riferimenti tecnici}
\begin{itemize}
\item Documentazione di React \\ \href{ https://react.dev/}{ https://react.dev/}
\item Documentazione di Typescript \\ \href{https://www.typescriptlang.org/docs/}{https://www.typescriptlang.org/docs/}
\item Documentazione di MongoDB \\ \href{https://www.mongodb.com/docs/}{https://www.mongodb.com/docs/}
\item Documentazione di Amazon AWS \\ \href{https://docs.aws.amazon.com/it_it/}{https://docs.aws.amazon.com/it\_it/}
\item Serverless Microservice Patterns \\\href{https://medium.com/@jeremydaly/serverless-microservice-patterns-for-aws-6dadcd21bc02}{https://medium.com/@jeremydaly/serverless-microservice-patterns-for-aws-6dadcd21bc02}
\item Aws Reference Architecture Diagrams \\ \href{https://aws.amazon.com/it/architecture/reference-architecture-diagrams}{https://aws.amazon.com/it/architecture/reference-architecture-diagrams}
\item React design patterns \\ \href{https://refine.dev/blog/react-design-patterns/}{https://refine.dev/blog/react-design-patterns/}
\end{itemize}

\section{Tecnologie}
In questa sezione è presente una panoramica generale delle tecnologie necessarie per la realizzazione del prodotto (in particolare del front-end, back-end,databse e plugIn), gli strumenti e le librerie utilizzate per lo sviluppo, il testing e la distribuzione.

\subsection{Tecnologie per la codifica}

\begin{center}
\begin{tabular}{|P{8em}|P{18em}|P{5em}|}
\hline
\rowcolor{Blue}
Tecnologia & Descrizione & Versione \\
\hline
\rowcolor{LighterBlue}
\multicolumn{3}{|c|}{Linguaggi} \\
\hline
\rowcolor{LightBlue}
HTML & Linguaggio di markup per delineare la struttura delle pagine e definire i componenti dell'interfaccia. & 5 \\
\rowcolor{LighterBlue}
CSS & Linguaggio per la gestione dello stile degli HTML & 3 \\
\rowcolor{LightBlue}
Javascript & Superset di JavaScript per utilizzare tipizzazione & 5.0.x \\
\hline
\rowcolor{LighterBlue}
\multicolumn{3}{|c|}{Framework} \\
\hline
\rowcolor{LightBlue}
React & Libreria grafica per lo sviluppo front-end che permette di gestire le unità grafiche in maniera modulare& 18.0.x \\
\hline
\rowcolor{LighterBlue}
\multicolumn{3}{|c|}{Servizi e strumenti} \\
\hline
\rowcolor{LightBlue}
Node.js & Ambiente di runtime open-source per l'esecuzione di codice JavaScript lato
server tramite appositi script. & 19.0.x \\
\rowcolor{LighterBlue}
NPM & Gestore dell'installazione della gestione dei pacchetti utilizzati in TypeScript e nell'ambiente di esecuzione Node.js. & 3 \\
\rowcolor{LightBlue}
AWS Cognito & Servizio di gestione dell'autenticazione. & 2023-16-02\\
\rowcolor{LighterBlue}
AWS MongoDB & Servizio di database non relazionale gestito in modo scalabile. & 2019-11-21\\
\rowcolor{LightBlue}
AWS Lambda & Servizio che consente di eseguire codice in maniera serverless, garantendo la scalabilità automatica durante l'esecuzione. & 2023-03-16 \\
\rowcolor{LighterBlue}
AWS API Gateway & Servizio di gestione (creazione, pubblicazione e protezione) delle API. & 2023-04-06\\
\rowcolor{LightBlue}
Git & Sistema di controllo del versionamento e della gestione del codice. & 2.4.x\\
\hline
\end{tabular}
\captionof{table}{Tabella delle tecnologie per la codifica}
\label{tab:teccod}
\end{center}


\subsection{Tecnologie per l'analisi del codice}
\begin{center}
\begin{tabular}{|P{8em}|P{18em}|P{5em}|}
\hline
\rowcolor{Blue}
Tecnologia & Descrizione & Versione \\
\rowcolor{LighterBlue}
ViTest & Framework di test per TypeScript che permette la creazione di mock e il testing del codice in modo asincrono. & 1.6.0\\ 
\rowcolor{LightBlue}
\hline
\end{tabular}
\end{center}


\section{Architettura}
Nella fase di progettazione è stata scelta un'architettura a microservizi come la più conforme alle carateristiche di funzionamento e strutturali delle tecnologie AWS, soprattutto API Gateway. Inoltre è quella che consente nel nostro caso la comunicazione  per consentire la comunicazione tra le diverse componenti, considerando anche la presenza di un plugIn. Dati i requisiti del nostro progetto, abbiamo deciso che non fosse adeguato adottare un'unica struttura architetturale per l'intera l'infrastruttura. Abbiamo suddiviso il sistema in tre parti principali: 
\begin{itemize}
    \item Front-end: la parte client dell'applicazione eseguibile localmente su qualsiasi browser.
    \item Back-end: utilizza le tecnologie AWS (listate nella sezione 3.1), con l'interazione tramite NodeJS lato client e la comunicazione con il plugIn. 
    \item PlugIn: comunica con il back-end grazie ad una libreria sviluppata dal team.
\end{itemize}
La comunicazione tra le diverse parti avviene attraverso l'uso di API Gateway.

\subsection{Architettura front-end}


\subsubsection{Pattern utilizzati}
\subsection{Architettura back-end}
\subsubsection{Pattern utilizzati}


\section{Requisiti soddisfatti}
\begin{center}
    \begin{tabular}{|P{2cm}| P{10cm}| P{4cm}|}
    \rowcolor{Blue} 
\hline
Codice & Descrizione & Stato  \\ 
\rowcolor{LightBlue}
\hline
ROF1 & Accesso a web app tramite login composto da email e password. & Soddisfatto \\ 
\rowcolor{LighterBlue}
\hline
ROF2& Scrittura di richieste di business tramite box testuale da web app. & Soddisfatto\\ 
\rowcolor{LightBlue}
\hline
ROF3& Invio delle richieste di businessda web app. & Soddisfatto\\
\hline
\rowcolor{LighterBlue}
ROF4& Visualizzazione andamento sviluppo richieste tramite barra di completamento basata sulla percentuale di user stories completate. & Soddisfatto\\
\rowcolor{LightBlue}
\hline
ROF5 & Approvazione o rifiuto del risultato relativo all'implementazione di una user story. & Soddisfatto \\
\hline
\rowcolor{LighterBlue}
RDF6&  Ricezione notifiche quando user story completata. & Non soddisfatto\\
\hline
\rowcolor{LightBlue}
\hline
ROF7&  Funzionalità di tag nel plug-in.  & Soddisfatto\\
\hline
\rowcolor{LighterBlue}
ROF8 & Lista di user stories assegnate da Project Manager sia su web app che su plug-in. & Soddisfatto \\
\hline
\rowcolor{LightBlue}
RDF9 & Ricezione notifica su web app quando nuova user story è assegnata dal Project Manager. & Soddisfatto\\
\hline
\rowcolor{LighterBlue}
ROF10 & Invio del codice sviluppato a IA per richiesta verifica.& Soddisfatto\\
\hline
\rowcolor{LightBlue}
ROF11 & Visualizzazione user stories generate da IA.  &Soddisfatto\\
\hline
\rowcolor{LighterBlue}
\hline
ROF12& Invio di feedback sulle user stories generate all'IA.& Soddisfatto \\
\hline
\rowcolor{LightBlue}
ROF13&Suddivisione delle user stories troppo grandi.  & Soddisfatto\\
\hline
\rowcolor{LighterBlue}
ROF14 & Assegnazione user storieS agli sviluppatori.& Soddisfatto\\
\hline
\rowcolor{LightBlue}
RDF15 & Ricezione notifiche quando user story\textsubscript{G} viene generata in seguito a richiesta del cliente. & Non soddisfatto\\
\hline
\rowcolor{LighterBlue}
ROF16& Invio richiesta di modifiche relative a user stories a IA prima di approvazione. & Soddisfatto\\
\hline
\rowcolor{LightBlue}
ROF17&Visualizzazione andamento epic/user stories assegnate.& Soddisfatto\\
\hline
\rowcolor{LighterBlue}
ROF18& Creazione di un plug-in per VSCode.& Soddisfatto\\
\hline
\rowcolor{LightBlue}
RDF19& Creazione di un plug-in per XCode.& Soddisfatto\\
\hline
\rowcolor{LighterBlue}
ROF20& I linguaggi supportati dal plug-in sono Typescript e Javascript. & Soddisfatto\\
\hline
\end{tabular}


    \begin{tabular}{|P{2cm}| P{10cm}| P{4cm}|}
\hline
\rowcolor{LightBlue}
\hline
\rowcolor{LightBlue}
RDF21 & Altri linguaggi che potrebbero essere supportati in futuro sono Kotlin\textsubscript{G} e Swift.& Non soddisfatto\\
\rowcolor{LighterBlue}
ROF22 & Gestione degli input (prevenzione da Injection Cross Site Scripting e sanificazione dell'input.) & Soddisfatto \\
ROQ1& Il progetto deve essere accessibile pubblicamente su GitHub o su un'altra repository pubblica. & Soddisfatto\\ 
\rowcolor{LighterBlue}
\hline
ROQ2 & Il prodotto deve essere sviluppato conformemente a quanto stabilito nelle \textit{Norme Way of Working\textsubscript{G}}.& Soddisfatto \\ 
\rowcolor{LightBlue}
\hline
ROQ3 & Deve essere effettuato il testing delle unità e dell'integrazione con una copertura minima dell'80\%.& Soddisfatto\\
\hline
\rowcolor{LighterBlue}
ROQ4 & Deve essere fornita una documentazione completa sulle scelte implementative e progettuali effettuate. & Soddisfatto\\
\hline
\rowcolor{LightBlue}
ROQ5& Deve essere fornito un manuale per l'utilizzo del prodotto. & Soddisfatto\\
\hline
\rowcolor{LighterBlue}
ROQ6& Deve essere fornita una documentazione che compara la capacità di ChatGPT e quella di AWS Bedrock nell'interpretare del codice sorgente ed associare le user stories generate. & Soddisfatto\\
\hline
\rowcolor{LightBlue}
ROQ7&  Deve essere fornita una documentazione che prova un'interpretazione corretta da parte dell'IA che si basa: sulle epic/user stories generate dall'IA, i test generati dall'IA, i criteri di accettazione delle epic/user stories forniti dal proponente. & Soddisfatto\\
\rowcolor{LighterBlue}
\hline
ROV1& L'applicazione per l'interazione con la piattaforma dev'essere sviluppata attraverso l'uso di tecnologie web. & Soddisfatto\\ 
\hline
\rowcolor{LightBlue}
ROV2& Le due IA utilizzate per l'analisi sono AWS Bedrock e ChatGPT. & Soddisfatto \\ 
\rowcolor{LighterBlue}
\hline
RDV3& L'applicazione deve essere utilizzabile tramite browser (Chrome 123.0, Firefox 124.0, Safari 17.0) di dispositivi mobili(Android 14.0, iOS 17.0). & Non soddisfatto\\
\hline
\rowcolor{LightBlue}
ROV4& Il front-end dell'applicazione verrà sviluppato in React. & Soddisfatto\\
\rowcolor{LighterBlue}
\hline
ROV5& Ogni AWS Lambda-function deve essere sviluppata in Node.js.& Soddisfatto\\
\hline
\rowcolor{LightBlue}
ROV6& Tutte le API devono essere integrate in AWS API-gateway.& Soddisfatto\\
\hline
\end{tabular}


    \begin{tabular}{|P{2cm}| P{10cm}| P{4cm}|}
\hline
\rowcolor{LighterBlue}
ROV7& L'applicativo deve essere compatibile con il browser Google Chrome dalla versione 121. & Soddisfatto\\
\hline
\rowcolor{LightBlue}
\hline
\rowcolor{LightBlue}
ROV8& L'applicativo deve essere compatibile con il browser Firefox dalla versione 122. & Soddisfatto\\
\hline
\rowcolor{LighterBlue} 
ROV9 & L'applicativo deve essere compatibile con il browser Microsoft Edge dalla versione 121. & Soddisfatto\\
\hline
\rowcolor{LightBlue}
ROV10 & Il plug-in deve essere compatibile con VSCode dalla versione 1.84.1 . & Soddisfatto\\
\hline
\rowcolor{LighterBlue}
ROV11 &  L'accesso deve essere controllato da AWS Cognito, con autenticazione univoca. & Soddisfatto\\
\hline
\rowcolor{LightBlue}
ROV12 & I ruoli devono essere definiti all'interno della piattaforma per evitare accessi non autorizzati. & Soddisfatto\\
\hline
\rowcolor{LighterBlue}
ROV13 & Protezione delle informazioni trasmesse tra browser e server tramite protocollo SSL/TLS. & Soddisfatto\\
\hline
\end{tabular}
\captionof{table}{Tabella dei requisiti soddisfatti}
\label{tab:reqvincolo}
\end{center}

\subsection{Resoconto dei requisiti soddisfatti}
\section{Requisiti soddisfatti}
\begin{center}
    \begin{tabular}{|P{7cm}| P{4cm}| P{4cm}|}
    \hline
    \rowcolor{Blue} 
    Tipologia requisito & Istanze & Totale soddisfatto\\
    \rowcolor{LighterBlue}
    Requisito funzionale & 22 & 90.91\%\\
    \rowcolor{LightBlue}
    Requisito obbligatorio & 36 & 100\% \\
\hline
\end{tabular}
\end{center}


\end{document}
