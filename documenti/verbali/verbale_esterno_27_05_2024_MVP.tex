\documentclass{article}
\usepackage[utf8]{inputenc}
\usepackage[includeheadfoot, margin=1em,headheight=2em]{geometry}
\usepackage{titling}
\geometry{a4paper, left=2cm, right=2cm, top=2cm, bottom=2cm}
\usepackage{graphicx}
\usepackage{enumitem}
\usepackage{array}
\newcolumntype{P}[1]{>{\centering\arraybackslash}p{#1}}
\renewcommand{\arraystretch}{1.5} % Default value: 1
\setlength{\droptitle}{-6em}
\providecommand{\versionnumber}{1.0.0}

%font
\usepackage[defaultfam,tabular,lining]{montserrat}
\usepackage[T1]{fontenc}
\renewcommand*\oldstylenums[1]{{\fontfamily{Montserrat-TOsF}\selectfont #1}}

%custom bold 
\usepackage[outline]{contour}
\usepackage{xcolor}
\newcommand{\custombold}{\contour{black}}

%table colors
\usepackage{color, colortbl}
\definecolor{Blue}{rgb}{0.51,0.68,0.79}
\definecolor{LightBlue}{rgb}{0.82,0.87,0.90}
\definecolor{LighterBlue}{rgb}{0.93,0.95,0.96}

%Header
\usepackage{fancyhdr, xcolor}
\pagestyle{fancy}
%\fancyhead[L]{\includegraphics[width=4em]{CS_tr.png}}
%\fancyhead[C]{Verbale riunione di progetto - gg/mm/aaaa}
%\fancyhead[R]{\includegraphics[width=4em]{CS_tr.png}}
\let\oldheadrule\headrule% Copy \headrule into \oldheadrule
\renewcommand{\headrule}{\color{Blue}\oldheadrule}% Add colour to \headrule
\renewcommand{\headrulewidth}{0.2em}
\fancyhead[L]{CyberSorcerers - Attestazione MVP}
\fancyhead[C]{}
\fancyhead[R]{12/12/2023}

\title{\Huge{\textbf{Verbale della riunione esterna - attestazione MVP}}\vspace{-1em}}
\date{}
\begin{document}
\maketitle
\vspace{-3em}
\begin{figure}[h]
  \centering
  \includegraphics[width=6cm, height=6cm]{documenti/logo rotondo.png}
  \label{fig:immagine}
\end{figure}

\begin{center}
\Large{27 Maggio 2021 - 11:00\\
Sede Zero12 s.r.l. - Google Meet\\
Verbale redatto da: \\}
\end{center}
\vspace{2em}
\large{
\begin{center}
    \begin{tabular}{P{24em}}
        \rowcolor{Blue}
        \textbf{Partecipanti alla riunione}\\
        \rowcolor{LightBlue}
        \custombold{Giulia Dentone}: presente\\
        \rowcolor{LighterBlue}
        \custombold{Giovanni Moretti}: presente\\
        \rowcolor{LightBlue}
        \custombold{Sabrina Caniato}: presente\\
        \rowcolor{LighterBlue}
        \custombold{Nicola Lazzarin}: presente\\
        \rowcolor{LightBlue}
        \custombold{Samuele Vignotto}: presente\\
        \rowcolor{LighterBlue}
        \custombold{Andrea Rezzi}: presente\\
        \rowcolor{LightBlue}
        \custombold{Michele Massaro - Zero12}: presente\\
        \rowcolor{LighterBlue}
        \custombold{Francesco Battistella - Zero12}: presente\\
        \rowcolor{LightBlue}
        \custombold{Samuele De Simone - Zero12}: presente\\
    \end{tabular}
\end{center}}


\vspace{50pt}
Responsabile interno: \line(1,0){175}\par
\vspace{35pt}
Responsabile esterno: \line(1,0){175}

\newpage

\textbf{Oggetto:} Attestazione di Minimum Viable Product.
A seguito dell’incontro avvenuto in data 27/05/2024, alla presenza di Michele Massaro, Francesco Battistella, Samuele De Simone, in qualità di rappresentanti di Zero12 s.r.l.:
\begin{itemize}
    \item Considerata la definizione di Minimum Viable Product, da ora denominato MVP, quale \textit{”prodotto software non finale né definitivo, con lo scopo di approssimare il prodotto atteso e dotato di funzionalità sufficienti per: valutare la bontà della visione iniziale di prodotto; consentire agevole uso esplorativo; prendere decisioni ben fondate per il completamento definitivo del prodotto.}” come da regolamento del progetto;
    \item Considerata l’esposizione e dimostrazione del prodotto software realizzato e delle funzionalità in esso implementate.
\end{itemize}

Con il presente documento si attesta che il prodotto software realizzato dal team Cybersorcerers, nella realizzazione del progetto denominato
\newline
\begin{center}
    ChatGPT vs BedRock developer Analysis
\end{center}
e proposto dall’azienda Zero12, riceve ufficialmente lo stato di MVP.
In particolare:
\begin{itemize}
    \item Si attesta la bontà del prodotto presentato, come approssimazione dell’effettivo prodotto atteso;
    \item Si attesta l’implementazione di tutte le funzionalità con cui soddisfare i requisiti minimi, definiti come requisiti obbligatori, suffcienti ad attribuire lo stato di MVP al prodotto.
\end{itemize}

\end{document}
