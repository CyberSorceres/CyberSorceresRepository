\documentclass{article}
\usepackage[utf8]{inputenc}
\usepackage[includeheadfoot, margin=1em,headheight=2em]{geometry}
\usepackage{titling}
\geometry{a4paper, left=2cm, right=2cm, top=2cm, bottom=2cm}
\usepackage{graphicx}
\usepackage{enumitem}
\usepackage{array}
\newcolumntype{P}[1]{>{\centering\arraybackslash}p{#1}}
\renewcommand{\arraystretch}{1.5} % Default value: 1
\setlength{\droptitle}{-6em}

%font
\usepackage[defaultfam,tabular,lining]{montserrat}
\usepackage[T1]{fontenc}
\renewcommand*\oldstylenums[1]{{\fontfamily{Montserrat-TOsF}\selectfont #1}}

%custom bold 
\usepackage[outline]{contour}
\usepackage{xcolor}
\newcommand{\custombold}{\contour{black}}

%table colors
\usepackage{color, colortbl}
\definecolor{Blue}{rgb}{0.51,0.68,0.79}
\definecolor{LightBlue}{rgb}{0.82,0.87,0.90}
\definecolor{LighterBlue}{rgb}{0.93,0.95,0.96}

%Header
\usepackage{fancyhdr, xcolor}
\pagestyle{fancy}
%\fancyhead[L]{\includegraphics[width=4em]{logo rotondo.png}}
%\fancyhead[C]{Verbale riunione di progetto - gg/mm/aaaa}
%\fancyhead[R]{\includegraphics[width=4em]{CS_tr.png}}
\let\oldheadrule\headrule% Copy \headrule into \oldheadrule
\renewcommand{\headrule}{\color{Blue}\oldheadrule}% Add colour to \headrule
\renewcommand{\headrulewidth}{0.2em}
\fancyhead[L]{CyberSorcerers - Verbale di riunione interna}
\fancyhead[C]{}
\fancyhead[R]{05/03/2024}

\title{\Huge{\textbf{Verbale di riunione}}\vspace{-1em}}
\date{}
\begin{document}
\maketitle
\vspace{-3em}
\begin{figure}[h]
  \centering
  \includegraphics[width=6cm, height=6cm]{documenti/logo rotondo.png}
  \label{fig:immagine}
\end{figure}

\begin{center}
\Large{05 Marzo 2024 - 14:30\\
Google Meet\\
Verbale redatto da: Samuele Vignotto\\}
\end{center}
\vspace{2em}
\large{
\begin{center}
    \begin{tabular}{P{24em}}
        \rowcolor{Blue}
        \textbf{Partecipanti alla riunione}\\
        \rowcolor{LightBlue}
        \custombold{Giulia Dentone}: presente\\
        \rowcolor{LighterBlue}
        \custombold{Giovanni Moretti}: presente\\
        \rowcolor{LightBlue}
        \custombold{Sabrina Caniato}: presente\\
        \rowcolor{LighterBlue}
        \custombold{Nicola Lazzarin}: presente\\
        \rowcolor{LightBlue}
        \custombold{Samuele Vignotto}: presente \\
        \rowcolor{LighterBlue}
        \custombold{Andrea Rezzi}: presente \\
    \end{tabular}
\end{center}}
\newpage
\section{Obiettivo della riunione}
Presentazione del lavoro svolto rigurdante il PoC ed assegnazione compiti all'interno del team.
\section{Agenda}
\subsection{Argomenti trattati}
\subsubsection{plug-in}
È stato mostrato il funzionamento della creazione dei test da perte di Bedrock, inoltre è stato presentato il comando che permette di lanciare i test e controllare se sono passati.
\subsubsection{web app}
È stato mostrato come ci siano tre visualizzazioni diverse:\begin{itemize}
    \item user
    \item dev
    \item pm
\end{itemize}
\subsection{Decisioni}
Nella web app del PoC è stato deciso di includere anche un collegamento con Bedrock per dimostrare la fattibilita, mentre nel plug-in è stato deciso di aggiungere un collegamento con MongoDB per dimostrare la fattibilità.\\
È stato deciso di suddividere i seguenti compiti:\begin{itemize}
    \item Creazione Lambda function per collegamento plug-in e MongoDB (Andrea Rezzi).
    \item Integrazione dell'API per collegamento plug-in e MongoDB e pulizia del codice (Giovanni Moretti).
    \item Verifica documentazione per RTB (Samuele Vignotto).
    \item Continuazione della stesura della presentazione per RTB (Giulia Dentone).
    \item Ultimazione frontend PoC per RTB (Nicola Lazzarin e Sabrina Caniato).
\end{itemize}
Inoltre è stato deciso che giovedì il team si candiderà per la valutazione RTB.
\section{Attività future}
\subsection{Riunione Interna:}
La prossima riunione interna si terrà su Google Meet il 07/03/2024 alle 11:30.
\subsection{Riunione esterna}
È stata fissata una riunione esterna con l'azienda, per il 07/03/2024 alle 12:00, al fine di presentare il PoC per un collaudo walkthrough prima di effettuare la richiesta per l'incontro relativo alla valutazione RTB con il prof. Cardin. 

\end{document}