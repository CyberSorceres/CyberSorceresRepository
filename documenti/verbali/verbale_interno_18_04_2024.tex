\documentclass{article}
\usepackage[utf8]{inputenc}
\usepackage[includeheadfoot, margin=1em,headheight=2em]{geometry}
\usepackage{titling}
\geometry{a4paper, left=2cm, right=2cm, top=2cm, bottom=2cm}
\usepackage{graphicx}
\usepackage{enumitem}
\usepackage{array}
\newcolumntype{P}[1]{>{\centering\arraybackslash}p{#1}}
\renewcommand{\arraystretch}{1.5} % Default value: 1
\setlength{\droptitle}{-6em}
\providecommand{\versionnumber}{1.0.0}

%font
\usepackage[defaultfam,tabular,lining]{montserrat}
\usepackage[T1]{fontenc}
\renewcommand*\oldstylenums[1]{{\fontfamily{Montserrat-TOsF}\selectfont #1}}

%custom bold 
\usepackage[outline]{contour}
\usepackage{xcolor}
\newcommand{\custombold}{\contour{black}}

%table colors
\usepackage{color, colortbl}
\definecolor{Blue}{rgb}{0.51,0.68,0.79}
\definecolor{LightBlue}{rgb}{0.82,0.87,0.90}
\definecolor{LighterBlue}{rgb}{0.93,0.95,0.96}

%Header
\usepackage{fancyhdr, xcolor}
\pagestyle{fancy}
%\fancyhead[L]{\includegraphics[width=4em]{CS_tr.png}}
%\fancyhead[C]{Verbale riunione di progetto - gg/mm/aaaa}
%\fancyhead[R]{\includegraphics[width=4em]{CS_tr.png}}
\let\oldheadrule\headrule% Copy \headrule into \oldheadrule
\renewcommand{\headrule}{\color{Blue}\oldheadrule}% Add colour to \headrule
\renewcommand{\headrulewidth}{0.2em}
\fancyhead[L]{CyberSorcerers - Verbale di riunione interna}
\fancyhead[C]{}
\fancyhead[R]{18/04/2024}

\title{\Huge{\textbf{Verbale di riunione}}\vspace{-1em}}
\date{}
\begin{document}
\maketitle
\vspace{-3em}
\begin{figure}[h]
  \centering
  \includegraphics[width=6cm, height=6cm]{documenti/logo rotondo.png}
  \label{fig:immagine}
\end{figure}

\begin{center}
\Large{18 aprile 2024 - 10:00\\
Google Meet\\
Verbale redatto da: Sabrina Caniato\\}
\end{center}
\vspace{2em}
\large{
\begin{center}
    \begin{tabular}{P{24em}}
        \rowcolor{Blue}
        \textbf{Partecipanti alla riunione}\\
        \rowcolor{LightBlue}
        \custombold{Giulia Dentone}: assente\\
        \rowcolor{LighterBlue}
        \custombold{Giovanni Moretti}: presente\\
        \rowcolor{LightBlue}
        \custombold{Sabrina Caniato}: presente\\
        \rowcolor{LighterBlue}
        \custombold{Nicola Lazzarin}: presente\\
        \rowcolor{LightBlue}
        \custombold{Samuele Vignotto}: assente \\
        \rowcolor{LighterBlue}
        \custombold{Andrea Rezzi}: presente \\
    \end{tabular}
\end{center}}
\newpage
\section{Obiettivo della riunione}
L'obiettivo della riunione è quello di dividerci le attività da svolgere subito dopo la fide della RTB, di conseguenza iniziare la PB.
\section{Agenda}
\subsection{Documenti da redarre}
Vengono discussi i nuovi documenti da redarre e le tempistiche di quando inizieremo. Vengono visualizzate insieme le annotazioni ed i miglioramenti che il professore ha trovato nei documenti precedentemente consegnati.
\subsection{Analisi dei requisiti}
Discussi i cambiamenti avvenuti per il miglioramento.
\subsection{Pianificazione e Designe}
Discusso il metodo di approccio al designe delle classi e le tempistiche per il quale andrà finito.
\section{Decisioni}
\subsection{Documenti da redarre}
Abbiamo pensato di iniziare i nuovi documenti subito dopo la parte di designe, in contemporanea alla parte di programmazione del codice. Precedentemente cercheremo di apportare le modifiche richieste dal professor Vardanega per i documenti già esistenti.
\subsection{Analisi dei requisiti}
la data limite per la fine delle modifiche dell'analisi dei requisiti sarà giovedì 25 aprile.
\subsection{Pianificazione e Designe}
Abbiamo deciso una data limite di domenica 21 aprile per pensare ad un designe da presentare all'azienda. Dovremo pianificare anche una riunione con il professor Cardin.


\section{Attività future}
\subsection{Prossima riunione interna}
La prossima riunione interna si terrà il giorno 25/04/2024 alle 10:00 sulla piattaforma Google Meet.
\subsection{Prossima riunione esterna}
Lunedì 22 aprile avverrà una riunione esterna con l'azienda per discutere dei metodi di designe.
\subsection{Riunione col prof Cardin}
Verrà fissata, quando avremo una visuale più concreta del designe, una riunione con il Professor Cardin per discutere di eventuali miglioramenti.
\end{document}
